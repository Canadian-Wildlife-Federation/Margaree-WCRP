% Options for packages loaded elsewhere
% Options for packages loaded elsewhere
\PassOptionsToPackage{unicode}{hyperref}
\PassOptionsToPackage{hyphens}{url}
\PassOptionsToPackage{dvipsnames,svgnames,x11names}{xcolor}
%
\documentclass[
  letterpaper,
  DIV=11,
  numbers=noendperiod]{scrreprt}
\usepackage{xcolor}
\usepackage{amsmath,amssymb}
\setcounter{secnumdepth}{5}
\usepackage{iftex}
\ifPDFTeX
  \usepackage[T1]{fontenc}
  \usepackage[utf8]{inputenc}
  \usepackage{textcomp} % provide euro and other symbols
\else % if luatex or xetex
  \usepackage{unicode-math} % this also loads fontspec
  \defaultfontfeatures{Scale=MatchLowercase}
  \defaultfontfeatures[\rmfamily]{Ligatures=TeX,Scale=1}
\fi
\usepackage{lmodern}
\ifPDFTeX\else
  % xetex/luatex font selection
\fi
% Use upquote if available, for straight quotes in verbatim environments
\IfFileExists{upquote.sty}{\usepackage{upquote}}{}
\IfFileExists{microtype.sty}{% use microtype if available
  \usepackage[]{microtype}
  \UseMicrotypeSet[protrusion]{basicmath} % disable protrusion for tt fonts
}{}
\makeatletter
\@ifundefined{KOMAClassName}{% if non-KOMA class
  \IfFileExists{parskip.sty}{%
    \usepackage{parskip}
  }{% else
    \setlength{\parindent}{0pt}
    \setlength{\parskip}{6pt plus 2pt minus 1pt}}
}{% if KOMA class
  \KOMAoptions{parskip=half}}
\makeatother
% Make \paragraph and \subparagraph free-standing
\makeatletter
\ifx\paragraph\undefined\else
  \let\oldparagraph\paragraph
  \renewcommand{\paragraph}{
    \@ifstar
      \xxxParagraphStar
      \xxxParagraphNoStar
  }
  \newcommand{\xxxParagraphStar}[1]{\oldparagraph*{#1}\mbox{}}
  \newcommand{\xxxParagraphNoStar}[1]{\oldparagraph{#1}\mbox{}}
\fi
\ifx\subparagraph\undefined\else
  \let\oldsubparagraph\subparagraph
  \renewcommand{\subparagraph}{
    \@ifstar
      \xxxSubParagraphStar
      \xxxSubParagraphNoStar
  }
  \newcommand{\xxxSubParagraphStar}[1]{\oldsubparagraph*{#1}\mbox{}}
  \newcommand{\xxxSubParagraphNoStar}[1]{\oldsubparagraph{#1}\mbox{}}
\fi
\makeatother


\usepackage{longtable,booktabs,array}
\usepackage{calc} % for calculating minipage widths
% Correct order of tables after \paragraph or \subparagraph
\usepackage{etoolbox}
\makeatletter
\patchcmd\longtable{\par}{\if@noskipsec\mbox{}\fi\par}{}{}
\makeatother
% Allow footnotes in longtable head/foot
\IfFileExists{footnotehyper.sty}{\usepackage{footnotehyper}}{\usepackage{footnote}}
\makesavenoteenv{longtable}
\usepackage{graphicx}
\makeatletter
\newsavebox\pandoc@box
\newcommand*\pandocbounded[1]{% scales image to fit in text height/width
  \sbox\pandoc@box{#1}%
  \Gscale@div\@tempa{\textheight}{\dimexpr\ht\pandoc@box+\dp\pandoc@box\relax}%
  \Gscale@div\@tempb{\linewidth}{\wd\pandoc@box}%
  \ifdim\@tempb\p@<\@tempa\p@\let\@tempa\@tempb\fi% select the smaller of both
  \ifdim\@tempa\p@<\p@\scalebox{\@tempa}{\usebox\pandoc@box}%
  \else\usebox{\pandoc@box}%
  \fi%
}
% Set default figure placement to htbp
\def\fps@figure{htbp}
\makeatother


% definitions for citeproc citations
\NewDocumentCommand\citeproctext{}{}
\NewDocumentCommand\citeproc{mm}{%
  \begingroup\def\citeproctext{#2}\cite{#1}\endgroup}
\makeatletter
 % allow citations to break across lines
 \let\@cite@ofmt\@firstofone
 % avoid brackets around text for \cite:
 \def\@biblabel#1{}
 \def\@cite#1#2{{#1\if@tempswa , #2\fi}}
\makeatother
\newlength{\cslhangindent}
\setlength{\cslhangindent}{1.5em}
\newlength{\csllabelwidth}
\setlength{\csllabelwidth}{3em}
\newenvironment{CSLReferences}[2] % #1 hanging-indent, #2 entry-spacing
 {\begin{list}{}{%
  \setlength{\itemindent}{0pt}
  \setlength{\leftmargin}{0pt}
  \setlength{\parsep}{0pt}
  % turn on hanging indent if param 1 is 1
  \ifodd #1
   \setlength{\leftmargin}{\cslhangindent}
   \setlength{\itemindent}{-1\cslhangindent}
  \fi
  % set entry spacing
  \setlength{\itemsep}{#2\baselineskip}}}
 {\end{list}}
\usepackage{calc}
\newcommand{\CSLBlock}[1]{\hfill\break\parbox[t]{\linewidth}{\strut\ignorespaces#1\strut}}
\newcommand{\CSLLeftMargin}[1]{\parbox[t]{\csllabelwidth}{\strut#1\strut}}
\newcommand{\CSLRightInline}[1]{\parbox[t]{\linewidth - \csllabelwidth}{\strut#1\strut}}
\newcommand{\CSLIndent}[1]{\hspace{\cslhangindent}#1}



\setlength{\emergencystretch}{3em} % prevent overfull lines

\providecommand{\tightlist}{%
  \setlength{\itemsep}{0pt}\setlength{\parskip}{0pt}}



 


\usepackage{fontspec}
\usepackage{multirow}
\usepackage{multicol}
\usepackage{colortbl}
\usepackage{hhline}
\newlength\Oldarrayrulewidth
\newlength\Oldtabcolsep
\usepackage{longtable}
\usepackage{array}
\usepackage{hyperref}
\usepackage{float}
\usepackage{wrapfig}
\KOMAoption{captions}{tableheading}
\makeatletter
\@ifpackageloaded{bookmark}{}{\usepackage{bookmark}}
\makeatother
\makeatletter
\@ifpackageloaded{caption}{}{\usepackage{caption}}
\AtBeginDocument{%
\ifdefined\contentsname
  \renewcommand*\contentsname{Table of contents}
\else
  \newcommand\contentsname{Table of contents}
\fi
\ifdefined\listfigurename
  \renewcommand*\listfigurename{List of Figures}
\else
  \newcommand\listfigurename{List of Figures}
\fi
\ifdefined\listtablename
  \renewcommand*\listtablename{List of Tables}
\else
  \newcommand\listtablename{List of Tables}
\fi
\ifdefined\figurename
  \renewcommand*\figurename{Figure}
\else
  \newcommand\figurename{Figure}
\fi
\ifdefined\tablename
  \renewcommand*\tablename{Table}
\else
  \newcommand\tablename{Table}
\fi
}
\@ifpackageloaded{float}{}{\usepackage{float}}
\floatstyle{ruled}
\@ifundefined{c@chapter}{\newfloat{codelisting}{h}{lop}}{\newfloat{codelisting}{h}{lop}[chapter]}
\floatname{codelisting}{Listing}
\newcommand*\listoflistings{\listof{codelisting}{List of Listings}}
\makeatother
\makeatletter
\makeatother
\makeatletter
\@ifpackageloaded{caption}{}{\usepackage{caption}}
\@ifpackageloaded{subcaption}{}{\usepackage{subcaption}}
\makeatother
\usepackage{bookmark}
\IfFileExists{xurl.sty}{\usepackage{xurl}}{} % add URL line breaks if available
\urlstyle{same}
\hypersetup{
  pdftitle={Margaree River Connectivity Restoration Plan: 2024 - 2035},
  colorlinks=true,
  linkcolor={blue},
  filecolor={Maroon},
  citecolor={Blue},
  urlcolor={Blue},
  pdfcreator={LaTeX via pandoc}}


\title{Margaree River Connectivity Restoration Plan: 2024 - 2035}
\author{}
\date{}
\begin{document}
\maketitle

\renewcommand*\contentsname{Table of contents}
{
\hypersetup{linkcolor=}
\setcounter{tocdepth}{1}
\tableofcontents
}

\bookmarksetup{startatroot}

\chapter*{Project Overview}\label{project-overview}
\addcontentsline{toc}{chapter}{Project Overview}

\markboth{Project Overview}{Project Overview}

\section*{Purpose}\label{purpose}
\addcontentsline{toc}{section}{Purpose}

\markright{Purpose}

The purpose of the Watershed Connectivity Restoration Plan (WCRP) is to
systematically identify and assess potential barriers to Atlantic salmon
migration in the Margaree Watershed. By conducting both desktop and
field assessments, the project aims to evaluate the extent of migration
barriers caused by manmade structures and identify opportunities for
increasing connectivity for salmonid species. This assessment will guide
the prioritization of barriers for remediation, ensuring that the most
impactful sites are addressed first to enhance Atlantic salmon habitat
and improve migration conditions.

\section*{Connectivity Status}\label{connectivity-status}
\addcontentsline{toc}{section}{Connectivity Status}

\markright{Connectivity Status}

In the Margaree watershed, 358 structures potentially disconnect
Atlantic Salmon habitat. Of these, 4 are identified as barriers in need
of rehabilitation (priority barriers), \# barriers that do not warrant
rehabilitation (non-actionable), and 266 require further field
assessment.

In the Margaree Watershed, 270 structures potentially disconnect
Atlantic Salmon habitat. Of these, 3 have been reconnected through
debris jam removal, 1 has been identified as an actionable barrier and
266 require further field assessment.

\section*{Habitat Accumulation Curve}\label{habitat-accumulation-curve}
\addcontentsline{toc}{section}{Habitat Accumulation Curve}

\markright{Habitat Accumulation Curve}

\begin{figure}

\centering{

\pandocbounded{\includegraphics[keepaspectratio]{content/images/HAC-marg.png}}

}

\caption{\label{fig-geoscope}Habitat Accumulation Curve (HAC) showing
structures in the Margaree watershed as of February 5th, 2026.
Structures are ranked based on how much (km) habitat is upstream to the
next structure or set of structures and that would be reconnected if the
structure(s) became passable. Habitat upstream of data-deficient
structures is considered disconnected until field assessments are
completed. Structures that were excluded as passable are not shown;
however, rehabilitated barriers are shown to demonstrate the
connectivity gains achieved through implementation of this plan.}

\end{figure}%

\section*{Map}\label{map}
\addcontentsline{toc}{section}{Map}

\markright{Map}

\begin{figure}

\centering{

\pandocbounded{\includegraphics[keepaspectratio]{content/images/geo-scope-marg.png}}

}

\caption{\label{fig-hac}Map of focal species habitat and structures that
are confirmed or potential barriers to fish passage in the Margaree
watershed as of February 5th, 2026. Structure data were obtained from
the Canadian Aquatic Barriers Database (aquaticbarriers.ca). The
accessibility model represents areas of the watershed that focal species
could access naturally in the absence of anthropogenic barriers. The
habitat model represents the subset of accessible areas that may be used
by focal species for key habitat function(s). Available local knowledge
and data were incorporated and overrule habitat model results. Thick
light-blue lines represent habitat that is currently connected, whereas
thick red lines represent habitat considered to be disconnected because
it is upstream of confirmed barriers or data-deficient structures. The
y-axis represents the amount of habitat (km) blocked by each structure,
discounted by 50\% for first-order streams, and by passability (25\%,
50\%, 0r 75\%) of partial barriers. Barriers that were rehabilitated
through implementation of this plan are shown, but other excluded
structures (e.g., those found to be passable) are not.}

\end{figure}%

\section*{Progress Summary}\label{progress-summary}
\addcontentsline{toc}{section}{Progress Summary}

\markright{Progress Summary}

The planning process for the Margaree River Connectivity Rehabilitation
Plan began in 2024. Since that time, 39 structures have been assessed to
determine their fish passage status, and habitat confirmations are
planned for the 2026 field season. In total, 3 barriers have been
rehabilitated through the removal of debris jams and one actionable
barrier has been identified for the 2026 field season. The removal of
debris jams at three sites has resulted in improved or restored access
to approximately 30.25 km of habitat for focal species.

\bookmarksetup{startatroot}

\chapter{Vision and Scope}\label{vision-and-scope}

\section*{Purpose}\label{purpose-1}
\addcontentsline{toc}{section}{Purpose}

\markright{Purpose}

The following Watershed Connectivity Restoration Plan (WCRP) represents
the culmination of a collaborative planning effort, for the Margaree
watershed, the overall aim of which is to build collaborative
partnerships focusing on the health of the Margaree watershed to reduce
the threat of aquatic barriers to migratory fish and the livelihoods
that they support. This plan was developed to identify priority
strategies that the Margaree River WCRP planning team (see Table 1)
proposes to undertake between 2024-2035 to conserve and restore fish
passage in the watershed, through crossing rehabilitation and barrier
prevention strategies.

WCRPs are long-term, actionable plans that blend local stakeholder and
rightsholder knowledge with innovative geographical information systems
(GIS) analyses to gain a shared understanding of where restoration
efforts will have the greatest benefit for migratory fish. The planning
process is inspired by the Conservation Standards (v.4.0), which is a
conservation planning framework that allows planning teams to
systematically identify, implement, and monitor strategies to apply the
most effective solutions to high priority conservation problems. There
is a rich history of fish and fish habitat conservation and restoration
work in the Margaree watershed that this WCRP builds upon and aims to
complement over the length of the plan. This includes work undertaken by
the Margaree Salmon Association and work supported by the Nova Scotia
Salmon Association (NSSA) and the Atlantic Salmon Federation (ASF). The
planning team will aim to work with the Nova Scotia Department of
Transportation and other local stakeholders to promote coordination,
decision-making, and implementation related to this plan.

The planning team compiled existing structure location and assessment
data, habitat data, and previously identified priorities in the
watershed, and combined this with local knowledge to create a strategic
watershed-scale plan to improve freshwater connectivity. The Margaree
River WCRP planning team applied the WCRP planning framework to define
the ``thematic'' scope of freshwater connectivity and refine the
``geographic'' scope to identify the portions of the watershed where
connectivity restoration efforts will take place. Additionally, the team
selected focal species, assessed the current key habitat connectivity
status of the watershed, defined concrete goals for gains in
connectivity, undertook an iterative structure-ranking process to
identify a list of priority barriers for rehabilitation to achieve those
goals (see Table 2.1). Although the current version of this plan is
based on the best-available information at the time of publishing, WCRPs
are intended to be ``living plans'' that are updated regularly as new
information becomes available, or if local priorities and contexts
change. As such, this document should be interpreted as a current
``snapshot'' in time, and future iterations of this WCRP will build upon
the results presented in this plan to continuously improve the practice
of aquatic barrier rehabilitation for migratory fishes in the Margaree
watershed. For more information on how WCRPs are developed, see
Mazany-Wright et al. (2021).

\section*{Vision}\label{vision}
\addcontentsline{toc}{section}{Vision}

\markright{Vision}

Healthy, well-connected stream networks within the Margaree watershed
support thriving populations of migratory fish. Both residents and
visitors to the watershed work together to mitigate the negative effects
of aquatic barriers, improving the resiliency of the stream network for
the benefit and appreciation of all.

\section*{Scope}\label{scope}
\addcontentsline{toc}{section}{Scope}

\markright{Scope}

\pandocbounded{\includegraphics[keepaspectratio]{content/images/geo-scope-marg.png}}
The geographic scope of this WCRP was further refined by identifying
naturally accessible waterbodies, which are defined as streams, lakes,
or reservoirs that focal species would access in the absence of
anthropogenic barriers. Naturally accessible waterbodies were spatially
delineated for each species using stream characteristics that define the
upper limit of their movement based on their swimming abilities
(Table~\ref{tbl-targspec}). The spatial extent of the naturally
accessible waterbodies layer was then refined based on existing fish
observation data and/or redd surveys (for Atlantic salmon). These maps
were explored by the planning team to incorporate additional local
knowledge, ensure accuracy, and finalize the criteria used to define
naturally accessible waterbodies, which are shown for Atlantic salmon
(\{numref\}fig2). The new geographic scope formed the foundation for all
subsequent analyses and planning steps, including mapping and modelling
key habitat, quantifying the current connectivity status, goal setting,
and action planning, see Mazany-Wright et al. (2021).

\section*{Focal Species}\label{focal-species}
\addcontentsline{toc}{section}{Focal Species}

\markright{Focal Species}

Focal species represent the ecologically and culturally important
species for which habitat connectivity is being conserved or restored in
the watershed. In the Margaree watershed, the planning team selected
Atlantic salmon as a primary focus with interest in incorporating
American eel in future work. The selection of these focal species was
driven primarily by the focal species of the primary funds supporting
this planning work, the rich angling history of the Margaree, and the
current threats to Atlantic salmon in the region.

\subsection*{Atlantic Salmon}\label{atlantic-salmon}
\addcontentsline{toc}{subsection}{Atlantic Salmon}

Atlantic salmon are anadromous fishes, meaning mature adults spawn in
freshwater and juveniles rear in freshwater before undergoing a process
called smoltification where they migrate out to the ocean for 1-3+ years
before returning to the freshwater to repeat this process. Due to this
migratory process, salmon require unimpeded access between the ocean and
freshwater habitats to complete their life cycle.

Atlantic salmon spawn in the pool-riffle transition zones of the main
stem and larger tributaries of a river \{cite\}deGaudemar2000;
Finstad2010. Within these stretches, they seek out the tails of pools,
where substrate size allows for females to excavate a space to deposit
the eggs to incubate below gravel beds. Fast flowing, well oxygenated
water with minimal fine sediments helps to maximize successful embryo
development. Atlantic salmon juveniles (fry and parr) rear in habitat
that is deeper and slower, to facilitate their weaker swimming and
predation capabilities. Water depth, velocity, and substrate size tend
to increase with parr as they increase in size and age. Shelter, in the
form of large rocks, boulders, or overhead cover provide crucial refuge
for overwintering habitat. Channel gradients most typically associated
with these preferred spawning and rearing habitats range from 0.12 to
25\%; with highest concentrations below 3\% \{cite\}Amiro1993.

\subsection*{American Eel}\label{american-eel}
\addcontentsline{toc}{subsection}{American Eel}

American eel are catadromous species, meaning they spawn in the Sargasso
Sea and drift with the ocean currents towards the continental shelf as
glass eels. Elvers then continue migrating towards freshwater
tributaries to feed and mature, although some populations have been
known to stay in the estuaries and bays or move between freshwater and
estuary habitat over their life cycle. Their ability to tolerate a wide
variety of salinities and temperature thresholds allows them to occupy a
variety of habitat types. See Appendix B for maps of modelled eel
habitat in the seven northeastern watersheds.

\section*{Structure Types}\label{structure-types}
\addcontentsline{toc}{section}{Structure Types}

\markright{Structure Types}

The following table highlights which barrier types pose the greatest
threat to in the watershed. The results of this assessment were used to
inform the subsequent planning steps, as well as to identify knowledge
gaps where there is little spatial data to inform the assessment for a
specific barrier type.

\begin{longtable}[]{@{}lllll@{}}

\caption{\label{tbl-barriertype}Structure types in the Margaree
watershed and barrier rating assessment results. For each structure type
listed `extent' refers to the proportion of key habitat for Atlantic
salmon that is being blocked by that structure type, `severity' is the
proportion of structures for each structure type that are known to block
passage for Atlantic salmon based on field assessment, and
`irreversibility' is the degree to which the effects of a structure type
can be reversed and connectivity restored. Key habitat for the purpose
of this exercise is the total amount of spawning and rearing habitat
represented within the Margaree watershed model.}

\tabularnewline

\caption{}\label{T_140b5}\tabularnewline
\toprule\noalign{}
Barrier Types & Extent & Severity & Irreversibility & Overall Threat
Rating: \\
\midrule\noalign{}
\endfirsthead
\toprule\noalign{}
Barrier Types & Extent & Severity & Irreversibility & Overall Threat
Rating: \\
\midrule\noalign{}
\endhead
\bottomrule\noalign{}
\endlastfoot
Small Dams (\textless5m Height) & Low & Low & Medium & Low \\
Natural Barriers & Medium & Low & Medium & Low \\
Road Stream Crossings & High & Very High & Low & High \\

\end{longtable}

\bookmarksetup{startatroot}

\chapter*{Glossary}\label{glossary}
\addcontentsline{toc}{chapter}{Glossary}

\markboth{Glossary}{Glossary}

\subsection*{Freshwater connectivity:}\label{freshwater-connectivity}
\addcontentsline{toc}{subsection}{Freshwater connectivity:}

The degree to which aquatic species can disperse and/or migrate freely
through freshwater systems.

\subsection*{Longitudinal
connectivity:}\label{longitudinal-connectivity}
\addcontentsline{toc}{subsection}{Longitudinal connectivity:}

Connectivity of a stream along the upstream-downstream plane, including
access to tributaries and spawning and rearing habitat. Longitudinal
connectivity can be fragmented by physical barriers (e.g.,
anthropogenic, or natural structures) or by physiological limits of
distribution for species (e.g., stream gradient, temperature, or flow
requirements).

\subsection*{Lateral connectivity:}\label{lateral-connectivity}
\addcontentsline{toc}{subsection}{Lateral connectivity:}

Connectivity of a stream bed to adjacent riparian wetlands, floodplains,
side-channels, or off-channel features, including access to rearing and
overwintering habitat. Lateral connectivity can be fragmented by
physical barriers, channelization, armoring of the stream bed, or
artificial flow regulation.

\subsection*{Vertical connectivity:}\label{vertical-connectivity}
\addcontentsline{toc}{subsection}{Vertical connectivity:}

Connectivity of a stream bed to groundwater/hyporheic zone, including
access to oxygen-rich water and temperature refugia. Vertical
connectivity can be fragmented by water withdrawals and
anthropogenically induced changes to the hydrological, thermal, and
sediment regimes of the watershed.

\subsection*{Temporal connectivity:}\label{temporal-connectivity}
\addcontentsline{toc}{subsection}{Temporal connectivity:}

Connectivity variability in any of the three spatial dimensions
(longitudinal, lateral, vertical) based on changes in the natural flow
regime over time. Variation in the temporal connectivity occurs
naturally; however, fragmentation can be exacerbated through
anthropogenically induced changes to the hydrological, thermal, and
sediment regimes of the watershed.

\subsection*{Geographic Scope:}\label{geographic-scope}
\addcontentsline{toc}{subsection}{Geographic Scope:}

The spatial bounds within which connectivity models are applied and
fish-passage improvement efforts are focused. This area is refined after
the accessibility model is developed to cover only streams that are
naturally accessible to each focal species or guild (see naturally
accessible streams below).

\subsection*{Focal Species:}\label{focal-species-1}
\addcontentsline{toc}{subsection}{Focal Species:}

The ecologically and culturally important species (e.g., Chinook Salmon)
for which habitat connectivity is being conserved and/or restored in the
watershed.

\subsection*{Focal Guild:}\label{focal-guild}
\addcontentsline{toc}{subsection}{Focal Guild:}

Two or more focal species grouped together by shared characteristics
(e.g., migratory behavior, swimming ability, and habitat use).

\subsection*{Key Ecological Attribute
(KEA):}\label{key-ecological-attribute-kea}
\addcontentsline{toc}{subsection}{Key Ecological Attribute (KEA):}

An aspect of a focal species or guild's biology or ecology that, if
missing or altered, would lead to the loss of that focal species over
time (e.g., proportion of connected spawning habitat).

\subsection*{Structure:}\label{structure}
\addcontentsline{toc}{subsection}{Structure:}

A feature, either anthropogenic (e.g., culvert, dam) or a temporary
natural feature (e.g., barrier beach, debris jam) that may fragment
freshwater systems and limit the ability of species to disperse and/or
migrate freely within their naturally accessible habitat. Differentiated
from more permanent natural features, such as waterfalls, that delineate
naturally accessible waterbodies.

\subsection*{Set of structures:}\label{set-of-structures}
\addcontentsline{toc}{subsection}{Set of structures:}

A set of two or more structures that must be addressed/considered
together for restoration or rehabilitation when considering the total
amount of connectivity to be gained during barrier prioritization.

\subsection*{Passability status:}\label{passability-status}
\addcontentsline{toc}{subsection}{Passability status:}

A categorization of how passable a structure is (unknown, partially
passable, passable, impassable) to the focal species or guild. Note:
this status may differ from a passability status defined in another
database that reflects passability to non-focal species or uses other
terms such as ``potentially passable''.

\subsection*{Barrier:}\label{barrier}
\addcontentsline{toc}{subsection}{Barrier:}

A structure that has been assessed as impassable to the focal species
for all life stages.

\subsection*{Partial barrier:}\label{partial-barrier}
\addcontentsline{toc}{subsection}{Partial barrier:}

A structure that differs in passability status (impassable and passable)
across life stages or through time (e.g., seasonally).

\subsection*{Passable:}\label{passable}
\addcontentsline{toc}{subsection}{Passable:}

A structure that is fully passable to the focal species for all life
stages.

\subsection*{Unknown:}\label{unknown}
\addcontentsline{toc}{subsection}{Unknown:}

A structure for which passability by the focal species is unknown. These
are typically structures that have not been assessed or have been
assessed but require further assessment. These structures are considered
barriers in connectivity models.

\subsection*{Modeled barrier:}\label{modeled-barrier}
\addcontentsline{toc}{subsection}{Modeled barrier:}

All structures assessed as barriers and partial barriers, or with
unknown status (either unassessed or assessed but needs more information
is still needed) are treated as barriers in connectivity models.

\subsection*{Confirmed barrier:}\label{confirmed-barrier}
\addcontentsline{toc}{subsection}{Confirmed barrier:}

All barriers and partial barriers with confirmed key habitat upstream.

\subsection*{Deferred barrier:}\label{deferred-barrier}
\addcontentsline{toc}{subsection}{Deferred barrier:}

Priority barrier for rehabilitation: The subset of confirmed barriers
selected for proactive rehabilitation based on modelling outcomes.
Remaining confirmed barriers may never be addressed or may be addressed
at the end of their life cycle.

\subsection*{Natural barrier:}\label{natural-barrier}
\addcontentsline{toc}{subsection}{Natural barrier:}

A permanent natural feature, such as a waterfall, stream reach with a
steep gradient, or area of subsurface flow that focal species would not
be able to pass. For diadromous species, all waterbodies upstream of
such natural barriers are considered inaccessible. Resident species may
have populations upstream and downstream of such barriers.

\subsection*{Stream network (also, hydrographic
network):}\label{stream-network-also-hydrographic-network}
\addcontentsline{toc}{subsection}{Stream network (also, hydrographic
network):}

The spatial representation of all waterbodies (streams, rivers, and
lakes) in a watershed. The stream network is the foundation for all
spatial modelling, and the accessibility and habitat models create
subsets of the stream network.

\subsection*{Naturally accessible
waterbodies:}\label{naturally-accessible-waterbodies}
\addcontentsline{toc}{subsection}{Naturally accessible waterbodies:}

The portions of the stream network that are considered likely accessible
to focal species or guilds if no anthropogenic barriers existed on the
landscape. These are typically defined as all waterbodies downstream of
natural barriers for diadromous species. For resident species,
populations may persist upstream of a natural barrier; therefore,
naturally inaccessible waterbodies are those upstream of a natural
barrier where no observations of the focal species have been recorded.

\subsection*{Current Connectivity
Status:}\label{current-connectivity-status}
\addcontentsline{toc}{subsection}{Current Connectivity Status:}

The measure of important habitat (i.e., amount, proportion, or index)
that is currently connected (i.e., unaffected by structures with a
status of unknown, partially passable, or impassable) for each focal
species and KEA. For non-diadromous species, indices such as the longest
fragment or Dendridic Connectivity Index -- potadromous (DCIp; Cote et
al.~2009).

\subsection*{Connectivity Goal:}\label{connectivity-goal}
\addcontentsline{toc}{subsection}{Connectivity Goal:}

A SMART (Specific, Measurable, Achievable, Relevant, and Time-Bound)
goal representing the desired future connectivity status of each focal
species and KEA (e.g., a percent or total km of habitat) and the time by
which it will be achieved (a target year).

\subsection*{Structure Prioritization:}\label{structure-prioritization}
\addcontentsline{toc}{subsection}{Structure Prioritization:}

An analysis that ranks structures (individually and/or in sets) by the
amount of key habitat disconnected by each individual structure.
Structures are ranked in descending order according connectivity gains
(e.g., greatest amount of linear km of important habitat) if restored or
rehabilitated. This analysis includes structures with a passability
status of ``unknown'', ``partially passable'', or ``impassable''.

\subsection*{Restoration:}\label{restoration}
\addcontentsline{toc}{subsection}{Restoration:}

Returning the ecological structure and function (e.g., natural
channel-forming processes) of a site to pre-disturbance conditions.

\subsection*{Rehabilitation:}\label{rehabilitation}
\addcontentsline{toc}{subsection}{Rehabilitation:}

The partial recovery of ecosystem function. This is typical when
improving passage by replacing one type of crossing with another that is
passable but may still affect the streambed or channel-forming
processes.

\subsection*{Rehabilitated barriers:}\label{rehabilitated-barriers}
\addcontentsline{toc}{subsection}{Rehabilitated barriers:}

Structures for which connectivity for aquatic species was restored by
altering or maintaining a structure that previously impeded movement.

\subsection*{Remediation:}\label{remediation}
\addcontentsline{toc}{subsection}{Remediation:}

A term for a form of restoration most often associated with mining,
referring to the remediation of contaminated sediments. Often used as
jargon in fish-passage discussions because it applies to both fish
passage and barriers (i.e., both can be described as ``remediated'').
Most fish-passage projects involve ecological rehabilitation, but it is
inaccurate to write that a barrier was rehabilitated. Instead, fish
passage should be described as rehabilitated, improved, or restored (if
appropriate), and barriers described as removed or addressed.

\subsection*{Excluded structures:}\label{excluded-structures}
\addcontentsline{toc}{subsection}{Excluded structures:}

Structures that were investigated during a project but were removed from
consideration or the model. Reasons for their exclusion could be that
they do not exist, are passable, or no key habitat was found upstream.

\subsection*{Assessed structures that remain data
deficient:}\label{assessed-structures-that-remain-data-deficient}
\addcontentsline{toc}{subsection}{Assessed structures that remain data
deficient:}

Structures that have had some form of assessment but need further
investigation before a decision can be made to categorize them as either
confirmed barriers, excluded structures (passable, non-existent, no
suitable upstream habitat), or rehabilitated structures.

\bookmarksetup{startatroot}

\chapter{Methods}\label{methods}

\section*{Connectivity Modelling}\label{connectivity-modelling}
\addcontentsline{toc}{section}{Connectivity Modelling}

\markright{Connectivity Modelling}

The current connectivity status was estimated using three models:

Accessibility model: Naturally accessible waterbodies are those that are
considered likely accessible to focal species if no human-made barriers
existed on the landscape. These were spatially delineated for each focal
species using natural barriers (i.e., waterfalls, gradient barriers, or
subsurface flows) that would limit upstream movement (\{numref\}Table1).

Habitat model: A subset of the naturally accessible waterbody layer was
defined as key habitat based on the KEA. The habitat model identifies
waterbodies that have a higher potential to support key habitat based on
stream characteristics like channel gradient and discharge. The habitat
model criteria can be found in (\{numref\}Table 1).

Connectivity model: A layer of known or modelled structures was overlaid
on the key habitat results. Structures with unknown passability were
treated as a full barrier until confirmed passable (100\% passable) or
partially passable (25, 50, or 75\%) by either local knowledge, desktop
review, or field assessment. Watershed connectivity was estimated by
calculating the amount of key habitat that is connected to the ocean
(i.e., not fragmented by human-made barriers). The amount of key habitat
considered connected was determined by the cumulative passability of all
downstream barriers. Key habitat with only passable structures
downstream was considered fully connected. All key habitat with any full
barrier downstream was considered disconnected. Connected key habitat
upstream of partial barriers was weighted based on the passability
values of any downstream partial barriers. For example, a 10 km habitat
patch with two downstream partial barriers (both with 50\% passability)
was considered to represent 2.5 km of connected key habitat. All
connected habitats were summed and divided by the total amount of key
habitat in the watershed to determine the proportion of connected
habitat relative to the entire watershed.

A primary outcome of the WCRP is addressing barriers to restore
connectivity within the Margaree River watershed. To achieve the goals
set out in this plan it is necessary to identify a series of barriers
that if made passable would reconnect a large contribution of Atlantic
salmon and American eel habitat.

After all existing data and knowledge are collated for known and
modelled crossings, an iterative ranking process is conducted to help
confirm barriers to target for rehabilitation to meet the goals. The
ranking process is primarily used to guide field assessments and
maximize efficiency in ground truthing data/knowledge inputs and model
outputs, while providing a secondary purpose to evaluate the relative
key habitat gains of confirmed barriers in the watershed. Field
assessments can include an assessment of either the passability status
of a structure (whether fish can pass upstream, and to what degree),
whether the upstream habitat is suitable for the focal species, and
whether there are other undocumented anthropogenic or natural barriers
upstream or downstream. First, structures are grouped into `sets'. Sets
are identified by maximizing the key habitat gain for barriers in the
same tributary system. If adding a structure to a set reduces the
gain-per-barrier (i.e., the total habitat gain of the set divided by the
number of barriers in the set), then it is excluded and will be
considered part of another set. Each set is ranked by its potential to
contribute to restoring the overall habitat connectivity. The higher a
set is ranked, the more potential contribution it makes to restoring the
overall habitat connectivity, therefore making the set a higher priority
for field assessment. By assessing the passability status and upstream
habitat of the top-ranked sets, we close knowledge gaps with the
greatest influence on the overall connectivity model results.

Sets are ranked by a combination of two factors:

\begin{enumerate}
\def\labelenumi{\arabic{enumi})}
\item
  The long-term potential impact that restoring connectivity at that
  structure could have on the overall connectivity status. This was
  measured by calculating and ranking sets by the amount of total
  upstream key habitat (i.e., ignoring any additional upstream sets).
  This ranking identifies sets that would have the greatest long-term
  potential gain in habitat connectivity once any subsequent upstream
  barriers are addressed.
\item
  The immediate potential impact that restoring connectivity at that
  structure could have on the overall connectivity status. This was
  measured by calculating the amount of functional upstream habitat
  (i.e., key habitat between that set and the next upstream set of
  structures). Sets were then ranked by functional upstream habitat
  amount in tiers, where sets with no downstream sets were ranked first,
  then sets with one downstream set, and so on. This ranking identifies
  sets that have the greatest immediate potential gain in habitat by
  prioritizing sets that don't rely on rehabilitation of downstream or
  upstream sets to realize these gains.
\end{enumerate}

An overall ranking for each set was produced as a composite of the
long-term and immediate potential gain ranks to prioritize sets that
maximize long-term and immediate potential to improve key habitat
connectivity in the watershed. All structures in the watershed
(excluding those confirmed as passable) were ranked and a subset of
those structures were selected for field assessment.

\subsection*{Assessments}\label{assessments}
\addcontentsline{toc}{subsection}{Assessments}

The majority of crossing structures in the Margaree River watershed were
stream crossings for roads and trails making them the most abundant
structure type in the watershed. The Margaree River watershed is also
home to heavy beaver activity which can impact the flow and behaviour of
streams within the watershed. Rapid assessment and ground truthing was
completed for many of the structures within the watershed with formal
assessments completed at three sites. Subsequent planning was done in
partnership with the Nova Scotia Salmon Association to start the process
of restoring structures by applying for applicable permits and beginning
the restoration planning process.

\section*{Structure Ranking}\label{structure-ranking}
\addcontentsline{toc}{section}{Structure Ranking}

\markright{Structure Ranking}

\bookmarksetup{startatroot}

\chapter{Connectivity Status and
Ranking}\label{connectivity-status-and-ranking}

\section*{Current Connectivity
Status}\label{current-connectivity-status-1}
\addcontentsline{toc}{section}{Current Connectivity Status}

\markright{Current Connectivity Status}

::: \{\#tbl-goals .cell tbl-cap=' Goals to improve (1) spawning, and (2)
rearing habitat connectivity for focal species in the Margaree watershed
over the lifespan of the WCRP (2024-2035). The goals were established
through discussions with the planning team and represent the resulting
desired state of connectivity in the watershed. The goals are subject to
change as more information and data are collected over the course of the
plan timeline (e.g., the current connectivity status is updated based on
barrier field assessments).'\} ::: \{.cell-output-display\}

\begin{longtable}[]{@{}ll@{}}
\caption{}\label{T_06a15}\tabularnewline
\toprule\noalign{}
Goal \# & Goal \\
\midrule\noalign{}
\endfirsthead
\toprule\noalign{}
Goal \# & Goal \\
\midrule\noalign{}
\endhead
\bottomrule\noalign{}
\endlastfoot
1 & By 2035, increase the amount of connected habitat within the
Margaree watershed. \\
\end{longtable}

::: :::

\section*{Ranked Structures Table}\label{ranked-structures-table}
\addcontentsline{toc}{section}{Ranked Structures Table}

\markright{Ranked Structures Table}

\section*{Connectivity Status
Assessment}\label{connectivity-status-assessment}
\addcontentsline{toc}{section}{Connectivity Status Assessment}

\markright{Connectivity Status Assessment}

(see Table~\ref{tbl-connectivity}).

The current connectivity status assessment relies on GIS analyses to map
known and modelled barriers to fish passage, identify stream reaches
that have potential spawning and rearing habitat, estimate the
proportion of habitat that is currently accessible to target species,
and prioritize barriers for field assessment that would provide the
greatest gains in connectivity. To support a flexible prioritization
framework to identify priority barriers in the watershed, two
assumptions are made: 1,any modelled (i.e., passability status is
unknown) or partial barriers are treated as complete barriers to passage
and 2, the habitat modelling is binary, it does not assign any habitat
quality values. As such, the current connectivity status will be refined
over time as more data on habitat and barriers are collected. For more
detail on how the connectivity status assessments were conducted, see
Data Download and Methods.

\begin{longtable}[]{@{}lllllll@{}}

\caption{\label{tbl-connectivity}SAMPLE TABLE Connectivity status
assessment for spawning and rearing habitat.}

\tabularnewline

\caption{}\label{T_eb30e}\tabularnewline
\toprule\noalign{}
Target Species & KEA & Indicator & Poor & Fair & Good & Very Good \\
\midrule\noalign{}
\endfirsthead
\toprule\noalign{}
Target Species & KEA & Indicator & Poor & Fair & Good & Very Good \\
\midrule\noalign{}
\endhead
\bottomrule\noalign{}
\endlastfoot
Andromous Salmon & Available Spawning Habitat & \% of total habitat &
\textless50\% & 51-75\% & 76-90\% & \textgreater90\% \\
& & Current Status: & & & & 91 \\

\end{longtable}

\bookmarksetup{startatroot}

\chapter{Work Planning}\label{work-planning}

\section*{Workplan Table}\label{workplan-table}
\addcontentsline{toc}{section}{Workplan Table}

\markright{Workplan Table}

\section*{Operational Plan}\label{operational-plan}
\addcontentsline{toc}{section}{Operational Plan}

\markright{Operational Plan}

The operational plan represents a preliminary exercise undertaken by the
planning team to identify the potential leads, potential participants,
and estimated cost for the implementation of each action in . The table
below summarizes individuals, groups, or organizations that the planning
team felt could lead or participate in the implementation of the plan
and should be interpreted as the first step in on-going planning and
engagement to develop more detailed and sophisticated action plans for
each entry in the table. The individuals, groups, and organizations
listed under the ``Lead(s)'' or ``Potential Participants'' columns are
those that provisionally expressed interest in participating in one of
those roles or were suggested by the planning team for further
engagement (denoted in bold), for those that are not members of the
planning team. The leads, participants, and estimated costs in the
operational plan are not binding nor an official commitment of
resources, but rather provide a roadmap for future coordination and
engagement to work towards implementation of the WCRP.

\begin{longtable}[]{@{}lll@{}}

\caption{\label{tbl-opplan}SAMPLE Operational plan to support the
implementation of strategies and actions to improve connectivity for
target species in watershed.}

\tabularnewline

\caption{}\label{T_4f846}\tabularnewline
\toprule\noalign{}
Strategy / Actions & Lead(s) & Participants \\
\midrule\noalign{}
\endfirsthead
\toprule\noalign{}
Strategy / Actions & Lead(s) & Participants \\
\midrule\noalign{}
\endhead
\bottomrule\noalign{}
\endlastfoot
1.1 -- Conduct habitat assessment above and below confirmed actionable
barriers & MSA & MSA, CWF to support \\
1.2 -- Apply for permitting for actionable barriers with disconnected
confirmed habitat & MSA & NSSA \\
1.3-- Rehabilitate Stream-Road Crossings & MSA & MSA, NSSA, CWF to
support \\
2 -- Clear debris jams and identify opportunities to connect habitat
without structure restoration & MSA & \\
3-- Knowledge gap: Look into thermal refugia and lateral barriers within
the Margaree & MSA & CWF, ASF \\

\end{longtable}

\bookmarksetup{startatroot}

\chapter{Progress Reporting}\label{progress-reporting}

\section*{Progress Summary}\label{progress-summary-1}
\addcontentsline{toc}{section}{Progress Summary}

\markright{Progress Summary}

Initial field assessments were completed in the Margaree watershed where
39 of the top ranked structures were visited for rapid assessment to
determine whether the structures exist, if they were functionally
passing fish, and if further assessment would be needed to determine
fish passage status. Three debris jams were cleared during the 2025
field season restoring passage through three structures located on
Gallant river, Mill Brook, and an Unnamed brook. Full assessments were
done on 8 structures and a structure on MacSweens brook was identified
as a high priority for restoration in 2026 in consultation with the
NSSA.

\section*{Rehabilitated Barriers}\label{rehabilitated-barriers-1}
\addcontentsline{toc}{section}{Rehabilitated Barriers}

\markright{Rehabilitated Barriers}

\global\setlength{\Oldarrayrulewidth}{\arrayrulewidth}

\global\setlength{\Oldtabcolsep}{\tabcolsep}

\setlength{\tabcolsep}{0pt}

\renewcommand*{\arraystretch}{1.5}



\providecommand{\ascline}[3]{\noalign{\global\arrayrulewidth #1}\arrayrulecolor[HTML]{#2}\cline{#3}}

\begin{longtable}[c]{|p{0.44in}|p{1.58in}|p{1.08in}|p{1.70in}|p{2.54in}|p{2.54in}|p{1.38in}|p{1.52in}|p{1.10in}|p{1.53in}|p{1.71in}|p{1.37in}}

\caption{\label{tbl-keyact}Rehabilitated barriers}

\tabularnewline

\hhline{>{\arrayrulecolor[HTML]{666666}\global\arrayrulewidth=1.5pt}->{\arrayrulecolor[HTML]{666666}\global\arrayrulewidth=1.5pt}->{\arrayrulecolor[HTML]{666666}\global\arrayrulewidth=1.5pt}->{\arrayrulecolor[HTML]{666666}\global\arrayrulewidth=1.5pt}->{\arrayrulecolor[HTML]{666666}\global\arrayrulewidth=1.5pt}->{\arrayrulecolor[HTML]{666666}\global\arrayrulewidth=1.5pt}->{\arrayrulecolor[HTML]{666666}\global\arrayrulewidth=1.5pt}->{\arrayrulecolor[HTML]{666666}\global\arrayrulewidth=1.5pt}->{\arrayrulecolor[HTML]{666666}\global\arrayrulewidth=1.5pt}->{\arrayrulecolor[HTML]{666666}\global\arrayrulewidth=1.5pt}->{\arrayrulecolor[HTML]{666666}\global\arrayrulewidth=1.5pt}->{\arrayrulecolor[HTML]{666666}\global\arrayrulewidth=1.5pt}-}

\multicolumn{1}{>{\cellcolor[HTML]{008270}\raggedleft}m{\dimexpr 0.44in+0\tabcolsep}}{\textcolor[HTML]{FFFFFF}{\fontsize{11}{11}\selectfont{ID}}} & \multicolumn{1}{>{\cellcolor[HTML]{008270}\raggedleft}m{\dimexpr 1.58in+0\tabcolsep}}{\textcolor[HTML]{FFFFFF}{\fontsize{11}{11}\selectfont{Watercourse\ name}}} & \multicolumn{1}{>{\cellcolor[HTML]{008270}\raggedleft}m{\dimexpr 1.08in+0\tabcolsep}}{\textcolor[HTML]{FFFFFF}{\fontsize{11}{11}\selectfont{Road\ name}}} & \multicolumn{1}{>{\cellcolor[HTML]{008270}\raggedleft}m{\dimexpr 1.7in+0\tabcolsep}}{\textcolor[HTML]{FFFFFF}{\fontsize{11}{11}\selectfont{Location/coordinates}}} & \multicolumn{1}{>{\cellcolor[HTML]{008270}\raggedleft}m{\dimexpr 2.54in+0\tabcolsep}}{\textcolor[HTML]{FFFFFF}{\fontsize{11}{11}\selectfont{Type\ of\ rehabilitation\ -\ cover\ new}}} & \multicolumn{1}{>{\cellcolor[HTML]{008270}\raggedleft}m{\dimexpr 2.54in+0\tabcolsep}}{\textcolor[HTML]{FFFFFF}{\fontsize{11}{11}\selectfont{Structure\ type\ as\ categories\ here}}} & \multicolumn{1}{>{\cellcolor[HTML]{008270}\raggedleft}m{\dimexpr 1.38in+0\tabcolsep}}{\textcolor[HTML]{FFFFFF}{\fontsize{11}{11}\selectfont{Rehabilitated\ by}}} & \multicolumn{1}{>{\cellcolor[HTML]{008270}\raggedleft}m{\dimexpr 1.52in+0\tabcolsep}}{\textcolor[HTML]{FFFFFF}{\fontsize{11}{11}\selectfont{Rehabilitated\ date}}} & \multicolumn{1}{>{\cellcolor[HTML]{008270}\raggedleft}m{\dimexpr 1.1in+0\tabcolsep}}{\textcolor[HTML]{FFFFFF}{\fontsize{11}{11}\selectfont{Habitat\ gain}}} & \multicolumn{1}{>{\cellcolor[HTML]{008270}\raggedleft}m{\dimexpr 1.53in+0\tabcolsep}}{\textcolor[HTML]{FFFFFF}{\fontsize{11}{11}\selectfont{Actual\ project\ cost}}} & \multicolumn{1}{>{\cellcolor[HTML]{008270}\raggedleft}m{\dimexpr 1.71in+0\tabcolsep}}{\textcolor[HTML]{FFFFFF}{\fontsize{11}{11}\selectfont{Comments\ (external)}}} & \multicolumn{1}{>{\cellcolor[HTML]{008270}\raggedleft}m{\dimexpr 1.37in+0\tabcolsep}}{\textcolor[HTML]{FFFFFF}{\fontsize{11}{11}\selectfont{Supporting\ links}}} \\

\noalign{\global\arrayrulewidth 0pt}\arrayrulecolor[HTML]{000000}

\hhline{>{\arrayrulecolor[HTML]{666666}\global\arrayrulewidth=1.5pt}->{\arrayrulecolor[HTML]{666666}\global\arrayrulewidth=1.5pt}->{\arrayrulecolor[HTML]{666666}\global\arrayrulewidth=1.5pt}->{\arrayrulecolor[HTML]{666666}\global\arrayrulewidth=1.5pt}->{\arrayrulecolor[HTML]{666666}\global\arrayrulewidth=1.5pt}->{\arrayrulecolor[HTML]{666666}\global\arrayrulewidth=1.5pt}->{\arrayrulecolor[HTML]{666666}\global\arrayrulewidth=1.5pt}->{\arrayrulecolor[HTML]{666666}\global\arrayrulewidth=1.5pt}->{\arrayrulecolor[HTML]{666666}\global\arrayrulewidth=1.5pt}->{\arrayrulecolor[HTML]{666666}\global\arrayrulewidth=1.5pt}->{\arrayrulecolor[HTML]{666666}\global\arrayrulewidth=1.5pt}->{\arrayrulecolor[HTML]{666666}\global\arrayrulewidth=1.5pt}-}\endfirsthead 

\hhline{>{\arrayrulecolor[HTML]{666666}\global\arrayrulewidth=1.5pt}->{\arrayrulecolor[HTML]{666666}\global\arrayrulewidth=1.5pt}->{\arrayrulecolor[HTML]{666666}\global\arrayrulewidth=1.5pt}->{\arrayrulecolor[HTML]{666666}\global\arrayrulewidth=1.5pt}->{\arrayrulecolor[HTML]{666666}\global\arrayrulewidth=1.5pt}->{\arrayrulecolor[HTML]{666666}\global\arrayrulewidth=1.5pt}->{\arrayrulecolor[HTML]{666666}\global\arrayrulewidth=1.5pt}->{\arrayrulecolor[HTML]{666666}\global\arrayrulewidth=1.5pt}->{\arrayrulecolor[HTML]{666666}\global\arrayrulewidth=1.5pt}->{\arrayrulecolor[HTML]{666666}\global\arrayrulewidth=1.5pt}->{\arrayrulecolor[HTML]{666666}\global\arrayrulewidth=1.5pt}->{\arrayrulecolor[HTML]{666666}\global\arrayrulewidth=1.5pt}-}

\multicolumn{1}{>{\cellcolor[HTML]{008270}\raggedleft}m{\dimexpr 0.44in+0\tabcolsep}}{\textcolor[HTML]{FFFFFF}{\fontsize{11}{11}\selectfont{ID}}} & \multicolumn{1}{>{\cellcolor[HTML]{008270}\raggedleft}m{\dimexpr 1.58in+0\tabcolsep}}{\textcolor[HTML]{FFFFFF}{\fontsize{11}{11}\selectfont{Watercourse\ name}}} & \multicolumn{1}{>{\cellcolor[HTML]{008270}\raggedleft}m{\dimexpr 1.08in+0\tabcolsep}}{\textcolor[HTML]{FFFFFF}{\fontsize{11}{11}\selectfont{Road\ name}}} & \multicolumn{1}{>{\cellcolor[HTML]{008270}\raggedleft}m{\dimexpr 1.7in+0\tabcolsep}}{\textcolor[HTML]{FFFFFF}{\fontsize{11}{11}\selectfont{Location/coordinates}}} & \multicolumn{1}{>{\cellcolor[HTML]{008270}\raggedleft}m{\dimexpr 2.54in+0\tabcolsep}}{\textcolor[HTML]{FFFFFF}{\fontsize{11}{11}\selectfont{Type\ of\ rehabilitation\ -\ cover\ new}}} & \multicolumn{1}{>{\cellcolor[HTML]{008270}\raggedleft}m{\dimexpr 2.54in+0\tabcolsep}}{\textcolor[HTML]{FFFFFF}{\fontsize{11}{11}\selectfont{Structure\ type\ as\ categories\ here}}} & \multicolumn{1}{>{\cellcolor[HTML]{008270}\raggedleft}m{\dimexpr 1.38in+0\tabcolsep}}{\textcolor[HTML]{FFFFFF}{\fontsize{11}{11}\selectfont{Rehabilitated\ by}}} & \multicolumn{1}{>{\cellcolor[HTML]{008270}\raggedleft}m{\dimexpr 1.52in+0\tabcolsep}}{\textcolor[HTML]{FFFFFF}{\fontsize{11}{11}\selectfont{Rehabilitated\ date}}} & \multicolumn{1}{>{\cellcolor[HTML]{008270}\raggedleft}m{\dimexpr 1.1in+0\tabcolsep}}{\textcolor[HTML]{FFFFFF}{\fontsize{11}{11}\selectfont{Habitat\ gain}}} & \multicolumn{1}{>{\cellcolor[HTML]{008270}\raggedleft}m{\dimexpr 1.53in+0\tabcolsep}}{\textcolor[HTML]{FFFFFF}{\fontsize{11}{11}\selectfont{Actual\ project\ cost}}} & \multicolumn{1}{>{\cellcolor[HTML]{008270}\raggedleft}m{\dimexpr 1.71in+0\tabcolsep}}{\textcolor[HTML]{FFFFFF}{\fontsize{11}{11}\selectfont{Comments\ (external)}}} & \multicolumn{1}{>{\cellcolor[HTML]{008270}\raggedleft}m{\dimexpr 1.37in+0\tabcolsep}}{\textcolor[HTML]{FFFFFF}{\fontsize{11}{11}\selectfont{Supporting\ links}}} \\

\noalign{\global\arrayrulewidth 0pt}\arrayrulecolor[HTML]{000000}

\hhline{>{\arrayrulecolor[HTML]{666666}\global\arrayrulewidth=1.5pt}->{\arrayrulecolor[HTML]{666666}\global\arrayrulewidth=1.5pt}->{\arrayrulecolor[HTML]{666666}\global\arrayrulewidth=1.5pt}->{\arrayrulecolor[HTML]{666666}\global\arrayrulewidth=1.5pt}->{\arrayrulecolor[HTML]{666666}\global\arrayrulewidth=1.5pt}->{\arrayrulecolor[HTML]{666666}\global\arrayrulewidth=1.5pt}->{\arrayrulecolor[HTML]{666666}\global\arrayrulewidth=1.5pt}->{\arrayrulecolor[HTML]{666666}\global\arrayrulewidth=1.5pt}->{\arrayrulecolor[HTML]{666666}\global\arrayrulewidth=1.5pt}->{\arrayrulecolor[HTML]{666666}\global\arrayrulewidth=1.5pt}->{\arrayrulecolor[HTML]{666666}\global\arrayrulewidth=1.5pt}->{\arrayrulecolor[HTML]{666666}\global\arrayrulewidth=1.5pt}-}\endhead


\end{longtable}

\arrayrulecolor[HTML]{000000}

\global\setlength{\arrayrulewidth}{\Oldarrayrulewidth}

\global\setlength{\tabcolsep}{\Oldtabcolsep}

\renewcommand*{\arraystretch}{1}

\section*{Annual Progress Report}\label{annual-progress-report}
\addcontentsline{toc}{section}{Annual Progress Report}

\markright{Annual Progress Report}

The Margaree River Salmon Association (MSA) completed rapid and full
barrier assessments at several sited throughout the Margaree watershed.
In depth habitat assessment and restoration planning are in progress and
are planned to continue into the 2026 field season in partnership with
the Nova Scotia Salmon Association.

\bookmarksetup{startatroot}

\chapter{Acknowledgements and
References}\label{acknowledgements-and-references}

\section*{Recommended Citation}\label{recommended-citation}
\addcontentsline{toc}{section}{Recommended Citation}

\markright{Recommended Citation}

\section*{Acknowledgements}\label{acknowledgements}
\addcontentsline{toc}{section}{Acknowledgements}

\markright{Acknowledgements}

\section*{References}\label{references}
\addcontentsline{toc}{section}{References}

\markright{References}

\phantomsection\label{refs}
\begin{CSLReferences}{1}{0}
\bibitem[\citeproctext]{ref-Mazany-Wright2021-hs}
Mazany-Wright, N, J Noseworthy, S Sra, S M Norris, and N W Lapointe.
2021. {``Breaking down Barriers: A Practitioners' Guide to Watershed
Connectivity Remediation Planning.''} \emph{Canadian Wildlife
Federation}.

\end{CSLReferences}

\bookmarksetup{startatroot}

\chapter*{Version History}\label{version-history}
\addcontentsline{toc}{chapter}{Version History}

\markboth{Version History}{Version History}

\part{Appendix A}

\chapter*{Partners}\label{partners}
\addcontentsline{toc}{chapter}{Partners}

\markboth{Partners}{Partners}

\section*{Planning Team}\label{planning-team}
\addcontentsline{toc}{section}{Planning Team}

\markright{Planning Team}

\global\setlength{\Oldarrayrulewidth}{\arrayrulewidth}

\global\setlength{\Oldtabcolsep}{\tabcolsep}

\setlength{\tabcolsep}{0pt}

\renewcommand*{\arraystretch}{1.5}



\providecommand{\ascline}[3]{\noalign{\global\arrayrulewidth #1}\arrayrulecolor[HTML]{#2}\cline{#3}}

\begin{longtable}[c]{|p{1.75in}|p{2.54in}}

\caption{\label{tbl-planteam}Margaree River WCRP planning team members.
Planning team members contributed to the development of this plan by
participating in a series of workshops and document and data review. The
plan was generated based on the input and feedback of the local groups
and organizations list in this table.}

\tabularnewline

\hhline{>{\arrayrulecolor[HTML]{666666}\global\arrayrulewidth=1.5pt}->{\arrayrulecolor[HTML]{666666}\global\arrayrulewidth=1.5pt}-}

\multicolumn{1}{>{\cellcolor[HTML]{008270}\raggedright}m{\dimexpr 1.75in+0\tabcolsep}}{\textcolor[HTML]{FFFFFF}{\fontsize{11}{11}\selectfont{Name}}} & \multicolumn{1}{>{\cellcolor[HTML]{008270}\raggedright}m{\dimexpr 2.54in+0\tabcolsep}}{\textcolor[HTML]{FFFFFF}{\fontsize{11}{11}\selectfont{Organization}}} \\

\noalign{\global\arrayrulewidth 0pt}\arrayrulecolor[HTML]{000000}

\hhline{>{\arrayrulecolor[HTML]{666666}\global\arrayrulewidth=1.5pt}->{\arrayrulecolor[HTML]{666666}\global\arrayrulewidth=1.5pt}-}\endfirsthead 

\hhline{>{\arrayrulecolor[HTML]{666666}\global\arrayrulewidth=1.5pt}->{\arrayrulecolor[HTML]{666666}\global\arrayrulewidth=1.5pt}-}

\multicolumn{1}{>{\cellcolor[HTML]{008270}\raggedright}m{\dimexpr 1.75in+0\tabcolsep}}{\textcolor[HTML]{FFFFFF}{\fontsize{11}{11}\selectfont{Name}}} & \multicolumn{1}{>{\cellcolor[HTML]{008270}\raggedright}m{\dimexpr 2.54in+0\tabcolsep}}{\textcolor[HTML]{FFFFFF}{\fontsize{11}{11}\selectfont{Organization}}} \\

\noalign{\global\arrayrulewidth 0pt}\arrayrulecolor[HTML]{000000}

\hhline{>{\arrayrulecolor[HTML]{666666}\global\arrayrulewidth=1.5pt}->{\arrayrulecolor[HTML]{666666}\global\arrayrulewidth=1.5pt}-}\endhead



\multicolumn{1}{>{\raggedright}m{\dimexpr 1.75in+0\tabcolsep}}{\textcolor[HTML]{000000}{\fontsize{11}{11}\selectfont{Micheal\ Fabiano }}} & \multicolumn{1}{>{\raggedright}m{\dimexpr 2.54in+0\tabcolsep}}{\textcolor[HTML]{000000}{\fontsize{11}{11}\selectfont{Margaree\ Salmon\ Association }}} \\

\noalign{\global\arrayrulewidth 0pt}\arrayrulecolor[HTML]{000000}





\multicolumn{1}{>{\raggedright}m{\dimexpr 1.75in+0\tabcolsep}}{\textcolor[HTML]{000000}{\fontsize{11}{11}\selectfont{Paul\ MacNeil }}} & \multicolumn{1}{>{\raggedright}m{\dimexpr 2.54in+0\tabcolsep}}{\textcolor[HTML]{000000}{\fontsize{11}{11}\selectfont{Margaree\ Salmon\ Association }}} \\

\noalign{\global\arrayrulewidth 0pt}\arrayrulecolor[HTML]{000000}





\multicolumn{1}{>{\raggedright}m{\dimexpr 1.75in+0\tabcolsep}}{\textcolor[HTML]{000000}{\fontsize{11}{11}\selectfont{Thomas\ Sweeny }}} & \multicolumn{1}{>{\raggedright}m{\dimexpr 2.54in+0\tabcolsep}}{\textcolor[HTML]{000000}{\fontsize{11}{11}\selectfont{Nova\ Scotia\ Salmon\ Association }}} \\

\noalign{\global\arrayrulewidth 0pt}\arrayrulecolor[HTML]{000000}





\multicolumn{1}{>{\raggedright}m{\dimexpr 1.75in+0\tabcolsep}}{\textcolor[HTML]{000000}{\fontsize{11}{11}\selectfont{Fielding\ Montgomery }}} & \multicolumn{1}{>{\raggedright}m{\dimexpr 2.54in+0\tabcolsep}}{\textcolor[HTML]{000000}{\fontsize{11}{11}\selectfont{Canadian\ Wildlife\ Federation }}} \\

\noalign{\global\arrayrulewidth 0pt}\arrayrulecolor[HTML]{000000}





\multicolumn{1}{>{\raggedright}m{\dimexpr 1.75in+0\tabcolsep}}{\textcolor[HTML]{000000}{\fontsize{11}{11}\selectfont{Nick\ Mazany-Wright }}} & \multicolumn{1}{>{\raggedright}m{\dimexpr 2.54in+0\tabcolsep}}{\textcolor[HTML]{000000}{\fontsize{11}{11}\selectfont{Canadian\ Wildlife\ Federation }}} \\

\noalign{\global\arrayrulewidth 0pt}\arrayrulecolor[HTML]{000000}





\multicolumn{1}{>{\raggedright}m{\dimexpr 1.75in+0\tabcolsep}}{\textcolor[HTML]{000000}{\fontsize{11}{11}\selectfont{Nicolas\ Lapointe }}} & \multicolumn{1}{>{\raggedright}m{\dimexpr 2.54in+0\tabcolsep}}{\textcolor[HTML]{000000}{\fontsize{11}{11}\selectfont{Canadian\ Wildlife\ Federation }}} \\

\noalign{\global\arrayrulewidth 0pt}\arrayrulecolor[HTML]{000000}





\multicolumn{1}{>{\raggedright}m{\dimexpr 1.75in+0\tabcolsep}}{\textcolor[HTML]{000000}{\fontsize{11}{11}\selectfont{Courtney\ Brake }}} & \multicolumn{1}{>{\raggedright}m{\dimexpr 2.54in+0\tabcolsep}}{\textcolor[HTML]{000000}{\fontsize{11}{11}\selectfont{Canadian\ Wildlife\ Federation }}} \\

\noalign{\global\arrayrulewidth 0pt}\arrayrulecolor[HTML]{000000}





\multicolumn{1}{>{\raggedright}m{\dimexpr 1.75in+0\tabcolsep}}{\textcolor[HTML]{000000}{\fontsize{11}{11}\selectfont{Kris\ Hunter }}} & \multicolumn{1}{>{\raggedright}m{\dimexpr 2.54in+0\tabcolsep}}{\textcolor[HTML]{000000}{\fontsize{11}{11}\selectfont{Atlantic\ Salmon\ Federation }}} \\

\noalign{\global\arrayrulewidth 0pt}\arrayrulecolor[HTML]{000000}

\hhline{>{\arrayrulecolor[HTML]{666666}\global\arrayrulewidth=1.5pt}->{\arrayrulecolor[HTML]{666666}\global\arrayrulewidth=1.5pt}-}


\end{longtable}

\arrayrulecolor[HTML]{000000}

\global\setlength{\arrayrulewidth}{\Oldarrayrulewidth}

\global\setlength{\tabcolsep}{\Oldtabcolsep}

\renewcommand*{\arraystretch}{1}

\section*{Key Actors}\label{key-actors}
\addcontentsline{toc}{section}{Key Actors}

\markright{Key Actors}

\global\setlength{\Oldarrayrulewidth}{\arrayrulewidth}

\global\setlength{\Oldtabcolsep}{\tabcolsep}

\setlength{\tabcolsep}{0pt}

\renewcommand*{\arraystretch}{1.5}



\providecommand{\ascline}[3]{\noalign{\global\arrayrulewidth #1}\arrayrulecolor[HTML]{#2}\cline{#3}}

\begin{longtable}[c]{|p{5.01in}|p{14.85in}}

\caption{\label{tbl-keyact}Additional key actors in the Margaree
watershed. Key actors are the individuals, groups, and/or organizations,
outside of the planning team, with influence and relevant experience in
the watershed, whose engagement will be critical for the successful
implementation of this WCRP. Key actors were identified by the planning
team and do not reflect a commitment to contribute to the implementation
and updating of this WCRP.}

\tabularnewline

\hhline{>{\arrayrulecolor[HTML]{666666}\global\arrayrulewidth=1.5pt}->{\arrayrulecolor[HTML]{666666}\global\arrayrulewidth=1.5pt}-}

\multicolumn{1}{>{\cellcolor[HTML]{008270}\raggedright}m{\dimexpr 5.01in+0\tabcolsep}}{\textcolor[HTML]{FFFFFF}{\fontsize{11}{11}\selectfont{Individual\ or\ Organization\ Name}}} & \multicolumn{1}{>{\cellcolor[HTML]{008270}\raggedright}m{\dimexpr 14.85in+0\tabcolsep}}{\textcolor[HTML]{FFFFFF}{\fontsize{11}{11}\selectfont{Role\ and\ Primary\ Interest}}} \\

\noalign{\global\arrayrulewidth 0pt}\arrayrulecolor[HTML]{000000}

\hhline{>{\arrayrulecolor[HTML]{666666}\global\arrayrulewidth=1.5pt}->{\arrayrulecolor[HTML]{666666}\global\arrayrulewidth=1.5pt}-}\endfirsthead 

\hhline{>{\arrayrulecolor[HTML]{666666}\global\arrayrulewidth=1.5pt}->{\arrayrulecolor[HTML]{666666}\global\arrayrulewidth=1.5pt}-}

\multicolumn{1}{>{\cellcolor[HTML]{008270}\raggedright}m{\dimexpr 5.01in+0\tabcolsep}}{\textcolor[HTML]{FFFFFF}{\fontsize{11}{11}\selectfont{Individual\ or\ Organization\ Name}}} & \multicolumn{1}{>{\cellcolor[HTML]{008270}\raggedright}m{\dimexpr 14.85in+0\tabcolsep}}{\textcolor[HTML]{FFFFFF}{\fontsize{11}{11}\selectfont{Role\ and\ Primary\ Interest}}} \\

\noalign{\global\arrayrulewidth 0pt}\arrayrulecolor[HTML]{000000}

\hhline{>{\arrayrulecolor[HTML]{666666}\global\arrayrulewidth=1.5pt}->{\arrayrulecolor[HTML]{666666}\global\arrayrulewidth=1.5pt}-}\endhead



\multicolumn{1}{>{\raggedright}m{\dimexpr 5.01in+0\tabcolsep}}{\textcolor[HTML]{000000}{\fontsize{11}{11}\selectfont{Cariboo\ Mining\ Association }}} & \multicolumn{1}{>{\raggedright}m{\dimexpr 14.85in+0\tabcolsep}}{\textcolor[HTML]{000000}{\fontsize{11}{11}\selectfont{A\ mining\ company\ that\ has\ been\ operating\ in\ central\ BC\ since\ the\ 1950’s\ and\ can\ help\ provide\ data\ and\ facilitate\ remediation\ work. }}} \\

\noalign{\global\arrayrulewidth 0pt}\arrayrulecolor[HTML]{000000}





\multicolumn{1}{>{\raggedright}m{\dimexpr 5.01in+0\tabcolsep}}{\textcolor[HTML]{000000}{\fontsize{11}{11}\selectfont{Consus\ Management\ Ltd.  }}} & \multicolumn{1}{>{\raggedright}m{\dimexpr 14.85in+0\tabcolsep}}{\textcolor[HTML]{000000}{\fontsize{11}{11}\selectfont{Local\ wildlife\ consultants\ in\ the\ watershed\ to\ consider\ for\ future\ work. }}} \\

\noalign{\global\arrayrulewidth 0pt}\arrayrulecolor[HTML]{000000}





\multicolumn{1}{>{\raggedright}m{\dimexpr 5.01in+0\tabcolsep}}{\textcolor[HTML]{000000}{\fontsize{11}{11}\selectfont{Dawson\ Road\ Maintenance\ Ltd }}} & \multicolumn{1}{>{\raggedright}m{\dimexpr 14.85in+0\tabcolsep}}{\textcolor[HTML]{000000}{\fontsize{11}{11}\selectfont{A\ road\ design\ and\ maintenance\ company\ at\ the\ roadway-watershed\ interface. }}} \\

\noalign{\global\arrayrulewidth 0pt}\arrayrulecolor[HTML]{000000}





\multicolumn{1}{>{\raggedright}m{\dimexpr 5.01in+0\tabcolsep}}{\textcolor[HTML]{000000}{\fontsize{11}{11}\selectfont{DWB\ Consulting\ Services\ Ltd.  }}} & \multicolumn{1}{>{\raggedright}m{\dimexpr 14.85in+0\tabcolsep}}{\textcolor[HTML]{000000}{\fontsize{11}{11}\selectfont{Local\ wildlife\ consultants\ in\ the\ watershed\ to\ consider\ for\ future\ work. }}} \\

\noalign{\global\arrayrulewidth 0pt}\arrayrulecolor[HTML]{000000}





\multicolumn{1}{>{\raggedright}m{\dimexpr 5.01in+0\tabcolsep}}{\textcolor[HTML]{000000}{\fontsize{11}{11}\selectfont{Freshwater\ Fisheries\ Society\ of\ British\ Columbia }}} & \multicolumn{1}{>{\raggedright}m{\dimexpr 14.85in+0\tabcolsep}}{\textcolor[HTML]{000000}{\fontsize{11}{11}\selectfont{This\ group\ can\ provide\ project\ assistance\ with\ non-anadromous\ species.  }}} \\

\noalign{\global\arrayrulewidth 0pt}\arrayrulecolor[HTML]{000000}





\multicolumn{1}{>{\raggedright}m{\dimexpr 5.01in+0\tabcolsep}}{\textcolor[HTML]{000000}{\fontsize{11}{11}\selectfont{Larry\ Davis  }}} & \multicolumn{1}{>{\raggedright}m{\dimexpr 14.85in+0\tabcolsep}}{\textcolor[HTML]{000000}{\fontsize{11}{11}\selectfont{A\ biologist\ and\ local\ wildlife\ consultant\ in\ the\ watershed. }}} \\

\noalign{\global\arrayrulewidth 0pt}\arrayrulecolor[HTML]{000000}





\multicolumn{1}{>{\raggedright}m{\dimexpr 5.01in+0\tabcolsep}}{\textcolor[HTML]{000000}{\fontsize{11}{11}\selectfont{Local\ ranchers }}} & \multicolumn{1}{>{\raggedright}m{\dimexpr 14.85in+0\tabcolsep}}{\textcolor[HTML]{000000}{\fontsize{11}{11}\selectfont{These\ individuals\ can\ facilitate\ construction\ as\ well\ as\ consent/facilitate\ complimentary\ works\ on\ private\ property\ to\ improve\ fish\ habitat\ upstream\ and\ downstream. }}} \\

\noalign{\global\arrayrulewidth 0pt}\arrayrulecolor[HTML]{000000}





\multicolumn{1}{>{\raggedright}m{\dimexpr 5.01in+0\tabcolsep}}{\textcolor[HTML]{000000}{\fontsize{11}{11}\selectfont{Ministry\ of\ Forests,\ Lands\ and\ Natural\ Resource\ Operations\ (FLNRO)}}} & \multicolumn{1}{>{\raggedright}m{\dimexpr 14.85in+0\tabcolsep}}{\textcolor[HTML]{000000}{\fontsize{11}{11}\selectfont{FLNRO\ can\ assist\ with\ providing\ local\ knowledge,\ data,\ expertise\ and\ can\ facilitate\ remediation\ work. }}} \\

\noalign{\global\arrayrulewidth 0pt}\arrayrulecolor[HTML]{000000}





\multicolumn{1}{>{\raggedright}m{\dimexpr 5.01in+0\tabcolsep}}{\textcolor[HTML]{000000}{\fontsize{11}{11}\selectfont{Ministry of Transportation\ and Infrastructure (MOTI)}}} & \multicolumn{1}{>{\raggedright}m{\dimexpr 14.85in+0\tabcolsep}}{\textcolor[HTML]{000000}{\fontsize{11}{11}\selectfont{MOTI\ may\ own\ barriers\ and\ can\ play\ a\ role\ in\ improving\ and\ replacing\ barriers\ at\ highway\ crossings. }}} \\

\noalign{\global\arrayrulewidth 0pt}\arrayrulecolor[HTML]{000000}





\multicolumn{1}{>{\raggedright}m{\dimexpr 5.01in+0\tabcolsep}}{\textcolor[HTML]{000000}{\fontsize{11}{11}\selectfont{Property\ owners\ along\ river\ and\ tributaries }}} & \multicolumn{1}{>{\raggedright}m{\dimexpr 14.85in+0\tabcolsep}}{\textcolor[HTML]{000000}{\fontsize{11}{11}\selectfont{These\ individuals\ can\ facilitate\ construction\ as\ well\ as\ consent/facilitate\ complimentary\ works\ on\ private\ property\ to\ improve\ fish\ habitat\ upstream\ and\ downstream. }}} \\

\noalign{\global\arrayrulewidth 0pt}\arrayrulecolor[HTML]{000000}





\multicolumn{1}{>{\raggedright}m{\dimexpr 5.01in+0\tabcolsep}}{\textcolor[HTML]{000000}{\fontsize{11}{11}\selectfont{Quesnel\ River\ Research\ Centre }}} & \multicolumn{1}{>{\raggedright}m{\dimexpr 14.85in+0\tabcolsep}}{\textcolor[HTML]{000000}{\fontsize{11}{11}\selectfont{This\ group\ can\ help\ with\ field\ assessments\ and\ project\ implementation. }}} \\

\noalign{\global\arrayrulewidth 0pt}\arrayrulecolor[HTML]{000000}





\multicolumn{1}{>{\raggedright}m{\dimexpr 5.01in+0\tabcolsep}}{\textcolor[HTML]{000000}{\fontsize{11}{11}\selectfont{Steve Hocquard }}} & \multicolumn{1}{>{\raggedright}m{\dimexpr 14.85in+0\tabcolsep}}{\textcolor[HTML]{000000}{\fontsize{11}{11}\selectfont{A\ local\ consultant\ (Steve\ Hocquard\ Consulting)\ that\ provided\ valuable\ review\ of\ barrier\ and\ habitat\ data\ to\ inform\ the\ spatial\ models\ used\ in\ this\ plan,\ and\ can\ help\ with\ field\ assessments\ and\ project\ implementation. }}} \\

\noalign{\global\arrayrulewidth 0pt}\arrayrulecolor[HTML]{000000}





\multicolumn{1}{>{\raggedright}m{\dimexpr 5.01in+0\tabcolsep}}{\textcolor[HTML]{000000}{\fontsize{11}{11}\selectfont{Tolko\ Industries\ Ltd.  }}} & \multicolumn{1}{>{\raggedright}m{\dimexpr 14.85in+0\tabcolsep}}{\textcolor[HTML]{000000}{\fontsize{11}{11}\selectfont{A\ privately\ owned\ Canadian\ forest\ products\ company\ that\ maintains\ forest\ service\ road-stream\ crossings\ in\ the\ Horsefly\ River\ watershed. }}} \\

\noalign{\global\arrayrulewidth 0pt}\arrayrulecolor[HTML]{000000}





\multicolumn{1}{>{\raggedright}m{\dimexpr 5.01in+0\tabcolsep}}{\textcolor[HTML]{000000}{\fontsize{11}{11}\selectfont{Upper\ Fraser\ Fisheries\ Conservation\ Alliance }}} & \multicolumn{1}{>{\raggedright}m{\dimexpr 14.85in+0\tabcolsep}}{\textcolor[HTML]{000000}{\fontsize{11}{11}\selectfont{This\ group\ can\ be\ contacted\ for\ advice\ and\ assistance. }}} \\

\noalign{\global\arrayrulewidth 0pt}\arrayrulecolor[HTML]{000000}





\multicolumn{1}{>{\raggedright}m{\dimexpr 5.01in+0\tabcolsep}}{\textcolor[HTML]{000000}{\fontsize{11}{11}\selectfont{West\ Fraser }}} & \multicolumn{1}{>{\raggedright}m{\dimexpr 14.85in+0\tabcolsep}}{\textcolor[HTML]{000000}{\fontsize{11}{11}\selectfont{A\ integrated\ forestry\ and\ diversified\ wood\ products\ company\ that\ maintains\ forest\ service\ road-stream\ crossings\ in\ the\ Horsefly\ River\ watershed. }}} \\

\noalign{\global\arrayrulewidth 0pt}\arrayrulecolor[HTML]{000000}

\hhline{>{\arrayrulecolor[HTML]{666666}\global\arrayrulewidth=1.5pt}->{\arrayrulecolor[HTML]{666666}\global\arrayrulewidth=1.5pt}-}


\end{longtable}

\arrayrulecolor[HTML]{000000}

\global\setlength{\arrayrulewidth}{\Oldarrayrulewidth}

\global\setlength{\tabcolsep}{\Oldtabcolsep}

\renewcommand*{\arraystretch}{1}

\chapter*{Detailed Methods}\label{detailed-methods}
\addcontentsline{toc}{chapter}{Detailed Methods}

\markboth{Detailed Methods}{Detailed Methods}

\section*{Connectivity Status Assessment
Methods}\label{connectivity-status-assessment-methods}
\addcontentsline{toc}{section}{Connectivity Status Assessment Methods}

\markright{Connectivity Status Assessment Methods}

The connectivity status assessment for anadromous salmonids in the
Margaree watershed builds on existing connectivity modelling work
undertaken by the Canadian Wildlife Federation in partnership with the
Atlantic Salmon Federation. The Margaree model within the
NovaScotiaFishPass spatially locates known and modelled barriers to fish
passage, identifies potential spawning and rearing habitat for focal
species, and estimates the amount of habitat that is currently
accessible to focal species within the Margaree. The model uses an
adapted version of the Intrinsic Potential (IP) fish habitat modelling
framework (see Sheer et al.~(2009) for an overview of the IP framework).
The habitat model uses two geomorphic characteristics of the stream
network --- channel gradient and mean annual discharge --- to identify
potential spawning habitat and rearing habitat for each focal species.
The habitat model does not attempt to definitively map each habitat type
nor estimate habitat quality, but rather identifies stream segments that
have high potential to support spawning or rearing habitat for each
species based on the geomorphic characteristics of the segment. For more
details on the connectivity and habitat model structure and parameters,
please see Mazany-Wright, N and Norris, S. M. and Lapointe, N. W. R. and
Rebellato, B. (2021a). The variables and thresholds used to model
potential spawning and rearing habitat for each focal species are
summarized in Table 1. The quantity of modelled habitat for each species
was aggregated for each habitat type and represents a linear measure of
potential habitat.

\global\setlength{\Oldarrayrulewidth}{\arrayrulewidth}

\global\setlength{\Oldtabcolsep}{\tabcolsep}

\setlength{\tabcolsep}{0pt}

\renewcommand*{\arraystretch}{1.5}



\providecommand{\ascline}[3]{\noalign{\global\arrayrulewidth #1}\arrayrulecolor[HTML]{#2}\cline{#3}}

\begin{longtable}[c]{|p{1.43in}|p{3.34in}|p{8.40in}|p{3.09in}|p{3.03in}|p{1.94in}|p{1.33in}}

\caption{\label{tbl-param}Additional Key Actors. Key Actors are the
individuals, groups, and/or organizations, outside of the planning team,
with influence and relevant experience in the watershed, whose
engagement will be critical for the successful implementation of this
WCRP.}

\tabularnewline

\hhline{>{\arrayrulecolor[HTML]{666666}\global\arrayrulewidth=1.5pt}->{\arrayrulecolor[HTML]{666666}\global\arrayrulewidth=1.5pt}->{\arrayrulecolor[HTML]{666666}\global\arrayrulewidth=1.5pt}->{\arrayrulecolor[HTML]{666666}\global\arrayrulewidth=1.5pt}->{\arrayrulecolor[HTML]{666666}\global\arrayrulewidth=1.5pt}->{\arrayrulecolor[HTML]{666666}\global\arrayrulewidth=1.5pt}->{\arrayrulecolor[HTML]{666666}\global\arrayrulewidth=1.5pt}-}

\multicolumn{1}{>{\cellcolor[HTML]{008270}\raggedright}m{\dimexpr 1.43in+0\tabcolsep}}{\textcolor[HTML]{FFFFFF}{\fontsize{11}{11}\selectfont{Species}}} & \multicolumn{1}{>{\cellcolor[HTML]{008270}\raggedright}m{\dimexpr 3.34in+0\tabcolsep}}{\textcolor[HTML]{FFFFFF}{\fontsize{11}{11}\selectfont{Channel\ Gradient\ (\%)}}} & \multicolumn{1}{>{\cellcolor[HTML]{008270}\raggedright}m{\dimexpr 8.4in+0\tabcolsep}}{\textcolor[HTML]{FFFFFF}{\fontsize{11}{11}\selectfont{Mean\ annual\ discharge\ (m3/s)}}} & \multicolumn{1}{>{\cellcolor[HTML]{008270}\raggedright}m{\dimexpr 3.09in+0\tabcolsep}}{\textcolor[HTML]{FFFFFF}{\fontsize{11}{11}\selectfont{Channel\ gradient\ (\%)}}} & \multicolumn{1}{>{\cellcolor[HTML]{008270}\raggedright}m{\dimexpr 3.03in+0\tabcolsep}}{\textcolor[HTML]{FFFFFF}{\fontsize{11}{11}\selectfont{Mean\ Annual\ discharge\ (m3/s)}}} & \multicolumn{1}{>{\cellcolor[HTML]{008270}\raggedright}m{\dimexpr 1.94in+0\tabcolsep}}{\textcolor[HTML]{FFFFFF}{\fontsize{11}{11}\selectfont{Minimum\ Lake\ area\ (ha)}}} & \multicolumn{1}{>{\cellcolor[HTML]{008270}\raggedright}m{\dimexpr 1.33in+0\tabcolsep}}{\textcolor[HTML]{FFFFFF}{\fontsize{11}{11}\selectfont{Multiplier\ (1.5x)}}} \\

\noalign{\global\arrayrulewidth 0pt}\arrayrulecolor[HTML]{000000}

\hhline{>{\arrayrulecolor[HTML]{666666}\global\arrayrulewidth=1.5pt}->{\arrayrulecolor[HTML]{666666}\global\arrayrulewidth=1.5pt}->{\arrayrulecolor[HTML]{666666}\global\arrayrulewidth=1.5pt}->{\arrayrulecolor[HTML]{666666}\global\arrayrulewidth=1.5pt}->{\arrayrulecolor[HTML]{666666}\global\arrayrulewidth=1.5pt}->{\arrayrulecolor[HTML]{666666}\global\arrayrulewidth=1.5pt}->{\arrayrulecolor[HTML]{666666}\global\arrayrulewidth=1.5pt}-}\endfirsthead 

\hhline{>{\arrayrulecolor[HTML]{666666}\global\arrayrulewidth=1.5pt}->{\arrayrulecolor[HTML]{666666}\global\arrayrulewidth=1.5pt}->{\arrayrulecolor[HTML]{666666}\global\arrayrulewidth=1.5pt}->{\arrayrulecolor[HTML]{666666}\global\arrayrulewidth=1.5pt}->{\arrayrulecolor[HTML]{666666}\global\arrayrulewidth=1.5pt}->{\arrayrulecolor[HTML]{666666}\global\arrayrulewidth=1.5pt}->{\arrayrulecolor[HTML]{666666}\global\arrayrulewidth=1.5pt}-}

\multicolumn{1}{>{\cellcolor[HTML]{008270}\raggedright}m{\dimexpr 1.43in+0\tabcolsep}}{\textcolor[HTML]{FFFFFF}{\fontsize{11}{11}\selectfont{Species}}} & \multicolumn{1}{>{\cellcolor[HTML]{008270}\raggedright}m{\dimexpr 3.34in+0\tabcolsep}}{\textcolor[HTML]{FFFFFF}{\fontsize{11}{11}\selectfont{Channel\ Gradient\ (\%)}}} & \multicolumn{1}{>{\cellcolor[HTML]{008270}\raggedright}m{\dimexpr 8.4in+0\tabcolsep}}{\textcolor[HTML]{FFFFFF}{\fontsize{11}{11}\selectfont{Mean\ annual\ discharge\ (m3/s)}}} & \multicolumn{1}{>{\cellcolor[HTML]{008270}\raggedright}m{\dimexpr 3.09in+0\tabcolsep}}{\textcolor[HTML]{FFFFFF}{\fontsize{11}{11}\selectfont{Channel\ gradient\ (\%)}}} & \multicolumn{1}{>{\cellcolor[HTML]{008270}\raggedright}m{\dimexpr 3.03in+0\tabcolsep}}{\textcolor[HTML]{FFFFFF}{\fontsize{11}{11}\selectfont{Mean\ Annual\ discharge\ (m3/s)}}} & \multicolumn{1}{>{\cellcolor[HTML]{008270}\raggedright}m{\dimexpr 1.94in+0\tabcolsep}}{\textcolor[HTML]{FFFFFF}{\fontsize{11}{11}\selectfont{Minimum\ Lake\ area\ (ha)}}} & \multicolumn{1}{>{\cellcolor[HTML]{008270}\raggedright}m{\dimexpr 1.33in+0\tabcolsep}}{\textcolor[HTML]{FFFFFF}{\fontsize{11}{11}\selectfont{Multiplier\ (1.5x)}}} \\

\noalign{\global\arrayrulewidth 0pt}\arrayrulecolor[HTML]{000000}

\hhline{>{\arrayrulecolor[HTML]{666666}\global\arrayrulewidth=1.5pt}->{\arrayrulecolor[HTML]{666666}\global\arrayrulewidth=1.5pt}->{\arrayrulecolor[HTML]{666666}\global\arrayrulewidth=1.5pt}->{\arrayrulecolor[HTML]{666666}\global\arrayrulewidth=1.5pt}->{\arrayrulecolor[HTML]{666666}\global\arrayrulewidth=1.5pt}->{\arrayrulecolor[HTML]{666666}\global\arrayrulewidth=1.5pt}->{\arrayrulecolor[HTML]{666666}\global\arrayrulewidth=1.5pt}-}\endhead



\multicolumn{1}{>{\raggedright}m{\dimexpr 1.43in+0\tabcolsep}}{\textcolor[HTML]{000000}{\fontsize{11}{11}\selectfont{Chinook\ Salmon}}} & \multicolumn{1}{>{\raggedright}m{\dimexpr 3.34in+0\tabcolsep}}{\textcolor[HTML]{000000}{\fontsize{11}{11}\selectfont{0-3}}} & \multicolumn{1}{>{\raggedright}m{\dimexpr 8.4in+0\tabcolsep}}{\textcolor[HTML]{000000}{\fontsize{11}{11}\selectfont{0.46-322.5}}} & \multicolumn{1}{>{\raggedright}m{\dimexpr 3.09in+0\tabcolsep}}{\textcolor[HTML]{000000}{\fontsize{11}{11}\selectfont{0-5}}} & \multicolumn{1}{>{\raggedright}m{\dimexpr 3.03in+0\tabcolsep}}{\textcolor[HTML]{000000}{\fontsize{11}{11}\selectfont{0.28-100}}} & \multicolumn{1}{>{\raggedright}m{\dimexpr 1.94in+0\tabcolsep}}{\textcolor[HTML]{000000}{\fontsize{11}{11}\selectfont{}}} & \multicolumn{1}{>{\raggedright}m{\dimexpr 1.33in+0\tabcolsep}}{\textcolor[HTML]{000000}{\fontsize{11}{11}\selectfont{}}} \\

\noalign{\global\arrayrulewidth 0pt}\arrayrulecolor[HTML]{000000}





\multicolumn{1}{>{\raggedright}m{\dimexpr 1.43in+0\tabcolsep}}{\textcolor[HTML]{000000}{\fontsize{11}{11}\selectfont{}}} & \multicolumn{1}{>{\raggedright}m{\dimexpr 3.34in+0\tabcolsep}}{\textcolor[HTML]{000000}{\fontsize{11}{11}\selectfont{(Busch\ et\ al.\ 2011,\ Cooney\ and\ Holzer\ 2006)}}} & \multicolumn{1}{>{\raggedright}m{\dimexpr 8.4in+0\tabcolsep}}{\textcolor[HTML]{000000}{\fontsize{11}{11}\selectfont{(Bjornn\ and\ Reiser\ 1991,\ Neuman\ and\ Newcombe\ 1977,\ Woll\ et\ al.\ 2017,\ Roberge\ et\ al.\ 2002,\ Raleigh\ and\ Miller\ 1986)}}} & \multicolumn{1}{>{\raggedright}m{\dimexpr 3.09in+0\tabcolsep}}{\textcolor[HTML]{000000}{\fontsize{11}{11}\selectfont{(Woll\ et\ al.\ 2017,\ Porter\ et\ al.\ 2008)}}} & \multicolumn{1}{>{\raggedright}m{\dimexpr 3.03in+0\tabcolsep}}{\textcolor[HTML]{000000}{\fontsize{11}{11}\selectfont{(Agrawal\ et\ al.\ 2005)}}} & \multicolumn{1}{>{\raggedright}m{\dimexpr 1.94in+0\tabcolsep}}{\textcolor[HTML]{000000}{\fontsize{11}{11}\selectfont{}}} & \multicolumn{1}{>{\raggedright}m{\dimexpr 1.33in+0\tabcolsep}}{\textcolor[HTML]{000000}{\fontsize{11}{11}\selectfont{}}} \\

\noalign{\global\arrayrulewidth 0pt}\arrayrulecolor[HTML]{000000}





\multicolumn{1}{>{\raggedright}m{\dimexpr 1.43in+0\tabcolsep}}{\textcolor[HTML]{000000}{\fontsize{11}{11}\selectfont{Coho\ Salmon}}} & \multicolumn{1}{>{\raggedright}m{\dimexpr 3.34in+0\tabcolsep}}{\textcolor[HTML]{000000}{\fontsize{11}{11}\selectfont{0-5}}} & \multicolumn{1}{>{\raggedright}m{\dimexpr 8.4in+0\tabcolsep}}{\textcolor[HTML]{000000}{\fontsize{11}{11}\selectfont{0.164-59.15}}} & \multicolumn{1}{>{\raggedright}m{\dimexpr 3.09in+0\tabcolsep}}{\textcolor[HTML]{000000}{\fontsize{11}{11}\selectfont{0-5}}} & \multicolumn{1}{>{\raggedright}m{\dimexpr 3.03in+0\tabcolsep}}{\textcolor[HTML]{000000}{\fontsize{11}{11}\selectfont{0.03-40}}} & \multicolumn{1}{>{\raggedright}m{\dimexpr 1.94in+0\tabcolsep}}{\textcolor[HTML]{000000}{\fontsize{11}{11}\selectfont{}}} & \multicolumn{1}{>{\raggedright}m{\dimexpr 1.33in+0\tabcolsep}}{\textcolor[HTML]{000000}{\fontsize{11}{11}\selectfont{Wetland}}} \\

\noalign{\global\arrayrulewidth 0pt}\arrayrulecolor[HTML]{000000}





\multicolumn{1}{>{\raggedright}m{\dimexpr 1.43in+0\tabcolsep}}{\textcolor[HTML]{000000}{\fontsize{11}{11}\selectfont{}}} & \multicolumn{1}{>{\raggedright}m{\dimexpr 3.34in+0\tabcolsep}}{\textcolor[HTML]{000000}{\fontsize{11}{11}\selectfont{(Roberge\ et\ al.\ 2002,\ Sloat\ et\ al.\ 2017)}}} & \multicolumn{1}{>{\raggedright}m{\dimexpr 8.4in+0\tabcolsep}}{\textcolor[HTML]{000000}{\fontsize{11}{11}\selectfont{(Bjornn\ and\ Reiser\ 1991,\ Sloat\ et\ al.\ 2017,\ Neuman\ and\ Newcombe\ 1977,\ Woll\ et\ al.\ 2017,\ McMahon\ 1983)}}} & \multicolumn{1}{>{\raggedright}m{\dimexpr 3.09in+0\tabcolsep}}{\textcolor[HTML]{000000}{\fontsize{11}{11}\selectfont{(Porter\ et\ al.\ 2008,\ Rosenfeld\ et\ al.\ 2000)}}} & \multicolumn{1}{>{\raggedright}m{\dimexpr 3.03in+0\tabcolsep}}{\textcolor[HTML]{000000}{\fontsize{11}{11}\selectfont{(Agrawal\ et\ al.\ 2005,\ Burnett\ et\ al.\ 2007)}}} & \multicolumn{1}{>{\raggedright}m{\dimexpr 1.94in+0\tabcolsep}}{\textcolor[HTML]{000000}{\fontsize{11}{11}\selectfont{}}} & \multicolumn{1}{>{\raggedright}m{\dimexpr 1.33in+0\tabcolsep}}{\textcolor[HTML]{000000}{\fontsize{11}{11}\selectfont{}}} \\

\noalign{\global\arrayrulewidth 0pt}\arrayrulecolor[HTML]{000000}





\multicolumn{1}{>{\raggedright}m{\dimexpr 1.43in+0\tabcolsep}}{\textcolor[HTML]{000000}{\fontsize{11}{11}\selectfont{Sockeye\ Salmon}}} & \multicolumn{1}{>{\raggedright}m{\dimexpr 3.34in+0\tabcolsep}}{\textcolor[HTML]{000000}{\fontsize{11}{11}\selectfont{0-2}}} & \multicolumn{1}{>{\raggedright}m{\dimexpr 8.4in+0\tabcolsep}}{\textcolor[HTML]{000000}{\fontsize{11}{11}\selectfont{0.175-65}}} & \multicolumn{1}{>{\raggedright}m{\dimexpr 3.09in+0\tabcolsep}}{\textcolor[HTML]{000000}{\fontsize{11}{11}\selectfont{}}} & \multicolumn{1}{>{\raggedright}m{\dimexpr 3.03in+0\tabcolsep}}{\textcolor[HTML]{000000}{\fontsize{11}{11}\selectfont{}}} & \multicolumn{1}{>{\raggedright}m{\dimexpr 1.94in+0\tabcolsep}}{\textcolor[HTML]{000000}{\fontsize{11}{11}\selectfont{200}}} & \multicolumn{1}{>{\raggedright}m{\dimexpr 1.33in+0\tabcolsep}}{\textcolor[HTML]{000000}{\fontsize{11}{11}\selectfont{Lake}}} \\

\noalign{\global\arrayrulewidth 0pt}\arrayrulecolor[HTML]{000000}





\multicolumn{1}{>{\raggedright}m{\dimexpr 1.43in+0\tabcolsep}}{\textcolor[HTML]{000000}{\fontsize{11}{11}\selectfont{}}} & \multicolumn{1}{>{\raggedright}m{\dimexpr 3.34in+0\tabcolsep}}{\textcolor[HTML]{000000}{\fontsize{11}{11}\selectfont{(Lake\ 1999,\ Hoopes\ 1972)}}} & \multicolumn{1}{>{\raggedright}m{\dimexpr 8.4in+0\tabcolsep}}{\textcolor[HTML]{000000}{\fontsize{11}{11}\selectfont{(Bjornn\ and\ Reiser\ 1991,\ Woll\ et\ al.\ 2017,\ Neuman\ and\ Newcombe\ 1977,\ Roberge\ et\ al.\ 2002)}}} & \multicolumn{1}{>{\raggedright}m{\dimexpr 3.09in+0\tabcolsep}}{\textcolor[HTML]{000000}{\fontsize{11}{11}\selectfont{}}} & \multicolumn{1}{>{\raggedright}m{\dimexpr 3.03in+0\tabcolsep}}{\textcolor[HTML]{000000}{\fontsize{11}{11}\selectfont{}}} & \multicolumn{1}{>{\raggedright}m{\dimexpr 1.94in+0\tabcolsep}}{\textcolor[HTML]{000000}{\fontsize{11}{11}\selectfont{(Woll\ et\ al.\ 2017)}}} & \multicolumn{1}{>{\raggedright}m{\dimexpr 1.33in+0\tabcolsep}}{\textcolor[HTML]{000000}{\fontsize{11}{11}\selectfont{}}} \\

\noalign{\global\arrayrulewidth 0pt}\arrayrulecolor[HTML]{000000}

\hhline{>{\arrayrulecolor[HTML]{666666}\global\arrayrulewidth=1.5pt}->{\arrayrulecolor[HTML]{666666}\global\arrayrulewidth=1.5pt}->{\arrayrulecolor[HTML]{666666}\global\arrayrulewidth=1.5pt}->{\arrayrulecolor[HTML]{666666}\global\arrayrulewidth=1.5pt}->{\arrayrulecolor[HTML]{666666}\global\arrayrulewidth=1.5pt}->{\arrayrulecolor[HTML]{666666}\global\arrayrulewidth=1.5pt}->{\arrayrulecolor[HTML]{666666}\global\arrayrulewidth=1.5pt}-}


\end{longtable}

\arrayrulecolor[HTML]{000000}

\global\setlength{\arrayrulewidth}{\Oldarrayrulewidth}

\global\setlength{\tabcolsep}{\Oldtabcolsep}

\renewcommand*{\arraystretch}{1}

\chapter*{Supplementary Information}\label{supplementary-information}
\addcontentsline{toc}{chapter}{Supplementary Information}

\markboth{Supplementary Information}{Supplementary Information}

\section*{Situation Analysis}\label{situation-analysis}
\addcontentsline{toc}{section}{Situation Analysis}

\markright{Situation Analysis}

The following situation model was developed by the WCRP planning team to
``map'' the project context and brainstorm potential actions for
implementation. Green text is used to identify actions that were
selected for implementation (see Strategies \& Actions), and red text is
used to identify actions that the project team has decided to exclude
from the current iteration of the plan, as they were either outside of
the project scope, or were deemed to be ineffective by the planning
team.

\begin{figure}

\centering{

\pandocbounded{\includegraphics[keepaspectratio]{content/images/situation-analysis.png}}

}

\caption{\label{fig-sitan}SAMPLE Situation analysis developed by the
planning team to identify factors that contribute to fragmentation
(orange boxes), biophysical results (brown boxes), and potential
strategies/actions to improve connectivity (yellow hexagons) for target
species in the Horsefly River watershed.}

\end{figure}%

\section*{Strategies \& Actions}\label{strategies-actions}
\addcontentsline{toc}{section}{Strategies \& Actions}

\markright{Strategies \& Actions}

Effectiveness evaluation of identified conservation strategies and
associated actions to improve connectivity for target species in . The
planning team identified five broad strategies to implement through this
WCRP, 1) crossing remediation, 2) lateral barrier remediation, 3) dam
remediation, 4) barrier prevention, and 5) communication and education.
Individual actions were qualitatively evaluated based on the anticipated
effect each action will have on realizing on-the-ground gains in
connectivity. Effectiveness ratings are based on a combination of
``Feasibility and''Impact'', Feasibility is defined as the degree to
which the project team can implement the action within realistic
constraints (financial, time, ethical, etc.) and Impact is the degree to
which the action is likely to contribute to achieving one or more of the
goals established in this plan.

\section*{Strategy 1: Crossing
Remediation}\label{strategy-1-crossing-remediation}
\addcontentsline{toc}{section}{Strategy 1: Crossing Remediation}

\markright{Strategy 1: Crossing Remediation}

\begin{longtable}[]{@{}llllll@{}}

\caption{\label{tbl-S1}Strategy 1}

\tabularnewline

\caption{}\label{T_bb669}\tabularnewline
\toprule\noalign{}
ID & Actions & Details & Feasibility & Impact & Effectiveness \\
\midrule\noalign{}
\endfirsthead
\toprule\noalign{}
ID & Actions & Details & Feasibility & Impact & Effectiveness \\
\midrule\noalign{}
\endhead
\bottomrule\noalign{}
\endlastfoot
1.1 & Remediate crossings that are acting as barriers & This action
represents some projects that would be led by the planning team with
conservation funds (e.g., orphaned barriers or those owned by
individuals), while other remediation projects would be the
responsibility of the barrier owner. Industry will have to be engaged to
successfully implement this intervention. PSC Southern Boundary
Restoration and Enhancement Fund proposal: - Complete remediation of one
priority barrier, including engineering designs HCTF proposal: -
Complete remediation of one priority barrier CNFASAR proposal (2022-26):
- Complete remediation of one priority barrier per year for four years
HRR Can help with finding local people to implement remediation
projects. & High & Very high & Effective \\
1.2 & Lobby that the government enforce their regulations & This can
apply to both provincial and federal governments. For example,
advocating for increased discretionary decisions to remove barriers to
fish. One action could be to submit barrier assessment data to show
proof that regulations are not being followed. & Very high & High &
Effective \\
1.3 & Initiate a barrier owner outreach program for locations on the
barrier remediation shortlist & Work with landowners / users (e.g., ATV
groups) to identify and remediate their aquatic barriers. Education
component can help prevent barriers in the first place. HRR to reach out
to owners of confirmed barriers to discuss remediation options; CWF to
reach out to provincial representatives. & Very high & Very high & Very
effective \\
1.4 & Knowledge Gap: Continue updating the barrier prioritization model
& The model has been updated to reflect 2021 field assessments and
intermediate barrier review. & Very high & High & Effective \\
1.5 & Knowledge Gap: conduct field assessments on updated preliminary
barrier list using the provincial fish passage framework and update
connectivity goal if additional barriers are added to the barrier
remediation shortlist & Twenty-six field assessments performed in 2021.
& Very high & Very high & Very effective \\
1.6 & Update longitudinal connectivity goal if additional barriers are
added to the barrier remediation shortlist & & & & \\
1.7 & Knowledge Gap: Identify and map crossing ownership & For barriers
on the barrier remediation shortlist. & Very high & Very high & Very
effective \\
1.8 & Knowledge Gap: Compile road maintenance schedules &
Ground-truthing is important, as the schedules do not always reflect
what happens in the field. & High & High & Effective \\
1.9 & Knowledge Gap: Survey trail-stream crossings to confirm low
pressure rating values & Need to access detailed trail maps in the
watershed to prioritize our time and resources. This should be
accomplished as people are out surveying for other reasons rather than
spending time and resources specifically to fill this knowledge gap.
CNFASAR proposal: Collaborate with WLFN to: - Develop field assessment
protocols for whether ATV trail stream crossings pass fish, and for
assessing other effects on fish habitat - Map potential trail-stream
crossings on salmon habitat that could be assessed - Assess 30-50 trail
stream crossings, record measurements, and take pictures & Very high &
Medium & Need more information \\
& & & & & \\

\end{longtable}

\section*{Strategy 2: Lateral Barrier
Remediation}\label{strategy-2-lateral-barrier-remediation}
\addcontentsline{toc}{section}{Strategy 2: Lateral Barrier Remediation}

\markright{Strategy 2: Lateral Barrier Remediation}

\begin{longtable}[]{@{}llllll@{}}

\caption{\label{tbl-S2}Strategy 2}

\tabularnewline

\caption{}\label{T_08fee}\tabularnewline
\toprule\noalign{}
ID & Actions & Details & Feasibility & Impact & Effectiveness \\
\midrule\noalign{}
\endfirsthead
\toprule\noalign{}
ID & Actions & Details & Feasibility & Impact & Effectiveness \\
\midrule\noalign{}
\endhead
\bottomrule\noalign{}
\endlastfoot
2.1 & Remediate dikes / berms / other lateral barriers & & High & Very
high & Effective \\
2.2 & Initiate a barrier owner outreach program & & Very high & Very
high & Very effective \\
2.3 & Knowledge Gap: Identify and map year-round lateral habitat, as
well as overwintering habitat & Explore the use of a drone to identify
lateral habitat. - Volunteers from the HRR will conduct field habitat
assessments following modules in the Pacific Streamkeepers Handbook to
assess disconnected lateral and overwintering salmon habitats in the
Horsefly watershed CNFASAR proposal: -Funding for equipment in
2022-2023, and for field transportation in 2022-2023, 2023-2024 & Very
high & Very high & Very effective \\
2.4 & Knowledge Gap: Map lateral barriers and barrier ownership & Focus
on identifying ownership of priority lateral barriers that we want to
remediate in the short-term. & Very high & Very high & Very effective \\
2.5 & Knowledge Gap: Develop a framework to assess and prioritize
between different lateral barrier remediation projects & CWF is leading
a provincial-scale analysis of the effect of rail lines on connectivity
for Anadromous Salmonids, as part of this project lateral habitat and
barrier assessments and prioritization methods will be developed. & Very
high & Very high & Very effective \\

\end{longtable}

\section*{Strategy 3: Dam Remediation}\label{strategy-3-dam-remediation}
\addcontentsline{toc}{section}{Strategy 3: Dam Remediation}

\markright{Strategy 3: Dam Remediation}

\begin{longtable}[]{@{}llllll@{}}

\caption{\label{tbl-S3}Strategy 3}

\tabularnewline

\caption{}\label{T_39ecf}\tabularnewline
\toprule\noalign{}
ID & Actions & Details & Feasibility & Impact & Effectiveness \\
\midrule\noalign{}
\endfirsthead
\toprule\noalign{}
ID & Actions & Details & Feasibility & Impact & Effectiveness \\
\midrule\noalign{}
\endhead
\bottomrule\noalign{}
\endlastfoot
3.1 & Remediate Dams & & Medium & Very high & Need more information \\
3.2 & Install Fish Passage & & Medium & High & Need more information \\
3.3 & Connect with Cattleman\textquotesingle s Association to explore a
partnership to remediate dams & This may involve exploring alternative
water management actions that would allow for the remediation of
irrigation dams. & High & Medium & Need more information \\
3.4 & Knowledge Gap: Continue updating the barrier prioritization model
& The model has been updated to reflect 2021 field assessments and
intermediate barrier review. & Very high & High & Effective \\
3.5 & Knowledge Gap: Assess dams to determine whether they exist and are
truly blocking fish habitat & Four dams were assessed during 2021 field
season; additional field assessment needed. & Very high & High &
Effective \\
3.6 & Knowledge Gap: Identify and map dam ownership & & Very high & Very
high & Very effective \\

\end{longtable}

\section*{Strategy 4: Barrier
Prevention}\label{strategy-4-barrier-prevention}
\addcontentsline{toc}{section}{Strategy 4: Barrier Prevention}

\markright{Strategy 4: Barrier Prevention}

\begin{longtable}[]{@{}llllll@{}}

\caption{\label{tbl-S4}Strategy 4}

\tabularnewline

\caption{}\label{T_c8d54}\tabularnewline
\toprule\noalign{}
ID & Actions & Details & Feasibility & Impact & Effectiveness \\
\midrule\noalign{}
\endfirsthead
\toprule\noalign{}
ID & Actions & Details & Feasibility & Impact & Effectiveness \\
\midrule\noalign{}
\endhead
\bottomrule\noalign{}
\endlastfoot
4.1 & Explore potential partnerships with industrial companies & Invite
industrial players to a workshop on how to apply crossing / lateral
barrier BMPs. BMPs could include those that minimize the need for
road-stream crossings. & Very high & High & Effective \\
4.2 & Stabilize sediment sources that are explicitly linked to sediment
wedges or erosion that are acting as barriers & This could include
numerous bank stabilization techniques, including restoring riparian
vegetation. This applies to some tributaries that have altered
confluence areas - the link needs to be made between confluence
alterations and timing of movement for juvenile fish. Local ranchers and
Cattleman\textquotesingle s association could be engaged, as well as
forestry licensees. & Very high & Medium & Need more information \\

\end{longtable}

\section*{Strategy 5: Communication and
Education}\label{strategy-5-communication-and-education}
\addcontentsline{toc}{section}{Strategy 5: Communication and Education}

\markright{Strategy 5: Communication and Education}

\begin{longtable}[]{@{}lll@{}}

\caption{\label{tbl-S5}Strategy 5}

\tabularnewline

\caption{}\label{T_714d9}\tabularnewline
\toprule\noalign{}
ID & Actions & Details \\
\midrule\noalign{}
\endfirsthead
\toprule\noalign{}
ID & Actions & Details \\
\midrule\noalign{}
\endhead
\bottomrule\noalign{}
\endlastfoot
5.1 & Implement the WCRP Progress Tracking Plan & The WCRP Progress
Tracking Plan will help the team determine if we are achieving our goals
and objectives. \\
5.2 & Develop a communication strategy to raise awareness and support
for this WCRP & This intervention includes communicating both the WCRP
and the collaborative process in developing it, as well as communicating
outcomes (e.g., barrier remediations). CNFASAR proposal: - HRR will work
with CWF to develop outreach and communications materials, including
press releases, social media content, a video, and content for their
website - With HRR, CWF will present on fish passage issues and
solutions at the annual Horsefly River Salmon Festival \\

\end{longtable}

\section*{Theories of Change \&
Objectives}\label{theories-of-change-objectives}
\addcontentsline{toc}{section}{Theories of Change \& Objectives}

\markright{Theories of Change \& Objectives}

Theories of Change are explicit assumptions around how the identified
actions will achieve gains in connectivity and contribute towards
reaching the goals of the plan. To develop Theories of Change, the
planning team developed explicit assumptions for each strategy which
helped to clarify the rationale used for undertaking actions and
provided an opportunity for feedback on invalid assumptions or missing
opportunities. The Theories of Change are results oriented and clearly
define the expected outcome. The following theory of change models were
developed by the WCRP planning team to ``map'' the causal (``if-then'')
progression of assumptions of how the actions within a strategy work
together to achieve project goals.

\begin{figure}

\centering{

\pandocbounded{\includegraphics[keepaspectratio]{content/images/flowchart-crossing-rem.png}}

}

\caption{\label{fig-stra1}Theory of change developed by the planning
team for the actions identified under Strategy 1: Crossing Remediation
in the .}

\end{figure}%

\begin{figure}

\centering{

\pandocbounded{\includegraphics[keepaspectratio]{content/images/flowchart-lat-bar-rem.png}}

}

\caption{\label{fig-stra2}Theory of change developed by the planning
team for the actions identified under Strategy 2: Lateral Barrier
Remediation in .}

\end{figure}%

\begin{figure}

\centering{

\pandocbounded{\includegraphics[keepaspectratio]{content/images/flowchart-dam-rem.png}}

}

\caption{\label{fig-stra3}Theory of change developed by the planning
team for the actions identified under Strategy 3: Dam Remediation in .}

\end{figure}%

\begin{figure}

\centering{

\pandocbounded{\includegraphics[keepaspectratio]{content/images/flowchart-bar-prevent.png}}

}

\caption{\label{fig-stra4}Theory of change developed by the planning
team for the actions identified under Strategy 4: Barrier Prevention in
.}

\end{figure}%

\section*{Operational Plan}\label{operational-plan-1}
\addcontentsline{toc}{section}{Operational Plan}

\markright{Operational Plan}

The operational plan represents a preliminary exercise undertaken by the
planning team to identify the potential leads, potential participants,
and estimated cost for the implementation of each action in . The table
below summarizes individuals, groups, or organizations that the planning
team felt could lead or participate in the implementation of the plan
and should be interpreted as the first step in on-going planning and
engagement to develop more detailed and sophisticated action plans for
each entry in the table. The individuals, groups, and organizations
listed under the ``Lead(s)'' or ``Potential Participants'' columns are
those that provisionally expressed interest in participating in one of
those roles or were suggested by the planning team for further
engagement (denoted in bold), for those that are not members of the
planning team. The leads, participants, and estimated costs in the
operational plan are not binding nor an official commitment of
resources, but rather provide a roadmap for future coordination and
engagement to work towards implementation of the WCRP.

\begin{longtable}[]{@{}lll@{}}

\caption{\label{tbl-opplan}Operational plan to support the
implementation of strategies and actions to improve connectivity for
target species in .}

\tabularnewline

\caption{}\label{T_50238}\tabularnewline
\toprule\noalign{}
Strategy / Actions & Lead(s) & Participants \\
\midrule\noalign{}
\endfirsthead
\toprule\noalign{}
Strategy / Actions & Lead(s) & Participants \\
\midrule\noalign{}
\endhead
\bottomrule\noalign{}
\endlastfoot
1.1 -- Conduct habitat assessment above and below confirmed actionable
barriers & MSA & MSA, CWF to support \\
1.2 -- Apply for permitting for actionable barriers with disconnected
confirmed habitat & MSA & NSSA \\
1.3-- Rehabilitate Stream-Road Crossings & MSA & MSA, NSSA, CWF to
support \\
2 -- Clear debris jams and identify opportunities to connect habitat
without structure restoration & MSA & \\
3-- Knowledge gap: Look into thermal refugia and lateral barriers within
the Margaree & MSA & CWF, ASF \\

\end{longtable}

\section*{Funding Sources}\label{funding-sources}
\addcontentsline{toc}{section}{Funding Sources}

\markright{Funding Sources}

\global\setlength{\Oldarrayrulewidth}{\arrayrulewidth}

\global\setlength{\Oldtabcolsep}{\tabcolsep}

\setlength{\tabcolsep}{0pt}

\renewcommand*{\arraystretch}{1.5}



\providecommand{\ascline}[3]{\noalign{\global\arrayrulewidth #1}\arrayrulecolor[HTML]{#2}\cline{#3}}

\begin{longtable}[c]{|p{5.52in}|p{60.28in}}

\caption{\label{tbl-fund}Potential funding sources for plan
implementation in . The Canadian Wildlife Federation and the planning
team can coordinate proposal submission through these sources.}

\tabularnewline

\hhline{>{\arrayrulecolor[HTML]{666666}\global\arrayrulewidth=1.5pt}->{\arrayrulecolor[HTML]{666666}\global\arrayrulewidth=1.5pt}-}

\multicolumn{1}{>{\cellcolor[HTML]{008270}\raggedright}m{\dimexpr 5.52in+0\tabcolsep}}{\textcolor[HTML]{FFFFFF}{\fontsize{11}{11}\selectfont{Funding\ Source}}} & \multicolumn{1}{>{\cellcolor[HTML]{008270}\raggedright}m{\dimexpr 60.28in+0\tabcolsep}}{\textcolor[HTML]{FFFFFF}{\fontsize{11}{11}\selectfont{Spending\ Restrictions\ and\ Other\ Consideration}}} \\

\noalign{\global\arrayrulewidth 0pt}\arrayrulecolor[HTML]{000000}

\hhline{>{\arrayrulecolor[HTML]{666666}\global\arrayrulewidth=1.5pt}->{\arrayrulecolor[HTML]{666666}\global\arrayrulewidth=1.5pt}-}\endfirsthead 

\hhline{>{\arrayrulecolor[HTML]{666666}\global\arrayrulewidth=1.5pt}->{\arrayrulecolor[HTML]{666666}\global\arrayrulewidth=1.5pt}-}

\multicolumn{1}{>{\cellcolor[HTML]{008270}\raggedright}m{\dimexpr 5.52in+0\tabcolsep}}{\textcolor[HTML]{FFFFFF}{\fontsize{11}{11}\selectfont{Funding\ Source}}} & \multicolumn{1}{>{\cellcolor[HTML]{008270}\raggedright}m{\dimexpr 60.28in+0\tabcolsep}}{\textcolor[HTML]{FFFFFF}{\fontsize{11}{11}\selectfont{Spending\ Restrictions\ and\ Other\ Consideration}}} \\

\noalign{\global\arrayrulewidth 0pt}\arrayrulecolor[HTML]{000000}

\hhline{>{\arrayrulecolor[HTML]{666666}\global\arrayrulewidth=1.5pt}->{\arrayrulecolor[HTML]{666666}\global\arrayrulewidth=1.5pt}-}\endhead



\multicolumn{1}{>{\raggedright}m{\dimexpr 5.52in+0\tabcolsep}}{\textcolor[HTML]{000000}{\fontsize{11}{11}\selectfont{Land\ Based\ Investment\ Strategy}}} & \multicolumn{1}{>{\raggedright}m{\dimexpr 60.28in+0\tabcolsep}}{\textcolor[HTML]{000000}{\fontsize{11}{11}\selectfont{Assessment\ and\ remediation\ of\ fish\ passage\ using\ provincial\ strategic\ approach.\ Primarily\ for\ remediation\ of\ Ministry-owned/orphaned\ barriers\ on\ forest\ service\ roads.}}} \\

\noalign{\global\arrayrulewidth 0pt}\arrayrulecolor[HTML]{000000}





\multicolumn{1}{>{\raggedright}m{\dimexpr 5.52in+0\tabcolsep}}{\textcolor[HTML]{000000}{\fontsize{11}{11}\selectfont{Environmental\ Enhancement\ Fund}}} & \multicolumn{1}{>{\raggedright}m{\dimexpr 60.28in+0\tabcolsep}}{\textcolor[HTML]{000000}{\fontsize{11}{11}\selectfont{Fish\ and\ wildlife\ passage\ improvements\ and\ restoration\ at\ stream\ and\ animal\ crossings\ at\ MOTI\ roads\ including\ culvert\ retrofits\ and\ replacement\ to\ restore\ Pacific\ salmon\ and\ trout\ access,\ and\ wildlife\ tunnels.\ Primarily\ for\ crossings\ linked\ to\ highway\ infrastructure.}}} \\

\noalign{\global\arrayrulewidth 0pt}\arrayrulecolor[HTML]{000000}





\multicolumn{1}{>{\raggedright}m{\dimexpr 5.52in+0\tabcolsep}}{\textcolor[HTML]{000000}{\fontsize{11}{11}\selectfont{Community\ Salmon\ Program}}} & \multicolumn{1}{>{\raggedright}m{\dimexpr 60.28in+0\tabcolsep}}{\textcolor[HTML]{000000}{\fontsize{11}{11}\selectfont{For\ projects\ supporting\ the\ protection,\ conservation\ and\ enhancement\ or\ rehabilitation\ of\ Pacific\ salmonids\ and\ their\ habitat.\ Funding\ for\ volunteer\ and\ not-for-profit\ community-based\ groups.\ Applicant\ must\ have\ a\ significant\ volunteer\ component\ to\ their\ group\ and\ to\ the\ project.\ Requires\ 50\%\ match\ for\ funding\ (volunteer,\ in-kind,\ donation\ or\ other\ grants).\ }}} \\

\noalign{\global\arrayrulewidth 0pt}\arrayrulecolor[HTML]{000000}





\multicolumn{1}{>{\raggedright}m{\dimexpr 5.52in+0\tabcolsep}}{\textcolor[HTML]{000000}{\fontsize{11}{11}\selectfont{Southern\ Boundary\ Restoration\ and\ Enhancement\ Fund}}} & \multicolumn{1}{>{\raggedright}m{\dimexpr 60.28in+0\tabcolsep}}{\textcolor[HTML]{000000}{\fontsize{11}{11}\selectfont{Supports\ 3\ activities:\ (1)\ develop\ improved\ information\ for\ resource\ management;\ (2)\ Rehabilitate\ and\ restore\ marine\ and\ freshwater\ habitat;\ and\ (3)\ enhance\ wild\ stock\ production\ through\ low\ technology\ techniques.\ Emphasis\ for\ funding\ is\ on\ stocks\ of\ conservation\ concern,\ particularly\ those\ contributing\ to\ a\ fishery\ and\ stocks\ of\ bilateral\ fishery\ relevance.}}} \\

\noalign{\global\arrayrulewidth 0pt}\arrayrulecolor[HTML]{000000}





\multicolumn{1}{>{\raggedright}m{\dimexpr 5.52in+0\tabcolsep}}{\textcolor[HTML]{000000}{\fontsize{11}{11}\selectfont{Habitat\ Conservation\ Trust\ Foundation\ Enhancement\ and\ Restoration\ Grants}}} & \multicolumn{1}{>{\raggedright}m{\dimexpr 60.28in+0\tabcolsep}}{\textcolor[HTML]{000000}{\fontsize{11}{11}\selectfont{Projects\ that\ focus\ on\ freshwater\ wild\ fish,\ native\ wildlife\ species\ and\ their\ habitats,\ have\ the\ potential\ to\ achieve\ a\ significant\ conservation\ outcome,\ while\ maintaining\ or\ enhancing\ opportunities\ for\ fishing,\ hunting,\ trapping,\ wildlife\ viewing\ and\ associated\ outdoor\ recreational\ activities.\ Primary\ focus\ is\ on\ provincially\ managed\ fisheries\ such\ as\ Steelhead\ and\ Westslope\ Cutthroat\ Trout.\ Requires\ 50\%\ funding\ match.}}} \\

\noalign{\global\arrayrulewidth 0pt}\arrayrulecolor[HTML]{000000}





\multicolumn{1}{>{\raggedright}m{\dimexpr 5.52in+0\tabcolsep}}{\textcolor[HTML]{000000}{\fontsize{11}{11}\selectfont{Environmental\ Damages\ Fund}}} & \multicolumn{1}{>{\raggedright}m{\dimexpr 60.28in+0\tabcolsep}}{\textcolor[HTML]{000000}{\fontsize{11}{11}\selectfont{Direct\ funds\ received\ from\ fines,\ court\ orders\ and\ voluntary\ payments\ to\ priority\ projects\ that\ will\ benefit\ Canada’s\ natural\ environment,\ under\ 4\ categories\ of\ improvement\ (in\ order\ of\ preference):\ (1)\ restoration,\ (2)\ environmental\ quality\ improvement,\ (3)\ research\ and\ development,\ and\ (4)\ education\ and\ awareness.}}} \\

\noalign{\global\arrayrulewidth 0pt}\arrayrulecolor[HTML]{000000}





\multicolumn{1}{>{\raggedright}m{\dimexpr 5.52in+0\tabcolsep}}{\textcolor[HTML]{000000}{\fontsize{11}{11}\selectfont{Habitat\ Stewardship\ Program\ for\ Aquatic\ Species\ at\ Risk}}} & \multicolumn{1}{>{\raggedright}m{\dimexpr 60.28in+0\tabcolsep}}{\textcolor[HTML]{000000}{\fontsize{11}{11}\selectfont{Program\ for\ non-profits,\ Indigenous\ governments,\ academic\ institutions\ for\ activities\ that\ align\ with\ recovery\ actions\ identified\ in\ SARA\ recovery\ documents\ and/or\ COSEWIC\ assessment\ documents.\ Project\ must\ address\ one\ or\ more\ of\ 3\ broad\ categories:\ (1)\ Important\ habitat\ for\ aquatic\ species\ at\ risk\ is\ improved\ and/or\ managed\ to\ meet\ their\ recovery\ needs;\ (2)\ Threats\ to\ aquatic\ species\ at\ risk\ and/or\ their\ habitat\ are\ stopped,\ removed,\ and/or\ mitigated;\ (3)\ Collaboration\ and\ partnerships\ support\ the\ conservation\ and\ recovery\ of\ aquatic\ species\ at\ risk.\ Limited\ to\ at-risk\ species\ listed\ under\ COSEWIC\ and/or\ SARA\ as\ threatened,\ endangered\ or\ special\ concern.\ }}} \\

\noalign{\global\arrayrulewidth 0pt}\arrayrulecolor[HTML]{000000}





\multicolumn{1}{>{\raggedright}m{\dimexpr 5.52in+0\tabcolsep}}{\textcolor[HTML]{000000}{\fontsize{11}{11}\selectfont{Canada\ Nature\ Fund\ for\ Aquatic\ Species\ at\ Risk}}} & \multicolumn{1}{>{\raggedright}m{\dimexpr 60.28in+0\tabcolsep}}{\textcolor[HTML]{000000}{\fontsize{11}{11}\selectfont{Funding\ program\ aimed\ at\ addressing\ priority\ threats\ for\ aquatic\ species\ at\ risk\ listed\ as\ endangered,\ threatened\ or\ Special\ Concern\ by\ COSEWIC,\ as\ they\ align\ with\ existing\ federal,\ provincial\ or\ other\ local\ recovery\ plans.\ Limited\ to\ species\ in\ the\ Columbia\ and\ Fraser\ basins\ in\ BC,\ among\ other\ priority\ areas\ across\ Canada.\ Focus\ on\ multi-year,\ multi-partner\ initiatives\ that\ apply\ an\ ecosystem\ or\ multi-species\ approach\ and\ create\ a\ legacy\ by\ enabling\ recovery\ actions\ that\ carry\ beyond\ the\ life\ of\ the\ funding\ program.\ Amounts\ from\ \$100K-\$1M\ available\ per\ year.}}} \\

\noalign{\global\arrayrulewidth 0pt}\arrayrulecolor[HTML]{000000}





\multicolumn{1}{>{\raggedright}m{\dimexpr 5.52in+0\tabcolsep}}{\textcolor[HTML]{000000}{\fontsize{11}{11}\selectfont{BC\ Salmon\ Restoration\ and\ Innovation\ Fund}}} & \multicolumn{1}{>{\raggedright}m{\dimexpr 60.28in+0\tabcolsep}}{\textcolor[HTML]{000000}{\fontsize{11}{11}\selectfont{Funding\ for\ Indigenous\ enterprises,\ academia,\ industry\ associations,\ stewardship\ groups\ and\ commercial\ groups\ to\ support\ initiatives\ that\ support\ the\ protection\ and\ restoration\ of\ wild\ Pacific\ salmon\ and\ other\ BC\ fish\ stocks\ or\ ensure\ fish\ and\ seafood\ sector\ in\ BC\ is\ environmentally\ and\ economically\ sustainable.\ Five\ main\ priorities\ including\ species\ of\ concern\ rebuilding\ through\ habitat\ restoration\ with\ priority\ for\ projects\ that\ are\ part\ of\ a\ watershed-scale\ restoration\ plan/prioritization\ effort;\ build\ on\ successful\ previous\ restoration\ efforts;\ focus\ on\ critical\ habitat\ and/or\ the\ rehabilitation\ of\ natural\ ecosystem\ processes.}}} \\

\noalign{\global\arrayrulewidth 0pt}\arrayrulecolor[HTML]{000000}





\multicolumn{1}{>{\raggedright}m{\dimexpr 5.52in+0\tabcolsep}}{\textcolor[HTML]{000000}{\fontsize{11}{11}\selectfont{Aboriginal\ Fund\ for\ Species\ at\ Risk}}} & \multicolumn{1}{>{\raggedright}m{\dimexpr 60.28in+0\tabcolsep}}{\textcolor[HTML]{000000}{\fontsize{11}{11}\selectfont{Program\ for\ Indigenous\ groups\ for\ activities\ that\ align\ with\ recovery\ actions\ identified\ in\ SARA\ recovery\ documents\ and/or\ COSEWIC\ assessment\ documents\ for\ species\ listed\ as\ Endangered,\ Threatened,\ or\ Special\ Concern\ by\ SARA\ or\ COSEWIC.\ Project\ must\ address\ one\ or\ more\ of\ 4\ broad\ categories:\ (1)\ Habitat\ for\ species\ at\ risk\ is\ improved\ and/or\ managed\ to\ meet\ their\ recovery\ needs;\ (2)\ Threats\ to\ species\ at\ risk\ and/or\ their\ habitat\ are\ stopped,\ removed\ and/or\ mitigated;\ (3)\ Collaboration,\ information\ sharing\ and\ partnership\ between\ Indigenous\ communities,\ governments\ and\ organizations\ and\ other\ interested\ parties\ (e.g.\ federal/provincial/territorial\ governments,\ academia,\ industry,\ private\ sector)\ is\ enhanced;\ and\ (4)\ Capacity\ within\ Indigenous\ communities,\ to\ lead\ in\ the\ stewardship\ of\ species\ at\ risk\ and\ contribute\ to\ broader\ SARA\ implementation,\ is\ strengthened.\ }}} \\

\noalign{\global\arrayrulewidth 0pt}\arrayrulecolor[HTML]{000000}





\multicolumn{1}{>{\raggedright}m{\dimexpr 5.52in+0\tabcolsep}}{\textcolor[HTML]{000000}{\fontsize{11}{11}\selectfont{Federal\ Gas\ Tax\ Fund\ -\ Community\ Works\ Fund}}} & \multicolumn{1}{>{\raggedright}m{\dimexpr 60.28in+0\tabcolsep}}{\textcolor[HTML]{000000}{\fontsize{11}{11}\selectfont{Funding\ available\ to\ local\ governments\ from\ federal\ gas\ tax,\ with\ funds\ to\ be\ allocated\ for\ a\ variety\ of\ municipal\ projects/initiatives,\ including\ local\ roads/bridges\ and\ disaster\ mitigation.}}} \\

\noalign{\global\arrayrulewidth 0pt}\arrayrulecolor[HTML]{000000}





\multicolumn{1}{>{\raggedright}m{\dimexpr 5.52in+0\tabcolsep}}{\textcolor[HTML]{000000}{\fontsize{11}{11}\selectfont{Disaster\ Mitigation\ and\ Adaptation\ Fund}}} & \multicolumn{1}{>{\raggedright}m{\dimexpr 60.28in+0\tabcolsep}}{\textcolor[HTML]{000000}{\fontsize{11}{11}\selectfont{For\ those\ projects\ where\ flood\ risk\ is\ high:\ Funding\ available\ to\ local,\ regional\ and\ provincial\ governments,\ private\ sector,\ non-profit\ organizations,\ and\ Indigenous\ groups\ for\ projects\ aimed\ at\ reducing\ the\ socio-economic,\ environmental\ and\ cultural\ impacts\ triggered\ by\ natural\ hazards\ and\ extreme\ weather\ events\ and\ taking\ into\ consideration\ current\ and\ future\ impacts\ of\ climate\ change\ in\ communities\ and\ infrastructure\ at\ high\ risk.\ Includes\ both\ new\ construction\ of\ public\ infrastructure\ and\ modification/reinforcement\ of\ existing\ infrastructure.\ Projects\ must\ have\ a\ minimum\ of\ \$20\ M\ in\ eligible\ expenditures\ and\ can\ be\ bundled\ together.}}} \\

\noalign{\global\arrayrulewidth 0pt}\arrayrulecolor[HTML]{000000}





\multicolumn{1}{>{\raggedright}m{\dimexpr 5.52in+0\tabcolsep}}{\textcolor[HTML]{000000}{\fontsize{11}{11}\selectfont{Community\ Gaming\ Grants}}} & \multicolumn{1}{>{\raggedright}m{\dimexpr 60.28in+0\tabcolsep}}{\textcolor[HTML]{000000}{\fontsize{11}{11}\selectfont{Funding\ for\ non-profit\ organizations\ (check\ funding\ program\ guidelines\ for\ specific\ eligibility\ requirements)\ for\ programs\ that\ help\ to\ protect\ and\ improve\ the\ environment\ by:\ (1)\ Conserving\ or\ revitalizing\ local\ ecosystems,\ (2)\ Reducing\ greenhouse\ gas\ emissions,\ (3)\ Providing\ community\ education\ or\ engagement\ opportunities\ related\ to\ the\ environment\ and\ agriculture\ or\ (4)\ Supporting\ the\ welfare\ of\ domestic\ animals\ and/or\ wildlife.\ Grants\ range\ from\ \$100K-250K\ per\ year.}}} \\

\noalign{\global\arrayrulewidth 0pt}\arrayrulecolor[HTML]{000000}





\multicolumn{1}{>{\raggedright}m{\dimexpr 5.52in+0\tabcolsep}}{\textcolor[HTML]{000000}{\fontsize{11}{11}\selectfont{Sitka\ Foundation}}} & \multicolumn{1}{>{\raggedright}m{\dimexpr 60.28in+0\tabcolsep}}{\textcolor[HTML]{000000}{\fontsize{11}{11}\selectfont{Funding\ for\ registered\ charities,\ universities,\ and\ government\ agencies\ (qualified\ Canadian\ organizations)\ for\ projects\ related\ to\ coastline\ and\ watershed\ conservation\ and\ climate\ change\ in\ 4\ key\ areas:\ (1)\ land,\ water,\ and\ ocean\ conservation,\ (2)\ scientific\ research\ for\ nature\ and\ the\ environment,\ (3)\ \ public\ engagement\ around\ the\ importance\ of\ a\ healthy\ environment,\ (4)\ innovative\ conservation\ efforts\ in\ Canadian\ communities,\ at\ the\ local,\ provincial,\ and\ federal\ levels}}} \\

\noalign{\global\arrayrulewidth 0pt}\arrayrulecolor[HTML]{000000}





\multicolumn{1}{>{\raggedright}m{\dimexpr 5.52in+0\tabcolsep}}{\textcolor[HTML]{000000}{\fontsize{11}{11}\selectfont{TULA\ Foundation}}} & \multicolumn{1}{>{\raggedright}m{\dimexpr 60.28in+0\tabcolsep}}{\textcolor[HTML]{000000}{\fontsize{11}{11}\selectfont{Supports\ various\ environmental\ programs\ of\ interest\ to\ the\ Foundation\ on\ a\ case-by-case\ basis.}}} \\

\noalign{\global\arrayrulewidth 0pt}\arrayrulecolor[HTML]{000000}





\multicolumn{1}{>{\raggedright}m{\dimexpr 5.52in+0\tabcolsep}}{\textcolor[HTML]{000000}{\fontsize{11}{11}\selectfont{Vancouver\ Foundation}}} & \multicolumn{1}{>{\raggedright}m{\dimexpr 60.28in+0\tabcolsep}}{\textcolor[HTML]{000000}{\fontsize{11}{11}\selectfont{Granting\ agency\ for\ community,\ social\ and\ environmental\ initiatives\ for\ qualified\ Canadian\ organizations\ (charitable\ organizations,\ universities,\ government\ agencies).\ Granting\ programs\ change\ on\ an\ annual\ basis.}}} \\

\noalign{\global\arrayrulewidth 0pt}\arrayrulecolor[HTML]{000000}





\multicolumn{1}{>{\raggedright}m{\dimexpr 5.52in+0\tabcolsep}}{\textcolor[HTML]{000000}{\fontsize{11}{11}\selectfont{BC\ Conservation\ Foundation\ Small\ Project\ Fund}}} & \multicolumn{1}{>{\raggedright}m{\dimexpr 60.28in+0\tabcolsep}}{\textcolor[HTML]{000000}{\fontsize{11}{11}\selectfont{Funding\ available\ to\ Non-profits,\ fish\ and\ wildlife\ clubs\ (sportsmen’s\ associations),\ businesses,\ local/regional\ governments,\ public\ organizations\ and\ First\ Nations\ for\ projects\ with\ demonstrated\ positive\ impact\ for\ fish,\ wildlife\ and\ habitat,\ including\ outreach\ programs.\ Preference\ given\ to\ projects\ where\ BCCF\ is\ not\ the\ sole\ funder.}}} \\

\noalign{\global\arrayrulewidth 0pt}\arrayrulecolor[HTML]{000000}





\multicolumn{1}{>{\raggedright}m{\dimexpr 5.52in+0\tabcolsep}}{\textcolor[HTML]{000000}{\fontsize{11}{11}\selectfont{Real\ Estate\ Foundation\ of\ BC\ General\ Grants}}} & \multicolumn{1}{>{\raggedright}m{\dimexpr 60.28in+0\tabcolsep}}{\textcolor[HTML]{000000}{\fontsize{11}{11}\selectfont{Funding\ for\ First\ Nations,\ charities\ and\ societies,\ non-governmental\ organizations,\ universities\ and\ colleges,\ trade\ associations,\ local\ and\ regional\ governments,\ and\ social\ enterprises\ registered\ as\ C3s\ for\ sustainable\ land\ use\ and\ real\ estate\ practices\ in\ BC.\ Funds\ up\ to\ 50\%\ of\ cash\ portion\ of\ a\ project.}}} \\

\noalign{\global\arrayrulewidth 0pt}\arrayrulecolor[HTML]{000000}

\hhline{>{\arrayrulecolor[HTML]{666666}\global\arrayrulewidth=1.5pt}->{\arrayrulecolor[HTML]{666666}\global\arrayrulewidth=1.5pt}-}


\end{longtable}

\arrayrulecolor[HTML]{000000}

\global\setlength{\arrayrulewidth}{\Oldarrayrulewidth}

\global\setlength{\tabcolsep}{\Oldtabcolsep}

\renewcommand*{\arraystretch}{1}

\part{Appendix B}

\chapter*{Priority Barriers}\label{priority-barriers}
\addcontentsline{toc}{chapter}{Priority Barriers}

\markboth{Priority Barriers}{Priority Barriers}

\global\setlength{\Oldarrayrulewidth}{\arrayrulewidth}

\global\setlength{\Oldtabcolsep}{\tabcolsep}

\setlength{\tabcolsep}{0pt}

\renewcommand*{\arraystretch}{1.5}



\providecommand{\ascline}[3]{\noalign{\global\arrayrulewidth #1}\arrayrulecolor[HTML]{#2}\cline{#3}}

\begin{longtable}[c]{|p{3.20in}|p{1.56in}|p{1.08in}|p{2.44in}|p{1.13in}|p{1.29in}|p{1.97in}|p{2.17in}|p{1.10in}|p{1.46in}|p{0.76in}|p{5.36in}|p{1.46in}|p{4.18in}}

\caption{\label{tbl-keyact}Priority barriers}

\tabularnewline

\hhline{>{\arrayrulecolor[HTML]{666666}\global\arrayrulewidth=1.5pt}->{\arrayrulecolor[HTML]{666666}\global\arrayrulewidth=1.5pt}->{\arrayrulecolor[HTML]{666666}\global\arrayrulewidth=1.5pt}->{\arrayrulecolor[HTML]{666666}\global\arrayrulewidth=1.5pt}->{\arrayrulecolor[HTML]{666666}\global\arrayrulewidth=1.5pt}->{\arrayrulecolor[HTML]{666666}\global\arrayrulewidth=1.5pt}->{\arrayrulecolor[HTML]{666666}\global\arrayrulewidth=1.5pt}->{\arrayrulecolor[HTML]{666666}\global\arrayrulewidth=1.5pt}->{\arrayrulecolor[HTML]{666666}\global\arrayrulewidth=1.5pt}->{\arrayrulecolor[HTML]{666666}\global\arrayrulewidth=1.5pt}->{\arrayrulecolor[HTML]{666666}\global\arrayrulewidth=1.5pt}->{\arrayrulecolor[HTML]{666666}\global\arrayrulewidth=1.5pt}->{\arrayrulecolor[HTML]{666666}\global\arrayrulewidth=1.5pt}->{\arrayrulecolor[HTML]{666666}\global\arrayrulewidth=1.5pt}-}

\multicolumn{1}{>{\cellcolor[HTML]{008270}\raggedright}m{\dimexpr 3.2in+0\tabcolsep}}{\textcolor[HTML]{FFFFFF}{\fontsize{11}{11}\selectfont{ID}}} & \multicolumn{1}{>{\cellcolor[HTML]{008270}\raggedright}m{\dimexpr 1.56in+0\tabcolsep}}{\textcolor[HTML]{FFFFFF}{\fontsize{11}{11}\selectfont{Stream\ name}}} & \multicolumn{1}{>{\cellcolor[HTML]{008270}\raggedleft}m{\dimexpr 1.08in+0\tabcolsep}}{\textcolor[HTML]{FFFFFF}{\fontsize{11}{11}\selectfont{Road\ name}}} & \multicolumn{1}{>{\cellcolor[HTML]{008270}\raggedright}m{\dimexpr 2.44in+0\tabcolsep}}{\textcolor[HTML]{FFFFFF}{\fontsize{11}{11}\selectfont{Owner}}} & \multicolumn{1}{>{\cellcolor[HTML]{008270}\raggedleft}m{\dimexpr 1.13in+0\tabcolsep}}{\textcolor[HTML]{FFFFFF}{\fontsize{11}{11}\selectfont{Proposed\ fix}}} & \multicolumn{1}{>{\cellcolor[HTML]{008270}\raggedright}m{\dimexpr 1.29in+0\tabcolsep}}{\textcolor[HTML]{FFFFFF}{\fontsize{11}{11}\selectfont{Estimated\ cost}}} & \multicolumn{1}{>{\cellcolor[HTML]{008270}\raggedright}m{\dimexpr 1.97in+0\tabcolsep}}{\textcolor[HTML]{FFFFFF}{\fontsize{11}{11}\selectfont{Upstream\ habitat\ Quality}}} & \multicolumn{1}{>{\cellcolor[HTML]{008270}\raggedright}m{\dimexpr 2.17in+0\tabcolsep}}{\textcolor[HTML]{FFFFFF}{\fontsize{11}{11}\selectfont{Barrier\ type}}} & \multicolumn{1}{>{\cellcolor[HTML]{008270}\raggedright}m{\dimexpr 1.1in+0\tabcolsep}}{\textcolor[HTML]{FFFFFF}{\fontsize{11}{11}\selectfont{Habitat\ gain}}} & \multicolumn{1}{>{\cellcolor[HTML]{008270}\raggedright}m{\dimexpr 1.46in+0\tabcolsep}}{\textcolor[HTML]{FFFFFF}{\fontsize{11}{11}\selectfont{Cost\ Benefit\ ratio}}} & \multicolumn{1}{>{\cellcolor[HTML]{008270}\raggedright}m{\dimexpr 0.76in+0\tabcolsep}}{\textcolor[HTML]{FFFFFF}{\fontsize{11}{11}\selectfont{Priority}}} & \multicolumn{1}{>{\cellcolor[HTML]{008270}\raggedright}m{\dimexpr 5.36in+0\tabcolsep}}{\textcolor[HTML]{FFFFFF}{\fontsize{11}{11}\selectfont{Next\ steps}}} & \multicolumn{1}{>{\cellcolor[HTML]{008270}\raggedright}m{\dimexpr 1.46in+0\tabcolsep}}{\textcolor[HTML]{FFFFFF}{\fontsize{11}{11}\selectfont{Reason}}} & \multicolumn{1}{>{\cellcolor[HTML]{008270}\raggedright}m{\dimexpr 4.18in+0\tabcolsep}}{\textcolor[HTML]{FFFFFF}{\fontsize{11}{11}\selectfont{Notes}}} \\

\noalign{\global\arrayrulewidth 0pt}\arrayrulecolor[HTML]{000000}

\hhline{>{\arrayrulecolor[HTML]{666666}\global\arrayrulewidth=1.5pt}->{\arrayrulecolor[HTML]{666666}\global\arrayrulewidth=1.5pt}->{\arrayrulecolor[HTML]{666666}\global\arrayrulewidth=1.5pt}->{\arrayrulecolor[HTML]{666666}\global\arrayrulewidth=1.5pt}->{\arrayrulecolor[HTML]{666666}\global\arrayrulewidth=1.5pt}->{\arrayrulecolor[HTML]{666666}\global\arrayrulewidth=1.5pt}->{\arrayrulecolor[HTML]{666666}\global\arrayrulewidth=1.5pt}->{\arrayrulecolor[HTML]{666666}\global\arrayrulewidth=1.5pt}->{\arrayrulecolor[HTML]{666666}\global\arrayrulewidth=1.5pt}->{\arrayrulecolor[HTML]{666666}\global\arrayrulewidth=1.5pt}->{\arrayrulecolor[HTML]{666666}\global\arrayrulewidth=1.5pt}->{\arrayrulecolor[HTML]{666666}\global\arrayrulewidth=1.5pt}->{\arrayrulecolor[HTML]{666666}\global\arrayrulewidth=1.5pt}->{\arrayrulecolor[HTML]{666666}\global\arrayrulewidth=1.5pt}-}\endfirsthead 

\hhline{>{\arrayrulecolor[HTML]{666666}\global\arrayrulewidth=1.5pt}->{\arrayrulecolor[HTML]{666666}\global\arrayrulewidth=1.5pt}->{\arrayrulecolor[HTML]{666666}\global\arrayrulewidth=1.5pt}->{\arrayrulecolor[HTML]{666666}\global\arrayrulewidth=1.5pt}->{\arrayrulecolor[HTML]{666666}\global\arrayrulewidth=1.5pt}->{\arrayrulecolor[HTML]{666666}\global\arrayrulewidth=1.5pt}->{\arrayrulecolor[HTML]{666666}\global\arrayrulewidth=1.5pt}->{\arrayrulecolor[HTML]{666666}\global\arrayrulewidth=1.5pt}->{\arrayrulecolor[HTML]{666666}\global\arrayrulewidth=1.5pt}->{\arrayrulecolor[HTML]{666666}\global\arrayrulewidth=1.5pt}->{\arrayrulecolor[HTML]{666666}\global\arrayrulewidth=1.5pt}->{\arrayrulecolor[HTML]{666666}\global\arrayrulewidth=1.5pt}->{\arrayrulecolor[HTML]{666666}\global\arrayrulewidth=1.5pt}->{\arrayrulecolor[HTML]{666666}\global\arrayrulewidth=1.5pt}-}

\multicolumn{1}{>{\cellcolor[HTML]{008270}\raggedright}m{\dimexpr 3.2in+0\tabcolsep}}{\textcolor[HTML]{FFFFFF}{\fontsize{11}{11}\selectfont{ID}}} & \multicolumn{1}{>{\cellcolor[HTML]{008270}\raggedright}m{\dimexpr 1.56in+0\tabcolsep}}{\textcolor[HTML]{FFFFFF}{\fontsize{11}{11}\selectfont{Stream\ name}}} & \multicolumn{1}{>{\cellcolor[HTML]{008270}\raggedleft}m{\dimexpr 1.08in+0\tabcolsep}}{\textcolor[HTML]{FFFFFF}{\fontsize{11}{11}\selectfont{Road\ name}}} & \multicolumn{1}{>{\cellcolor[HTML]{008270}\raggedright}m{\dimexpr 2.44in+0\tabcolsep}}{\textcolor[HTML]{FFFFFF}{\fontsize{11}{11}\selectfont{Owner}}} & \multicolumn{1}{>{\cellcolor[HTML]{008270}\raggedleft}m{\dimexpr 1.13in+0\tabcolsep}}{\textcolor[HTML]{FFFFFF}{\fontsize{11}{11}\selectfont{Proposed\ fix}}} & \multicolumn{1}{>{\cellcolor[HTML]{008270}\raggedright}m{\dimexpr 1.29in+0\tabcolsep}}{\textcolor[HTML]{FFFFFF}{\fontsize{11}{11}\selectfont{Estimated\ cost}}} & \multicolumn{1}{>{\cellcolor[HTML]{008270}\raggedright}m{\dimexpr 1.97in+0\tabcolsep}}{\textcolor[HTML]{FFFFFF}{\fontsize{11}{11}\selectfont{Upstream\ habitat\ Quality}}} & \multicolumn{1}{>{\cellcolor[HTML]{008270}\raggedright}m{\dimexpr 2.17in+0\tabcolsep}}{\textcolor[HTML]{FFFFFF}{\fontsize{11}{11}\selectfont{Barrier\ type}}} & \multicolumn{1}{>{\cellcolor[HTML]{008270}\raggedright}m{\dimexpr 1.1in+0\tabcolsep}}{\textcolor[HTML]{FFFFFF}{\fontsize{11}{11}\selectfont{Habitat\ gain}}} & \multicolumn{1}{>{\cellcolor[HTML]{008270}\raggedright}m{\dimexpr 1.46in+0\tabcolsep}}{\textcolor[HTML]{FFFFFF}{\fontsize{11}{11}\selectfont{Cost\ Benefit\ ratio}}} & \multicolumn{1}{>{\cellcolor[HTML]{008270}\raggedright}m{\dimexpr 0.76in+0\tabcolsep}}{\textcolor[HTML]{FFFFFF}{\fontsize{11}{11}\selectfont{Priority}}} & \multicolumn{1}{>{\cellcolor[HTML]{008270}\raggedright}m{\dimexpr 5.36in+0\tabcolsep}}{\textcolor[HTML]{FFFFFF}{\fontsize{11}{11}\selectfont{Next\ steps}}} & \multicolumn{1}{>{\cellcolor[HTML]{008270}\raggedright}m{\dimexpr 1.46in+0\tabcolsep}}{\textcolor[HTML]{FFFFFF}{\fontsize{11}{11}\selectfont{Reason}}} & \multicolumn{1}{>{\cellcolor[HTML]{008270}\raggedright}m{\dimexpr 4.18in+0\tabcolsep}}{\textcolor[HTML]{FFFFFF}{\fontsize{11}{11}\selectfont{Notes}}} \\

\noalign{\global\arrayrulewidth 0pt}\arrayrulecolor[HTML]{000000}

\hhline{>{\arrayrulecolor[HTML]{666666}\global\arrayrulewidth=1.5pt}->{\arrayrulecolor[HTML]{666666}\global\arrayrulewidth=1.5pt}->{\arrayrulecolor[HTML]{666666}\global\arrayrulewidth=1.5pt}->{\arrayrulecolor[HTML]{666666}\global\arrayrulewidth=1.5pt}->{\arrayrulecolor[HTML]{666666}\global\arrayrulewidth=1.5pt}->{\arrayrulecolor[HTML]{666666}\global\arrayrulewidth=1.5pt}->{\arrayrulecolor[HTML]{666666}\global\arrayrulewidth=1.5pt}->{\arrayrulecolor[HTML]{666666}\global\arrayrulewidth=1.5pt}->{\arrayrulecolor[HTML]{666666}\global\arrayrulewidth=1.5pt}->{\arrayrulecolor[HTML]{666666}\global\arrayrulewidth=1.5pt}->{\arrayrulecolor[HTML]{666666}\global\arrayrulewidth=1.5pt}->{\arrayrulecolor[HTML]{666666}\global\arrayrulewidth=1.5pt}->{\arrayrulecolor[HTML]{666666}\global\arrayrulewidth=1.5pt}->{\arrayrulecolor[HTML]{666666}\global\arrayrulewidth=1.5pt}-}\endhead



\multicolumn{1}{>{\raggedright}m{\dimexpr 3.2in+0\tabcolsep}}{\textcolor[HTML]{000000}{\fontsize{11}{11}\selectfont{6cd84598-ee60-48a7-8215-9482dd9d2ed5}}} & \multicolumn{1}{>{\raggedright}m{\dimexpr 1.56in+0\tabcolsep}}{\textcolor[HTML]{000000}{\fontsize{11}{11}\selectfont{MacSweens\ Brook}}} & \multicolumn{1}{>{\raggedleft}m{\dimexpr 1.08in+0\tabcolsep}}{\textcolor[HTML]{000000}{\fontsize{11}{11}\selectfont{}}} & \multicolumn{1}{>{\raggedright}m{\dimexpr 2.44in+0\tabcolsep}}{\textcolor[HTML]{000000}{\fontsize{11}{11}\selectfont{NS\ Department\ of\ Public\ Works}}} & \multicolumn{1}{>{\raggedleft}m{\dimexpr 1.13in+0\tabcolsep}}{\textcolor[HTML]{000000}{\fontsize{11}{11}\selectfont{}}} & \multicolumn{1}{>{\raggedright}m{\dimexpr 1.29in+0\tabcolsep}}{\textcolor[HTML]{000000}{\fontsize{11}{11}\selectfont{N/A}}} & \multicolumn{1}{>{\raggedright}m{\dimexpr 1.97in+0\tabcolsep}}{\textcolor[HTML]{000000}{\fontsize{11}{11}\selectfont{To\ be\ Assessed}}} & \multicolumn{1}{>{\raggedright}m{\dimexpr 2.17in+0\tabcolsep}}{\textcolor[HTML]{000000}{\fontsize{11}{11}\selectfont{Velocity\ and\ Vertical\ Barrier}}} & \multicolumn{1}{>{\raggedright}m{\dimexpr 1.1in+0\tabcolsep}}{\textcolor[HTML]{000000}{\fontsize{11}{11}\selectfont{1.9km}}} & \multicolumn{1}{>{\raggedright}m{\dimexpr 1.46in+0\tabcolsep}}{\textcolor[HTML]{000000}{\fontsize{11}{11}\selectfont{N/A}}} & \multicolumn{1}{>{\raggedright}m{\dimexpr 0.76in+0\tabcolsep}}{\textcolor[HTML]{000000}{\fontsize{11}{11}\selectfont{High}}} & \multicolumn{1}{>{\raggedright}m{\dimexpr 5.36in+0\tabcolsep}}{\textcolor[HTML]{000000}{\fontsize{11}{11}\selectfont{Habitat\ Assessment\ and\ Engagement\ with\ NS\ Department\ of\ Public\ Works}}} & \multicolumn{1}{>{\raggedright}m{\dimexpr 1.46in+0\tabcolsep}}{\textcolor[HTML]{000000}{\fontsize{11}{11}\selectfont{Restore\ Passage}}} & \multicolumn{1}{>{\raggedright}m{\dimexpr 4.18in+0\tabcolsep}}{\textcolor[HTML]{000000}{\fontsize{11}{11}\selectfont{basically\ a\ water\ fall,\ 3\ foot\ drop\ and\ large\ velocity\ barrier}}} \\

\noalign{\global\arrayrulewidth 0pt}\arrayrulecolor[HTML]{000000}

\hhline{>{\arrayrulecolor[HTML]{666666}\global\arrayrulewidth=1.5pt}->{\arrayrulecolor[HTML]{666666}\global\arrayrulewidth=1.5pt}->{\arrayrulecolor[HTML]{666666}\global\arrayrulewidth=1.5pt}->{\arrayrulecolor[HTML]{666666}\global\arrayrulewidth=1.5pt}->{\arrayrulecolor[HTML]{666666}\global\arrayrulewidth=1.5pt}->{\arrayrulecolor[HTML]{666666}\global\arrayrulewidth=1.5pt}->{\arrayrulecolor[HTML]{666666}\global\arrayrulewidth=1.5pt}->{\arrayrulecolor[HTML]{666666}\global\arrayrulewidth=1.5pt}->{\arrayrulecolor[HTML]{666666}\global\arrayrulewidth=1.5pt}->{\arrayrulecolor[HTML]{666666}\global\arrayrulewidth=1.5pt}->{\arrayrulecolor[HTML]{666666}\global\arrayrulewidth=1.5pt}->{\arrayrulecolor[HTML]{666666}\global\arrayrulewidth=1.5pt}->{\arrayrulecolor[HTML]{666666}\global\arrayrulewidth=1.5pt}->{\arrayrulecolor[HTML]{666666}\global\arrayrulewidth=1.5pt}-}


\end{longtable}

\arrayrulecolor[HTML]{000000}

\global\setlength{\arrayrulewidth}{\Oldarrayrulewidth}

\global\setlength{\tabcolsep}{\Oldtabcolsep}

\renewcommand*{\arraystretch}{1}

\chapter*{Non-Actionable Barriers}\label{non-actionable-barriers}
\addcontentsline{toc}{chapter}{Non-Actionable Barriers}

\markboth{Non-Actionable Barriers}{Non-Actionable Barriers}

\global\setlength{\Oldarrayrulewidth}{\arrayrulewidth}

\global\setlength{\Oldtabcolsep}{\tabcolsep}

\setlength{\tabcolsep}{0pt}

\renewcommand*{\arraystretch}{1.5}



\providecommand{\ascline}[3]{\noalign{\global\arrayrulewidth #1}\arrayrulecolor[HTML]{#2}\cline{#3}}

\begin{longtable}[c]{|p{3.20in}|p{1.56in}|p{1.08in}|p{2.44in}|p{1.13in}|p{1.29in}|p{1.97in}|p{2.17in}|p{1.10in}|p{1.46in}|p{0.76in}|p{5.36in}|p{1.46in}|p{4.18in}}

\caption{\label{tbl-keyact}Priority barriers}

\tabularnewline

\hhline{>{\arrayrulecolor[HTML]{666666}\global\arrayrulewidth=1.5pt}->{\arrayrulecolor[HTML]{666666}\global\arrayrulewidth=1.5pt}->{\arrayrulecolor[HTML]{666666}\global\arrayrulewidth=1.5pt}->{\arrayrulecolor[HTML]{666666}\global\arrayrulewidth=1.5pt}->{\arrayrulecolor[HTML]{666666}\global\arrayrulewidth=1.5pt}->{\arrayrulecolor[HTML]{666666}\global\arrayrulewidth=1.5pt}->{\arrayrulecolor[HTML]{666666}\global\arrayrulewidth=1.5pt}->{\arrayrulecolor[HTML]{666666}\global\arrayrulewidth=1.5pt}->{\arrayrulecolor[HTML]{666666}\global\arrayrulewidth=1.5pt}->{\arrayrulecolor[HTML]{666666}\global\arrayrulewidth=1.5pt}->{\arrayrulecolor[HTML]{666666}\global\arrayrulewidth=1.5pt}->{\arrayrulecolor[HTML]{666666}\global\arrayrulewidth=1.5pt}->{\arrayrulecolor[HTML]{666666}\global\arrayrulewidth=1.5pt}->{\arrayrulecolor[HTML]{666666}\global\arrayrulewidth=1.5pt}-}

\multicolumn{1}{>{\cellcolor[HTML]{008270}\raggedright}m{\dimexpr 3.2in+0\tabcolsep}}{\textcolor[HTML]{FFFFFF}{\fontsize{11}{11}\selectfont{ID}}} & \multicolumn{1}{>{\cellcolor[HTML]{008270}\raggedright}m{\dimexpr 1.56in+0\tabcolsep}}{\textcolor[HTML]{FFFFFF}{\fontsize{11}{11}\selectfont{Stream\ name}}} & \multicolumn{1}{>{\cellcolor[HTML]{008270}\raggedleft}m{\dimexpr 1.08in+0\tabcolsep}}{\textcolor[HTML]{FFFFFF}{\fontsize{11}{11}\selectfont{Road\ name}}} & \multicolumn{1}{>{\cellcolor[HTML]{008270}\raggedright}m{\dimexpr 2.44in+0\tabcolsep}}{\textcolor[HTML]{FFFFFF}{\fontsize{11}{11}\selectfont{Owner}}} & \multicolumn{1}{>{\cellcolor[HTML]{008270}\raggedleft}m{\dimexpr 1.13in+0\tabcolsep}}{\textcolor[HTML]{FFFFFF}{\fontsize{11}{11}\selectfont{Proposed\ fix}}} & \multicolumn{1}{>{\cellcolor[HTML]{008270}\raggedright}m{\dimexpr 1.29in+0\tabcolsep}}{\textcolor[HTML]{FFFFFF}{\fontsize{11}{11}\selectfont{Estimated\ cost}}} & \multicolumn{1}{>{\cellcolor[HTML]{008270}\raggedright}m{\dimexpr 1.97in+0\tabcolsep}}{\textcolor[HTML]{FFFFFF}{\fontsize{11}{11}\selectfont{Upstream\ habitat\ Quality}}} & \multicolumn{1}{>{\cellcolor[HTML]{008270}\raggedright}m{\dimexpr 2.17in+0\tabcolsep}}{\textcolor[HTML]{FFFFFF}{\fontsize{11}{11}\selectfont{Barrier\ type}}} & \multicolumn{1}{>{\cellcolor[HTML]{008270}\raggedright}m{\dimexpr 1.1in+0\tabcolsep}}{\textcolor[HTML]{FFFFFF}{\fontsize{11}{11}\selectfont{Habitat\ gain}}} & \multicolumn{1}{>{\cellcolor[HTML]{008270}\raggedright}m{\dimexpr 1.46in+0\tabcolsep}}{\textcolor[HTML]{FFFFFF}{\fontsize{11}{11}\selectfont{Cost\ Benefit\ ratio}}} & \multicolumn{1}{>{\cellcolor[HTML]{008270}\raggedright}m{\dimexpr 0.76in+0\tabcolsep}}{\textcolor[HTML]{FFFFFF}{\fontsize{11}{11}\selectfont{Priority}}} & \multicolumn{1}{>{\cellcolor[HTML]{008270}\raggedright}m{\dimexpr 5.36in+0\tabcolsep}}{\textcolor[HTML]{FFFFFF}{\fontsize{11}{11}\selectfont{Next\ steps}}} & \multicolumn{1}{>{\cellcolor[HTML]{008270}\raggedright}m{\dimexpr 1.46in+0\tabcolsep}}{\textcolor[HTML]{FFFFFF}{\fontsize{11}{11}\selectfont{Reason}}} & \multicolumn{1}{>{\cellcolor[HTML]{008270}\raggedright}m{\dimexpr 4.18in+0\tabcolsep}}{\textcolor[HTML]{FFFFFF}{\fontsize{11}{11}\selectfont{Notes}}} \\

\noalign{\global\arrayrulewidth 0pt}\arrayrulecolor[HTML]{000000}

\hhline{>{\arrayrulecolor[HTML]{666666}\global\arrayrulewidth=1.5pt}->{\arrayrulecolor[HTML]{666666}\global\arrayrulewidth=1.5pt}->{\arrayrulecolor[HTML]{666666}\global\arrayrulewidth=1.5pt}->{\arrayrulecolor[HTML]{666666}\global\arrayrulewidth=1.5pt}->{\arrayrulecolor[HTML]{666666}\global\arrayrulewidth=1.5pt}->{\arrayrulecolor[HTML]{666666}\global\arrayrulewidth=1.5pt}->{\arrayrulecolor[HTML]{666666}\global\arrayrulewidth=1.5pt}->{\arrayrulecolor[HTML]{666666}\global\arrayrulewidth=1.5pt}->{\arrayrulecolor[HTML]{666666}\global\arrayrulewidth=1.5pt}->{\arrayrulecolor[HTML]{666666}\global\arrayrulewidth=1.5pt}->{\arrayrulecolor[HTML]{666666}\global\arrayrulewidth=1.5pt}->{\arrayrulecolor[HTML]{666666}\global\arrayrulewidth=1.5pt}->{\arrayrulecolor[HTML]{666666}\global\arrayrulewidth=1.5pt}->{\arrayrulecolor[HTML]{666666}\global\arrayrulewidth=1.5pt}-}\endfirsthead 

\hhline{>{\arrayrulecolor[HTML]{666666}\global\arrayrulewidth=1.5pt}->{\arrayrulecolor[HTML]{666666}\global\arrayrulewidth=1.5pt}->{\arrayrulecolor[HTML]{666666}\global\arrayrulewidth=1.5pt}->{\arrayrulecolor[HTML]{666666}\global\arrayrulewidth=1.5pt}->{\arrayrulecolor[HTML]{666666}\global\arrayrulewidth=1.5pt}->{\arrayrulecolor[HTML]{666666}\global\arrayrulewidth=1.5pt}->{\arrayrulecolor[HTML]{666666}\global\arrayrulewidth=1.5pt}->{\arrayrulecolor[HTML]{666666}\global\arrayrulewidth=1.5pt}->{\arrayrulecolor[HTML]{666666}\global\arrayrulewidth=1.5pt}->{\arrayrulecolor[HTML]{666666}\global\arrayrulewidth=1.5pt}->{\arrayrulecolor[HTML]{666666}\global\arrayrulewidth=1.5pt}->{\arrayrulecolor[HTML]{666666}\global\arrayrulewidth=1.5pt}->{\arrayrulecolor[HTML]{666666}\global\arrayrulewidth=1.5pt}->{\arrayrulecolor[HTML]{666666}\global\arrayrulewidth=1.5pt}-}

\multicolumn{1}{>{\cellcolor[HTML]{008270}\raggedright}m{\dimexpr 3.2in+0\tabcolsep}}{\textcolor[HTML]{FFFFFF}{\fontsize{11}{11}\selectfont{ID}}} & \multicolumn{1}{>{\cellcolor[HTML]{008270}\raggedright}m{\dimexpr 1.56in+0\tabcolsep}}{\textcolor[HTML]{FFFFFF}{\fontsize{11}{11}\selectfont{Stream\ name}}} & \multicolumn{1}{>{\cellcolor[HTML]{008270}\raggedleft}m{\dimexpr 1.08in+0\tabcolsep}}{\textcolor[HTML]{FFFFFF}{\fontsize{11}{11}\selectfont{Road\ name}}} & \multicolumn{1}{>{\cellcolor[HTML]{008270}\raggedright}m{\dimexpr 2.44in+0\tabcolsep}}{\textcolor[HTML]{FFFFFF}{\fontsize{11}{11}\selectfont{Owner}}} & \multicolumn{1}{>{\cellcolor[HTML]{008270}\raggedleft}m{\dimexpr 1.13in+0\tabcolsep}}{\textcolor[HTML]{FFFFFF}{\fontsize{11}{11}\selectfont{Proposed\ fix}}} & \multicolumn{1}{>{\cellcolor[HTML]{008270}\raggedright}m{\dimexpr 1.29in+0\tabcolsep}}{\textcolor[HTML]{FFFFFF}{\fontsize{11}{11}\selectfont{Estimated\ cost}}} & \multicolumn{1}{>{\cellcolor[HTML]{008270}\raggedright}m{\dimexpr 1.97in+0\tabcolsep}}{\textcolor[HTML]{FFFFFF}{\fontsize{11}{11}\selectfont{Upstream\ habitat\ Quality}}} & \multicolumn{1}{>{\cellcolor[HTML]{008270}\raggedright}m{\dimexpr 2.17in+0\tabcolsep}}{\textcolor[HTML]{FFFFFF}{\fontsize{11}{11}\selectfont{Barrier\ type}}} & \multicolumn{1}{>{\cellcolor[HTML]{008270}\raggedright}m{\dimexpr 1.1in+0\tabcolsep}}{\textcolor[HTML]{FFFFFF}{\fontsize{11}{11}\selectfont{Habitat\ gain}}} & \multicolumn{1}{>{\cellcolor[HTML]{008270}\raggedright}m{\dimexpr 1.46in+0\tabcolsep}}{\textcolor[HTML]{FFFFFF}{\fontsize{11}{11}\selectfont{Cost\ Benefit\ ratio}}} & \multicolumn{1}{>{\cellcolor[HTML]{008270}\raggedright}m{\dimexpr 0.76in+0\tabcolsep}}{\textcolor[HTML]{FFFFFF}{\fontsize{11}{11}\selectfont{Priority}}} & \multicolumn{1}{>{\cellcolor[HTML]{008270}\raggedright}m{\dimexpr 5.36in+0\tabcolsep}}{\textcolor[HTML]{FFFFFF}{\fontsize{11}{11}\selectfont{Next\ steps}}} & \multicolumn{1}{>{\cellcolor[HTML]{008270}\raggedright}m{\dimexpr 1.46in+0\tabcolsep}}{\textcolor[HTML]{FFFFFF}{\fontsize{11}{11}\selectfont{Reason}}} & \multicolumn{1}{>{\cellcolor[HTML]{008270}\raggedright}m{\dimexpr 4.18in+0\tabcolsep}}{\textcolor[HTML]{FFFFFF}{\fontsize{11}{11}\selectfont{Notes}}} \\

\noalign{\global\arrayrulewidth 0pt}\arrayrulecolor[HTML]{000000}

\hhline{>{\arrayrulecolor[HTML]{666666}\global\arrayrulewidth=1.5pt}->{\arrayrulecolor[HTML]{666666}\global\arrayrulewidth=1.5pt}->{\arrayrulecolor[HTML]{666666}\global\arrayrulewidth=1.5pt}->{\arrayrulecolor[HTML]{666666}\global\arrayrulewidth=1.5pt}->{\arrayrulecolor[HTML]{666666}\global\arrayrulewidth=1.5pt}->{\arrayrulecolor[HTML]{666666}\global\arrayrulewidth=1.5pt}->{\arrayrulecolor[HTML]{666666}\global\arrayrulewidth=1.5pt}->{\arrayrulecolor[HTML]{666666}\global\arrayrulewidth=1.5pt}->{\arrayrulecolor[HTML]{666666}\global\arrayrulewidth=1.5pt}->{\arrayrulecolor[HTML]{666666}\global\arrayrulewidth=1.5pt}->{\arrayrulecolor[HTML]{666666}\global\arrayrulewidth=1.5pt}->{\arrayrulecolor[HTML]{666666}\global\arrayrulewidth=1.5pt}->{\arrayrulecolor[HTML]{666666}\global\arrayrulewidth=1.5pt}->{\arrayrulecolor[HTML]{666666}\global\arrayrulewidth=1.5pt}-}\endhead



\multicolumn{1}{>{\raggedright}m{\dimexpr 3.2in+0\tabcolsep}}{\textcolor[HTML]{000000}{\fontsize{11}{11}\selectfont{6cd84598-ee60-48a7-8215-9482dd9d2ed5}}} & \multicolumn{1}{>{\raggedright}m{\dimexpr 1.56in+0\tabcolsep}}{\textcolor[HTML]{000000}{\fontsize{11}{11}\selectfont{MacSweens\ Brook}}} & \multicolumn{1}{>{\raggedleft}m{\dimexpr 1.08in+0\tabcolsep}}{\textcolor[HTML]{000000}{\fontsize{11}{11}\selectfont{}}} & \multicolumn{1}{>{\raggedright}m{\dimexpr 2.44in+0\tabcolsep}}{\textcolor[HTML]{000000}{\fontsize{11}{11}\selectfont{NS\ Department\ of\ Public\ Works}}} & \multicolumn{1}{>{\raggedleft}m{\dimexpr 1.13in+0\tabcolsep}}{\textcolor[HTML]{000000}{\fontsize{11}{11}\selectfont{}}} & \multicolumn{1}{>{\raggedright}m{\dimexpr 1.29in+0\tabcolsep}}{\textcolor[HTML]{000000}{\fontsize{11}{11}\selectfont{N/A}}} & \multicolumn{1}{>{\raggedright}m{\dimexpr 1.97in+0\tabcolsep}}{\textcolor[HTML]{000000}{\fontsize{11}{11}\selectfont{To\ be\ Assessed}}} & \multicolumn{1}{>{\raggedright}m{\dimexpr 2.17in+0\tabcolsep}}{\textcolor[HTML]{000000}{\fontsize{11}{11}\selectfont{Velocity\ and\ Vertical\ Barrier}}} & \multicolumn{1}{>{\raggedright}m{\dimexpr 1.1in+0\tabcolsep}}{\textcolor[HTML]{000000}{\fontsize{11}{11}\selectfont{1.9km}}} & \multicolumn{1}{>{\raggedright}m{\dimexpr 1.46in+0\tabcolsep}}{\textcolor[HTML]{000000}{\fontsize{11}{11}\selectfont{N/A}}} & \multicolumn{1}{>{\raggedright}m{\dimexpr 0.76in+0\tabcolsep}}{\textcolor[HTML]{000000}{\fontsize{11}{11}\selectfont{High}}} & \multicolumn{1}{>{\raggedright}m{\dimexpr 5.36in+0\tabcolsep}}{\textcolor[HTML]{000000}{\fontsize{11}{11}\selectfont{Habitat\ Assessment\ and\ Engagement\ with\ NS\ Department\ of\ Public\ Works}}} & \multicolumn{1}{>{\raggedright}m{\dimexpr 1.46in+0\tabcolsep}}{\textcolor[HTML]{000000}{\fontsize{11}{11}\selectfont{Restore\ Passage}}} & \multicolumn{1}{>{\raggedright}m{\dimexpr 4.18in+0\tabcolsep}}{\textcolor[HTML]{000000}{\fontsize{11}{11}\selectfont{basically\ a\ water\ fall,\ 3\ foot\ drop\ and\ large\ velocity\ barrier}}} \\

\noalign{\global\arrayrulewidth 0pt}\arrayrulecolor[HTML]{000000}

\hhline{>{\arrayrulecolor[HTML]{666666}\global\arrayrulewidth=1.5pt}->{\arrayrulecolor[HTML]{666666}\global\arrayrulewidth=1.5pt}->{\arrayrulecolor[HTML]{666666}\global\arrayrulewidth=1.5pt}->{\arrayrulecolor[HTML]{666666}\global\arrayrulewidth=1.5pt}->{\arrayrulecolor[HTML]{666666}\global\arrayrulewidth=1.5pt}->{\arrayrulecolor[HTML]{666666}\global\arrayrulewidth=1.5pt}->{\arrayrulecolor[HTML]{666666}\global\arrayrulewidth=1.5pt}->{\arrayrulecolor[HTML]{666666}\global\arrayrulewidth=1.5pt}->{\arrayrulecolor[HTML]{666666}\global\arrayrulewidth=1.5pt}->{\arrayrulecolor[HTML]{666666}\global\arrayrulewidth=1.5pt}->{\arrayrulecolor[HTML]{666666}\global\arrayrulewidth=1.5pt}->{\arrayrulecolor[HTML]{666666}\global\arrayrulewidth=1.5pt}->{\arrayrulecolor[HTML]{666666}\global\arrayrulewidth=1.5pt}->{\arrayrulecolor[HTML]{666666}\global\arrayrulewidth=1.5pt}-}


\end{longtable}

\arrayrulecolor[HTML]{000000}

\global\setlength{\arrayrulewidth}{\Oldarrayrulewidth}

\global\setlength{\tabcolsep}{\Oldtabcolsep}

\renewcommand*{\arraystretch}{1}

\chapter*{Excluded Structures}\label{excluded-structures-1}
\addcontentsline{toc}{chapter}{Excluded Structures}

\markboth{Excluded Structures}{Excluded Structures}

\global\setlength{\Oldarrayrulewidth}{\arrayrulewidth}

\global\setlength{\Oldtabcolsep}{\tabcolsep}

\setlength{\tabcolsep}{0pt}

\renewcommand*{\arraystretch}{1.5}



\providecommand{\ascline}[3]{\noalign{\global\arrayrulewidth #1}\arrayrulecolor[HTML]{#2}\cline{#3}}

\begin{longtable}[c]{|p{3.20in}|p{1.56in}|p{1.08in}|p{2.44in}|p{1.13in}|p{1.29in}|p{1.97in}|p{2.17in}|p{1.10in}|p{1.46in}|p{0.76in}|p{5.36in}|p{1.46in}|p{4.18in}}

\caption{\label{tbl-keyact}Priority barriers}

\tabularnewline

\hhline{>{\arrayrulecolor[HTML]{666666}\global\arrayrulewidth=1.5pt}->{\arrayrulecolor[HTML]{666666}\global\arrayrulewidth=1.5pt}->{\arrayrulecolor[HTML]{666666}\global\arrayrulewidth=1.5pt}->{\arrayrulecolor[HTML]{666666}\global\arrayrulewidth=1.5pt}->{\arrayrulecolor[HTML]{666666}\global\arrayrulewidth=1.5pt}->{\arrayrulecolor[HTML]{666666}\global\arrayrulewidth=1.5pt}->{\arrayrulecolor[HTML]{666666}\global\arrayrulewidth=1.5pt}->{\arrayrulecolor[HTML]{666666}\global\arrayrulewidth=1.5pt}->{\arrayrulecolor[HTML]{666666}\global\arrayrulewidth=1.5pt}->{\arrayrulecolor[HTML]{666666}\global\arrayrulewidth=1.5pt}->{\arrayrulecolor[HTML]{666666}\global\arrayrulewidth=1.5pt}->{\arrayrulecolor[HTML]{666666}\global\arrayrulewidth=1.5pt}->{\arrayrulecolor[HTML]{666666}\global\arrayrulewidth=1.5pt}->{\arrayrulecolor[HTML]{666666}\global\arrayrulewidth=1.5pt}-}

\multicolumn{1}{>{\cellcolor[HTML]{008270}\raggedright}m{\dimexpr 3.2in+0\tabcolsep}}{\textcolor[HTML]{FFFFFF}{\fontsize{11}{11}\selectfont{ID}}} & \multicolumn{1}{>{\cellcolor[HTML]{008270}\raggedright}m{\dimexpr 1.56in+0\tabcolsep}}{\textcolor[HTML]{FFFFFF}{\fontsize{11}{11}\selectfont{Stream\ name}}} & \multicolumn{1}{>{\cellcolor[HTML]{008270}\raggedleft}m{\dimexpr 1.08in+0\tabcolsep}}{\textcolor[HTML]{FFFFFF}{\fontsize{11}{11}\selectfont{Road\ name}}} & \multicolumn{1}{>{\cellcolor[HTML]{008270}\raggedright}m{\dimexpr 2.44in+0\tabcolsep}}{\textcolor[HTML]{FFFFFF}{\fontsize{11}{11}\selectfont{Owner}}} & \multicolumn{1}{>{\cellcolor[HTML]{008270}\raggedleft}m{\dimexpr 1.13in+0\tabcolsep}}{\textcolor[HTML]{FFFFFF}{\fontsize{11}{11}\selectfont{Proposed\ fix}}} & \multicolumn{1}{>{\cellcolor[HTML]{008270}\raggedright}m{\dimexpr 1.29in+0\tabcolsep}}{\textcolor[HTML]{FFFFFF}{\fontsize{11}{11}\selectfont{Estimated\ cost}}} & \multicolumn{1}{>{\cellcolor[HTML]{008270}\raggedright}m{\dimexpr 1.97in+0\tabcolsep}}{\textcolor[HTML]{FFFFFF}{\fontsize{11}{11}\selectfont{Upstream\ habitat\ Quality}}} & \multicolumn{1}{>{\cellcolor[HTML]{008270}\raggedright}m{\dimexpr 2.17in+0\tabcolsep}}{\textcolor[HTML]{FFFFFF}{\fontsize{11}{11}\selectfont{Barrier\ type}}} & \multicolumn{1}{>{\cellcolor[HTML]{008270}\raggedright}m{\dimexpr 1.1in+0\tabcolsep}}{\textcolor[HTML]{FFFFFF}{\fontsize{11}{11}\selectfont{Habitat\ gain}}} & \multicolumn{1}{>{\cellcolor[HTML]{008270}\raggedright}m{\dimexpr 1.46in+0\tabcolsep}}{\textcolor[HTML]{FFFFFF}{\fontsize{11}{11}\selectfont{Cost\ Benefit\ ratio}}} & \multicolumn{1}{>{\cellcolor[HTML]{008270}\raggedright}m{\dimexpr 0.76in+0\tabcolsep}}{\textcolor[HTML]{FFFFFF}{\fontsize{11}{11}\selectfont{Priority}}} & \multicolumn{1}{>{\cellcolor[HTML]{008270}\raggedright}m{\dimexpr 5.36in+0\tabcolsep}}{\textcolor[HTML]{FFFFFF}{\fontsize{11}{11}\selectfont{Next\ steps}}} & \multicolumn{1}{>{\cellcolor[HTML]{008270}\raggedright}m{\dimexpr 1.46in+0\tabcolsep}}{\textcolor[HTML]{FFFFFF}{\fontsize{11}{11}\selectfont{Reason}}} & \multicolumn{1}{>{\cellcolor[HTML]{008270}\raggedright}m{\dimexpr 4.18in+0\tabcolsep}}{\textcolor[HTML]{FFFFFF}{\fontsize{11}{11}\selectfont{Notes}}} \\

\noalign{\global\arrayrulewidth 0pt}\arrayrulecolor[HTML]{000000}

\hhline{>{\arrayrulecolor[HTML]{666666}\global\arrayrulewidth=1.5pt}->{\arrayrulecolor[HTML]{666666}\global\arrayrulewidth=1.5pt}->{\arrayrulecolor[HTML]{666666}\global\arrayrulewidth=1.5pt}->{\arrayrulecolor[HTML]{666666}\global\arrayrulewidth=1.5pt}->{\arrayrulecolor[HTML]{666666}\global\arrayrulewidth=1.5pt}->{\arrayrulecolor[HTML]{666666}\global\arrayrulewidth=1.5pt}->{\arrayrulecolor[HTML]{666666}\global\arrayrulewidth=1.5pt}->{\arrayrulecolor[HTML]{666666}\global\arrayrulewidth=1.5pt}->{\arrayrulecolor[HTML]{666666}\global\arrayrulewidth=1.5pt}->{\arrayrulecolor[HTML]{666666}\global\arrayrulewidth=1.5pt}->{\arrayrulecolor[HTML]{666666}\global\arrayrulewidth=1.5pt}->{\arrayrulecolor[HTML]{666666}\global\arrayrulewidth=1.5pt}->{\arrayrulecolor[HTML]{666666}\global\arrayrulewidth=1.5pt}->{\arrayrulecolor[HTML]{666666}\global\arrayrulewidth=1.5pt}-}\endfirsthead 

\hhline{>{\arrayrulecolor[HTML]{666666}\global\arrayrulewidth=1.5pt}->{\arrayrulecolor[HTML]{666666}\global\arrayrulewidth=1.5pt}->{\arrayrulecolor[HTML]{666666}\global\arrayrulewidth=1.5pt}->{\arrayrulecolor[HTML]{666666}\global\arrayrulewidth=1.5pt}->{\arrayrulecolor[HTML]{666666}\global\arrayrulewidth=1.5pt}->{\arrayrulecolor[HTML]{666666}\global\arrayrulewidth=1.5pt}->{\arrayrulecolor[HTML]{666666}\global\arrayrulewidth=1.5pt}->{\arrayrulecolor[HTML]{666666}\global\arrayrulewidth=1.5pt}->{\arrayrulecolor[HTML]{666666}\global\arrayrulewidth=1.5pt}->{\arrayrulecolor[HTML]{666666}\global\arrayrulewidth=1.5pt}->{\arrayrulecolor[HTML]{666666}\global\arrayrulewidth=1.5pt}->{\arrayrulecolor[HTML]{666666}\global\arrayrulewidth=1.5pt}->{\arrayrulecolor[HTML]{666666}\global\arrayrulewidth=1.5pt}->{\arrayrulecolor[HTML]{666666}\global\arrayrulewidth=1.5pt}-}

\multicolumn{1}{>{\cellcolor[HTML]{008270}\raggedright}m{\dimexpr 3.2in+0\tabcolsep}}{\textcolor[HTML]{FFFFFF}{\fontsize{11}{11}\selectfont{ID}}} & \multicolumn{1}{>{\cellcolor[HTML]{008270}\raggedright}m{\dimexpr 1.56in+0\tabcolsep}}{\textcolor[HTML]{FFFFFF}{\fontsize{11}{11}\selectfont{Stream\ name}}} & \multicolumn{1}{>{\cellcolor[HTML]{008270}\raggedleft}m{\dimexpr 1.08in+0\tabcolsep}}{\textcolor[HTML]{FFFFFF}{\fontsize{11}{11}\selectfont{Road\ name}}} & \multicolumn{1}{>{\cellcolor[HTML]{008270}\raggedright}m{\dimexpr 2.44in+0\tabcolsep}}{\textcolor[HTML]{FFFFFF}{\fontsize{11}{11}\selectfont{Owner}}} & \multicolumn{1}{>{\cellcolor[HTML]{008270}\raggedleft}m{\dimexpr 1.13in+0\tabcolsep}}{\textcolor[HTML]{FFFFFF}{\fontsize{11}{11}\selectfont{Proposed\ fix}}} & \multicolumn{1}{>{\cellcolor[HTML]{008270}\raggedright}m{\dimexpr 1.29in+0\tabcolsep}}{\textcolor[HTML]{FFFFFF}{\fontsize{11}{11}\selectfont{Estimated\ cost}}} & \multicolumn{1}{>{\cellcolor[HTML]{008270}\raggedright}m{\dimexpr 1.97in+0\tabcolsep}}{\textcolor[HTML]{FFFFFF}{\fontsize{11}{11}\selectfont{Upstream\ habitat\ Quality}}} & \multicolumn{1}{>{\cellcolor[HTML]{008270}\raggedright}m{\dimexpr 2.17in+0\tabcolsep}}{\textcolor[HTML]{FFFFFF}{\fontsize{11}{11}\selectfont{Barrier\ type}}} & \multicolumn{1}{>{\cellcolor[HTML]{008270}\raggedright}m{\dimexpr 1.1in+0\tabcolsep}}{\textcolor[HTML]{FFFFFF}{\fontsize{11}{11}\selectfont{Habitat\ gain}}} & \multicolumn{1}{>{\cellcolor[HTML]{008270}\raggedright}m{\dimexpr 1.46in+0\tabcolsep}}{\textcolor[HTML]{FFFFFF}{\fontsize{11}{11}\selectfont{Cost\ Benefit\ ratio}}} & \multicolumn{1}{>{\cellcolor[HTML]{008270}\raggedright}m{\dimexpr 0.76in+0\tabcolsep}}{\textcolor[HTML]{FFFFFF}{\fontsize{11}{11}\selectfont{Priority}}} & \multicolumn{1}{>{\cellcolor[HTML]{008270}\raggedright}m{\dimexpr 5.36in+0\tabcolsep}}{\textcolor[HTML]{FFFFFF}{\fontsize{11}{11}\selectfont{Next\ steps}}} & \multicolumn{1}{>{\cellcolor[HTML]{008270}\raggedright}m{\dimexpr 1.46in+0\tabcolsep}}{\textcolor[HTML]{FFFFFF}{\fontsize{11}{11}\selectfont{Reason}}} & \multicolumn{1}{>{\cellcolor[HTML]{008270}\raggedright}m{\dimexpr 4.18in+0\tabcolsep}}{\textcolor[HTML]{FFFFFF}{\fontsize{11}{11}\selectfont{Notes}}} \\

\noalign{\global\arrayrulewidth 0pt}\arrayrulecolor[HTML]{000000}

\hhline{>{\arrayrulecolor[HTML]{666666}\global\arrayrulewidth=1.5pt}->{\arrayrulecolor[HTML]{666666}\global\arrayrulewidth=1.5pt}->{\arrayrulecolor[HTML]{666666}\global\arrayrulewidth=1.5pt}->{\arrayrulecolor[HTML]{666666}\global\arrayrulewidth=1.5pt}->{\arrayrulecolor[HTML]{666666}\global\arrayrulewidth=1.5pt}->{\arrayrulecolor[HTML]{666666}\global\arrayrulewidth=1.5pt}->{\arrayrulecolor[HTML]{666666}\global\arrayrulewidth=1.5pt}->{\arrayrulecolor[HTML]{666666}\global\arrayrulewidth=1.5pt}->{\arrayrulecolor[HTML]{666666}\global\arrayrulewidth=1.5pt}->{\arrayrulecolor[HTML]{666666}\global\arrayrulewidth=1.5pt}->{\arrayrulecolor[HTML]{666666}\global\arrayrulewidth=1.5pt}->{\arrayrulecolor[HTML]{666666}\global\arrayrulewidth=1.5pt}->{\arrayrulecolor[HTML]{666666}\global\arrayrulewidth=1.5pt}->{\arrayrulecolor[HTML]{666666}\global\arrayrulewidth=1.5pt}-}\endhead



\multicolumn{1}{>{\raggedright}m{\dimexpr 3.2in+0\tabcolsep}}{\textcolor[HTML]{000000}{\fontsize{11}{11}\selectfont{6cd84598-ee60-48a7-8215-9482dd9d2ed5}}} & \multicolumn{1}{>{\raggedright}m{\dimexpr 1.56in+0\tabcolsep}}{\textcolor[HTML]{000000}{\fontsize{11}{11}\selectfont{MacSweens\ Brook}}} & \multicolumn{1}{>{\raggedleft}m{\dimexpr 1.08in+0\tabcolsep}}{\textcolor[HTML]{000000}{\fontsize{11}{11}\selectfont{}}} & \multicolumn{1}{>{\raggedright}m{\dimexpr 2.44in+0\tabcolsep}}{\textcolor[HTML]{000000}{\fontsize{11}{11}\selectfont{NS\ Department\ of\ Public\ Works}}} & \multicolumn{1}{>{\raggedleft}m{\dimexpr 1.13in+0\tabcolsep}}{\textcolor[HTML]{000000}{\fontsize{11}{11}\selectfont{}}} & \multicolumn{1}{>{\raggedright}m{\dimexpr 1.29in+0\tabcolsep}}{\textcolor[HTML]{000000}{\fontsize{11}{11}\selectfont{N/A}}} & \multicolumn{1}{>{\raggedright}m{\dimexpr 1.97in+0\tabcolsep}}{\textcolor[HTML]{000000}{\fontsize{11}{11}\selectfont{To\ be\ Assessed}}} & \multicolumn{1}{>{\raggedright}m{\dimexpr 2.17in+0\tabcolsep}}{\textcolor[HTML]{000000}{\fontsize{11}{11}\selectfont{Velocity\ and\ Vertical\ Barrier}}} & \multicolumn{1}{>{\raggedright}m{\dimexpr 1.1in+0\tabcolsep}}{\textcolor[HTML]{000000}{\fontsize{11}{11}\selectfont{1.9km}}} & \multicolumn{1}{>{\raggedright}m{\dimexpr 1.46in+0\tabcolsep}}{\textcolor[HTML]{000000}{\fontsize{11}{11}\selectfont{N/A}}} & \multicolumn{1}{>{\raggedright}m{\dimexpr 0.76in+0\tabcolsep}}{\textcolor[HTML]{000000}{\fontsize{11}{11}\selectfont{High}}} & \multicolumn{1}{>{\raggedright}m{\dimexpr 5.36in+0\tabcolsep}}{\textcolor[HTML]{000000}{\fontsize{11}{11}\selectfont{Habitat\ Assessment\ and\ Engagement\ with\ NS\ Department\ of\ Public\ Works}}} & \multicolumn{1}{>{\raggedright}m{\dimexpr 1.46in+0\tabcolsep}}{\textcolor[HTML]{000000}{\fontsize{11}{11}\selectfont{Restore\ Passage}}} & \multicolumn{1}{>{\raggedright}m{\dimexpr 4.18in+0\tabcolsep}}{\textcolor[HTML]{000000}{\fontsize{11}{11}\selectfont{basically\ a\ water\ fall,\ 3\ foot\ drop\ and\ large\ velocity\ barrier}}} \\

\noalign{\global\arrayrulewidth 0pt}\arrayrulecolor[HTML]{000000}

\hhline{>{\arrayrulecolor[HTML]{666666}\global\arrayrulewidth=1.5pt}->{\arrayrulecolor[HTML]{666666}\global\arrayrulewidth=1.5pt}->{\arrayrulecolor[HTML]{666666}\global\arrayrulewidth=1.5pt}->{\arrayrulecolor[HTML]{666666}\global\arrayrulewidth=1.5pt}->{\arrayrulecolor[HTML]{666666}\global\arrayrulewidth=1.5pt}->{\arrayrulecolor[HTML]{666666}\global\arrayrulewidth=1.5pt}->{\arrayrulecolor[HTML]{666666}\global\arrayrulewidth=1.5pt}->{\arrayrulecolor[HTML]{666666}\global\arrayrulewidth=1.5pt}->{\arrayrulecolor[HTML]{666666}\global\arrayrulewidth=1.5pt}->{\arrayrulecolor[HTML]{666666}\global\arrayrulewidth=1.5pt}->{\arrayrulecolor[HTML]{666666}\global\arrayrulewidth=1.5pt}->{\arrayrulecolor[HTML]{666666}\global\arrayrulewidth=1.5pt}->{\arrayrulecolor[HTML]{666666}\global\arrayrulewidth=1.5pt}->{\arrayrulecolor[HTML]{666666}\global\arrayrulewidth=1.5pt}-}


\end{longtable}

\arrayrulecolor[HTML]{000000}

\global\setlength{\arrayrulewidth}{\Oldarrayrulewidth}

\global\setlength{\tabcolsep}{\Oldtabcolsep}

\renewcommand*{\arraystretch}{1}

\chapter*{Data Deficient Structures}\label{data-deficient-structures}
\addcontentsline{toc}{chapter}{Data Deficient Structures}

\markboth{Data Deficient Structures}{Data Deficient Structures}

\global\setlength{\Oldarrayrulewidth}{\arrayrulewidth}

\global\setlength{\Oldtabcolsep}{\tabcolsep}

\setlength{\tabcolsep}{0pt}

\renewcommand*{\arraystretch}{1.5}



\providecommand{\ascline}[3]{\noalign{\global\arrayrulewidth #1}\arrayrulecolor[HTML]{#2}\cline{#3}}

\begin{longtable}[c]{|p{0.44in}|p{1.58in}|p{1.08in}|p{1.70in}|p{2.20in}|p{1.49in}|p{1.81in}|p{2.23in}|p{2.06in}|p{1.24in}|p{1.61in}|p{1.43in}|p{2.43in}|p{1.01in}|p{1.71in}}

\caption{\label{tbl-keyact}Priority barriers}

\tabularnewline

\hhline{>{\arrayrulecolor[HTML]{666666}\global\arrayrulewidth=1.5pt}->{\arrayrulecolor[HTML]{666666}\global\arrayrulewidth=1.5pt}->{\arrayrulecolor[HTML]{666666}\global\arrayrulewidth=1.5pt}->{\arrayrulecolor[HTML]{666666}\global\arrayrulewidth=1.5pt}->{\arrayrulecolor[HTML]{666666}\global\arrayrulewidth=1.5pt}->{\arrayrulecolor[HTML]{666666}\global\arrayrulewidth=1.5pt}->{\arrayrulecolor[HTML]{666666}\global\arrayrulewidth=1.5pt}->{\arrayrulecolor[HTML]{666666}\global\arrayrulewidth=1.5pt}->{\arrayrulecolor[HTML]{666666}\global\arrayrulewidth=1.5pt}->{\arrayrulecolor[HTML]{666666}\global\arrayrulewidth=1.5pt}->{\arrayrulecolor[HTML]{666666}\global\arrayrulewidth=1.5pt}->{\arrayrulecolor[HTML]{666666}\global\arrayrulewidth=1.5pt}->{\arrayrulecolor[HTML]{666666}\global\arrayrulewidth=1.5pt}->{\arrayrulecolor[HTML]{666666}\global\arrayrulewidth=1.5pt}->{\arrayrulecolor[HTML]{666666}\global\arrayrulewidth=1.5pt}-}

\multicolumn{1}{>{\cellcolor[HTML]{008270}\raggedleft}m{\dimexpr 0.44in+0\tabcolsep}}{\textcolor[HTML]{FFFFFF}{\fontsize{11}{11}\selectfont{ID}}} & \multicolumn{1}{>{\cellcolor[HTML]{008270}\raggedleft}m{\dimexpr 1.58in+0\tabcolsep}}{\textcolor[HTML]{FFFFFF}{\fontsize{11}{11}\selectfont{Watercourse\ name}}} & \multicolumn{1}{>{\cellcolor[HTML]{008270}\raggedleft}m{\dimexpr 1.08in+0\tabcolsep}}{\textcolor[HTML]{FFFFFF}{\fontsize{11}{11}\selectfont{Road\ name}}} & \multicolumn{1}{>{\cellcolor[HTML]{008270}\raggedleft}m{\dimexpr 1.7in+0\tabcolsep}}{\textcolor[HTML]{FFFFFF}{\fontsize{11}{11}\selectfont{Location/coordinates}}} & \multicolumn{1}{>{\cellcolor[HTML]{008270}\raggedleft}m{\dimexpr 2.2in+0\tabcolsep}}{\textcolor[HTML]{FFFFFF}{\fontsize{11}{11}\selectfont{Assessment\ step\ completed}}} & \multicolumn{1}{>{\cellcolor[HTML]{008270}\raggedleft}m{\dimexpr 1.49in+0\tabcolsep}}{\textcolor[HTML]{FFFFFF}{\fontsize{11}{11}\selectfont{Passability\ Status}}} & \multicolumn{1}{>{\cellcolor[HTML]{008270}\raggedleft}m{\dimexpr 1.81in+0\tabcolsep}}{\textcolor[HTML]{FFFFFF}{\fontsize{11}{11}\selectfont{Total\ habitat\ gain\ (km)}}} & \multicolumn{1}{>{\cellcolor[HTML]{008270}\raggedleft}m{\dimexpr 2.23in+0\tabcolsep}}{\textcolor[HTML]{FFFFFF}{\fontsize{11}{11}\selectfont{Habitat\ gain\ -\ spawning\ (km)}}} & \multicolumn{1}{>{\cellcolor[HTML]{008270}\raggedleft}m{\dimexpr 2.06in+0\tabcolsep}}{\textcolor[HTML]{FFFFFF}{\fontsize{11}{11}\selectfont{Habitat\ gain\ -\ rearing\ (km)}}} & \multicolumn{1}{>{\cellcolor[HTML]{008270}\raggedleft}m{\dimexpr 1.24in+0\tabcolsep}}{\textcolor[HTML]{FFFFFF}{\fontsize{11}{11}\selectfont{Structure\ type}}} & \multicolumn{1}{>{\cellcolor[HTML]{008270}\raggedleft}m{\dimexpr 1.61in+0\tabcolsep}}{\textcolor[HTML]{FFFFFF}{\fontsize{11}{11}\selectfont{Barrier\ set\ identifier}}} & \multicolumn{1}{>{\cellcolor[HTML]{008270}\raggedleft}m{\dimexpr 1.43in+0\tabcolsep}}{\textcolor[HTML]{FFFFFF}{\fontsize{11}{11}\selectfont{\#\ of\ barrier\ in\ set}}} & \multicolumn{1}{>{\cellcolor[HTML]{008270}\raggedleft}m{\dimexpr 2.43in+0\tabcolsep}}{\textcolor[HTML]{FFFFFF}{\fontsize{11}{11}\selectfont{Number\ of\ downstream\ barriers}}} & \multicolumn{1}{>{\cellcolor[HTML]{008270}\raggedleft}m{\dimexpr 1.01in+0\tabcolsep}}{\textcolor[HTML]{FFFFFF}{\fontsize{11}{11}\selectfont{Next\ steps}}} & \multicolumn{1}{>{\cellcolor[HTML]{008270}\raggedleft}m{\dimexpr 1.71in+0\tabcolsep}}{\textcolor[HTML]{FFFFFF}{\fontsize{11}{11}\selectfont{Comments\ (external)}}} \\

\noalign{\global\arrayrulewidth 0pt}\arrayrulecolor[HTML]{000000}

\hhline{>{\arrayrulecolor[HTML]{666666}\global\arrayrulewidth=1.5pt}->{\arrayrulecolor[HTML]{666666}\global\arrayrulewidth=1.5pt}->{\arrayrulecolor[HTML]{666666}\global\arrayrulewidth=1.5pt}->{\arrayrulecolor[HTML]{666666}\global\arrayrulewidth=1.5pt}->{\arrayrulecolor[HTML]{666666}\global\arrayrulewidth=1.5pt}->{\arrayrulecolor[HTML]{666666}\global\arrayrulewidth=1.5pt}->{\arrayrulecolor[HTML]{666666}\global\arrayrulewidth=1.5pt}->{\arrayrulecolor[HTML]{666666}\global\arrayrulewidth=1.5pt}->{\arrayrulecolor[HTML]{666666}\global\arrayrulewidth=1.5pt}->{\arrayrulecolor[HTML]{666666}\global\arrayrulewidth=1.5pt}->{\arrayrulecolor[HTML]{666666}\global\arrayrulewidth=1.5pt}->{\arrayrulecolor[HTML]{666666}\global\arrayrulewidth=1.5pt}->{\arrayrulecolor[HTML]{666666}\global\arrayrulewidth=1.5pt}->{\arrayrulecolor[HTML]{666666}\global\arrayrulewidth=1.5pt}->{\arrayrulecolor[HTML]{666666}\global\arrayrulewidth=1.5pt}-}\endfirsthead 

\hhline{>{\arrayrulecolor[HTML]{666666}\global\arrayrulewidth=1.5pt}->{\arrayrulecolor[HTML]{666666}\global\arrayrulewidth=1.5pt}->{\arrayrulecolor[HTML]{666666}\global\arrayrulewidth=1.5pt}->{\arrayrulecolor[HTML]{666666}\global\arrayrulewidth=1.5pt}->{\arrayrulecolor[HTML]{666666}\global\arrayrulewidth=1.5pt}->{\arrayrulecolor[HTML]{666666}\global\arrayrulewidth=1.5pt}->{\arrayrulecolor[HTML]{666666}\global\arrayrulewidth=1.5pt}->{\arrayrulecolor[HTML]{666666}\global\arrayrulewidth=1.5pt}->{\arrayrulecolor[HTML]{666666}\global\arrayrulewidth=1.5pt}->{\arrayrulecolor[HTML]{666666}\global\arrayrulewidth=1.5pt}->{\arrayrulecolor[HTML]{666666}\global\arrayrulewidth=1.5pt}->{\arrayrulecolor[HTML]{666666}\global\arrayrulewidth=1.5pt}->{\arrayrulecolor[HTML]{666666}\global\arrayrulewidth=1.5pt}->{\arrayrulecolor[HTML]{666666}\global\arrayrulewidth=1.5pt}->{\arrayrulecolor[HTML]{666666}\global\arrayrulewidth=1.5pt}-}

\multicolumn{1}{>{\cellcolor[HTML]{008270}\raggedleft}m{\dimexpr 0.44in+0\tabcolsep}}{\textcolor[HTML]{FFFFFF}{\fontsize{11}{11}\selectfont{ID}}} & \multicolumn{1}{>{\cellcolor[HTML]{008270}\raggedleft}m{\dimexpr 1.58in+0\tabcolsep}}{\textcolor[HTML]{FFFFFF}{\fontsize{11}{11}\selectfont{Watercourse\ name}}} & \multicolumn{1}{>{\cellcolor[HTML]{008270}\raggedleft}m{\dimexpr 1.08in+0\tabcolsep}}{\textcolor[HTML]{FFFFFF}{\fontsize{11}{11}\selectfont{Road\ name}}} & \multicolumn{1}{>{\cellcolor[HTML]{008270}\raggedleft}m{\dimexpr 1.7in+0\tabcolsep}}{\textcolor[HTML]{FFFFFF}{\fontsize{11}{11}\selectfont{Location/coordinates}}} & \multicolumn{1}{>{\cellcolor[HTML]{008270}\raggedleft}m{\dimexpr 2.2in+0\tabcolsep}}{\textcolor[HTML]{FFFFFF}{\fontsize{11}{11}\selectfont{Assessment\ step\ completed}}} & \multicolumn{1}{>{\cellcolor[HTML]{008270}\raggedleft}m{\dimexpr 1.49in+0\tabcolsep}}{\textcolor[HTML]{FFFFFF}{\fontsize{11}{11}\selectfont{Passability\ Status}}} & \multicolumn{1}{>{\cellcolor[HTML]{008270}\raggedleft}m{\dimexpr 1.81in+0\tabcolsep}}{\textcolor[HTML]{FFFFFF}{\fontsize{11}{11}\selectfont{Total\ habitat\ gain\ (km)}}} & \multicolumn{1}{>{\cellcolor[HTML]{008270}\raggedleft}m{\dimexpr 2.23in+0\tabcolsep}}{\textcolor[HTML]{FFFFFF}{\fontsize{11}{11}\selectfont{Habitat\ gain\ -\ spawning\ (km)}}} & \multicolumn{1}{>{\cellcolor[HTML]{008270}\raggedleft}m{\dimexpr 2.06in+0\tabcolsep}}{\textcolor[HTML]{FFFFFF}{\fontsize{11}{11}\selectfont{Habitat\ gain\ -\ rearing\ (km)}}} & \multicolumn{1}{>{\cellcolor[HTML]{008270}\raggedleft}m{\dimexpr 1.24in+0\tabcolsep}}{\textcolor[HTML]{FFFFFF}{\fontsize{11}{11}\selectfont{Structure\ type}}} & \multicolumn{1}{>{\cellcolor[HTML]{008270}\raggedleft}m{\dimexpr 1.61in+0\tabcolsep}}{\textcolor[HTML]{FFFFFF}{\fontsize{11}{11}\selectfont{Barrier\ set\ identifier}}} & \multicolumn{1}{>{\cellcolor[HTML]{008270}\raggedleft}m{\dimexpr 1.43in+0\tabcolsep}}{\textcolor[HTML]{FFFFFF}{\fontsize{11}{11}\selectfont{\#\ of\ barrier\ in\ set}}} & \multicolumn{1}{>{\cellcolor[HTML]{008270}\raggedleft}m{\dimexpr 2.43in+0\tabcolsep}}{\textcolor[HTML]{FFFFFF}{\fontsize{11}{11}\selectfont{Number\ of\ downstream\ barriers}}} & \multicolumn{1}{>{\cellcolor[HTML]{008270}\raggedleft}m{\dimexpr 1.01in+0\tabcolsep}}{\textcolor[HTML]{FFFFFF}{\fontsize{11}{11}\selectfont{Next\ steps}}} & \multicolumn{1}{>{\cellcolor[HTML]{008270}\raggedleft}m{\dimexpr 1.71in+0\tabcolsep}}{\textcolor[HTML]{FFFFFF}{\fontsize{11}{11}\selectfont{Comments\ (external)}}} \\

\noalign{\global\arrayrulewidth 0pt}\arrayrulecolor[HTML]{000000}

\hhline{>{\arrayrulecolor[HTML]{666666}\global\arrayrulewidth=1.5pt}->{\arrayrulecolor[HTML]{666666}\global\arrayrulewidth=1.5pt}->{\arrayrulecolor[HTML]{666666}\global\arrayrulewidth=1.5pt}->{\arrayrulecolor[HTML]{666666}\global\arrayrulewidth=1.5pt}->{\arrayrulecolor[HTML]{666666}\global\arrayrulewidth=1.5pt}->{\arrayrulecolor[HTML]{666666}\global\arrayrulewidth=1.5pt}->{\arrayrulecolor[HTML]{666666}\global\arrayrulewidth=1.5pt}->{\arrayrulecolor[HTML]{666666}\global\arrayrulewidth=1.5pt}->{\arrayrulecolor[HTML]{666666}\global\arrayrulewidth=1.5pt}->{\arrayrulecolor[HTML]{666666}\global\arrayrulewidth=1.5pt}->{\arrayrulecolor[HTML]{666666}\global\arrayrulewidth=1.5pt}->{\arrayrulecolor[HTML]{666666}\global\arrayrulewidth=1.5pt}->{\arrayrulecolor[HTML]{666666}\global\arrayrulewidth=1.5pt}->{\arrayrulecolor[HTML]{666666}\global\arrayrulewidth=1.5pt}->{\arrayrulecolor[HTML]{666666}\global\arrayrulewidth=1.5pt}-}\endhead


\end{longtable}

\arrayrulecolor[HTML]{000000}

\global\setlength{\arrayrulewidth}{\Oldarrayrulewidth}

\global\setlength{\tabcolsep}{\Oldtabcolsep}

\renewcommand*{\arraystretch}{1}




\end{document}
